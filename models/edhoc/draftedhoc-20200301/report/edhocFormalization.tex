% !TEX root =  main.tex
The \mEdhoc{} \mSpec{} \cite{our-analysis-selander-lake-edhoc-01} claims
that \mEdhoc{} satisfies many security properties, but these are imprecisely
expressed and motivated.
%
In particular, there is no coherent adversary model.
%
It is therefore not clear in which context properties should be verified.
%
We resolve this by clearly specifying an adversary model, in which we can verify
properties.
%Therefore, we explore in which adversary models we can prove properties and
%incorporate those models into the properties themselves.
%

In this section, we describe our approach and modeling choices towards
formalizing the \mEdhoc{} protocol and the properties we verify.
%

\subsection{Adversary Model}\label{sec:threat-model}
We verify \mEdhoc{} in a symbolic Dolev-Yao model, with idealized
cryptographic primitives, e.g, encrypted messages can only be
decrypted using the key, no hash collisions exist etc.
%
The adversary controls the
communication channel, and can interact with an unbounded number of sessions
of the protocol, dropping, injecting and modifying messages at their liking.
%

In addition to the basic Dolev-Yao model, we also consider two more adversary
capabilities, namely long-term key reveal and ephemeral DH half-key reveal.
%
The long-term key reveal models an adversary compromising a party $A$'s
long-term key, and we denote this type of event by \mRevLTK$(A)$.
%
The event \mRevEph$(A, k)$ represents the reveal of the ephemeral DH half-key
used by party $A$ while establishing session key material $k$.
%
This two capabilities model the possibility for long-term keys to be stored in a
hardened storage, a Trusted Execution Environment (TEE), whereas ephemeral keys
may be stored in a less secure part of a device.
%

\subsection{Formalization of Properties}
\label{sec:desired-properties}
%Our model is based on event traces and we define the
%properties in terms of relations between timestamped events.
%
We use the \mTamarin{} verification
tool~\cite{DBLP:conf/cav/MeierSCB13} to encode the model and verify properties.
%
This tool uses a fragment of temporal first order logic to reason about
events and knowledge of the parties and of the adversary.
%
For conciseness we use a slightly different syntax than
that used by \mTamarin{}, but which has a direct mapping to \mTamarin{}'s logic.
%

Event types are predicates over global states of system execution.
%
Let $E$ be an event type and let $t$ be a timestamp indicating a point in a
trace.
%
Then $E^{t}(p_i)_{i\in\mathbb{N}}$ denotes an event of type $E$ associated with a sequence of parameters
$(p_i)_{i\in\mathbb{N}}$ at time $t$ in that trace.
%
In general, more than one event may have the same timestamp and hence
timestamps form a quasi order, which we denote by $t_1 \lessdot t_2$ when $t_1$
is before $t_2$ in a trace.
%
We define $\doteq$ analogously.
%
However, two events of the same type cannot have the same timestamp, so
$t_1 \doteq t_2$ implies $E^{t_1} = E^{t_2}$.
%
This corresponds to \mTamarin{}'s notation $E(p_i)_{i\in\mathbb{N}}@t$ for action facts.
%
%
%Other first order logic connectives and parenthesis have their standard
%interpretation.
%
%A special event type $\mK$ represents adversary knowledge.
%
The event $\mK^t(p)$ denotes that the adversary knows a parameter $p$ at
time $t$.
%
%Intuitively, events of type $\mK$ evaluate to true when the parameter is in
%the closure of the
%parameters the adversary observed from interacting with parties using the
%protocol, under the operations of the Dolev-Yao message deduction rules and
%the advanced adversary capabilities up until time $t$.
%

%In reference to Figure~\ref{fig:edhocFramework}, 
% KARL: I appreciate the better English word choice "begun", but the event is
% called I_S, where S is intended to associate to "Start".
An initiator $I$ considers the
protocol run started when it sends message \mMsgone{} (event type \mIStart)
and the run completed after sending message \mMsgthree{} (event type
\mIComplete).
%
%We denote these event types by \mIStart{} and \mIComplete{} respectively.
%
Similarly, a responder $R$ considers the run begun upon receiving
\mMsgone{} (event type \mRStart), and completed upon receiving \mMsgthree{}
(event type \mRComplete). \\
%
%We denote these event types by \mRStart{} and \mRComplete{} respectively.
%
%As is customary, the properties we show are guaranteed to hold for a party when
%the party completes their part of the protocol run.\\
%

%----------------------------------------------------------------------- PFS
%\subsubsection{Secrecy}
\runhead{Perfect Forward Secrecy (PFS)}
\label{sec:secrecy}
%We prove a form of \emph{Perfect Forward Secrecy} (PFS).
%
Informally, PFS captures the idea that session key material remains secret
even if a long-term key leaks in the future.
%
%We formalize PFS for session key material $k$ with respect to the strongest
%adversary model in which we could successfully verify it.
%
We define PFS as follows: %through the trace predicate
%
\begin{align*}
    \mPredPfs \triangleq
    \forall I, R, k, t_2, t_3\mLogicDot
    \mK^{t_3}(k) & \land (\mIComplete^{t_2}(I, R, k)\, \lor\,  \mRComplete^{t_2}(I, R, k))
    \rightarrow\\
    &(\exists t_1\mLogicDot \mRevLTK^{t_1}(I) \land t_1 \lessdot t_2)
    \lor (\exists t_1\mLogicDot \mRevLTK^{t_1}(R) \land t_1 \lessdot t_2)\\
    \lor&(\exists t_1\mLogicDot \mRevEph^{t_1}(R, k))
    \lor(\exists t_1\mLogicDot \mRevEph^{t_1}(I, k)).
\end{align*}
%
% \knote{Note that our Tamarin model only reveals ephemeral keys first when a party
%     completes the run. So even if the formula above checks for ephemeral keys to
%     be revealed as soon as they are created, our Tamarin model will never
%     generate any such traces. That is, we do not actually verify the above
%     formula truly.  How to deal with that?
%     I explain these things in the Tamarin model below, but please check that
%     this sufficiently explicit?
% }
The first parameter $I$ of the \mIComplete{} event represents the
initiator's identity,
and the second, $R$, represents that $I$ believes $R$ to be playing
the responder role.
%
The third parameter, $k$, is the established session key.
%
The parameters of the \mRComplete{} event are defined analogously.
%
Specifically, the first parameter of \mRComplete{} represents the identity of
whom $R$ believes is playing the initiator role.
%
%The meaning of remaining variables and types should be clear from previous
%descriptions.
%
The essence of the definition is that an adversary only knows $k$ if they
compromised one of the
parties long-term keys before that party completed the run, or if the adversary
compromised any of the ephemeral DH half-keys at any time after a party starts
its protocol run.
% KARL: I don't understand the description below, especially sins it introduces
% a session concept that is not yet defined.  Also, talking about "any other
% information" is a very broad, sounding almost like information theoretic
% security.

% The property states that for any session between $I$ and $R$ which establishes a session key $k$, if the adversary comes to know $k$, then it is because one (or more) of the following happened: the long-term key of $I$ or that of $R$ was obtained by the adversary before the completion of the session, or the ephemeral key used by $I$ or by $R$ was compromised anytime during the session. Any other information obtained by the adversary (or even this same information, obtained at a point of time not as above) does not help them gain access to the established session key. For example, with reference to the informal description above, compromising the long-term key of either party \emph{after} the completion of the session does not give the adversary access to the established session key.

One way the definition slightly differs from the corresponding \mTamarin{} lemma
is that \mTamarin{} does not allow a disjunction on the left hand side of an
implication in a universally quantified formula.
%
In the lemma, therefore, instead of the disjunction
$\mIComplete^{t_2}(I, R, k)\, \lor\,  \mRComplete^{t_2}(I, R, k)$,
we use a single action parametrized by $I$, $R$, and $k$ to signify that
\emph{either} party has completed their role. \\
%
%This is depicted in Figure~\ref{fig:pfs}.
%%
%\begin{figure}[h!]
%    \begin{center}
%        \tikzset{>=latex}
%        \begin{tikzpicture}
%            % All units in pt
%            \pgfmathsetmacro{\LOFFS}{-20} % Offset between horizontal lines
%            \pgfmathsetmacro{\IS}{50}    % X-coord
%            \pgfmathsetmacro{\IC}{200}    % X-coord
%            \pgfmathsetmacro{\RS}{80}    % X-coord
%            \pgfmathsetmacro{\RC}{230}    % X-coord
%            \pgfmathsetmacro{\LTKI}{200}  % X-coord
%            \pgfmathsetmacro{\LTKR}{230}  % X-coord
%            % I's session ------------------------------------------------
%            % Baseline
%            \draw [->] (0pt,0pt) -- (300pt,0pt);
%            % First event
%            \draw (\IS pt,3pt) node[above] {\mIStart{}};
%            \draw (\IS pt,3pt) -- (\IS pt,-3pt);
%            % Second event
%            \draw (\IC pt,3pt) node[above] {\mIComplete{}};
%            \draw (\IC pt,3pt) -- (\IC pt,-3pt);
%            % Thick session span line
%            \draw [line width=2pt] (\IS pt,0pt) -- (\IC pt,0pt);
%            % R's session ------------------------------------------------
%            % Baseline
%            \draw [->] (0pt,\LOFFS + 0pt) -- (300pt,\LOFFS + 0pt);
%            % First event
%            \draw (\RS pt,\LOFFS + 3pt) node[above] {\mRStart{}};
%            \draw (\RS pt,\LOFFS + 3pt) -- (\RS pt,\LOFFS - 3pt);
%            % Second event
%            \draw (\RC pt,\LOFFS + 3pt) node[above] {\mRComplete{}};
%            \draw (\RC pt,\LOFFS + 3pt) -- (\RC pt,\LOFFS - 3pt);
%            % Thick session span line
%            \draw [line width=2pt] (\RS pt,\LOFFS + 0pt) -- (\RC pt,\LOFFS + 0pt);
%            % LTKRev(I,R)  -----------------------------------------------
%            % Baseline
%            \draw [->] (0pt,2*\LOFFS + 0pt) -- (300pt,2*\LOFFS + 0pt);
%            % Attack events
%            \draw (\LTKR pt,2*\LOFFS + 3pt) node[above] {\mRevLTK$(I)$, \mRevLTK$(R)$};
%            \draw (\LTKR pt,2*\LOFFS + 3pt) -- (\LTKR pt,2*\LOFFS - 3pt);
%        \end{tikzpicture}
%        \caption{Perfect Forward Secrecy: the earliest the adversary can use its
%                 advanced capabilities while secrecy is still maintained.
%                 Ephemeral DH half-keys can never be revealed.}
%        \label{fig:pfs}
%    \end{center}
%\end{figure}\\

%-------------------------------------------------------------------- InjAgree
%\subsubsection{Authentication}
\runhead{Authentication}
\label{sec:authenticationDef}
We prove two different flavors of authentication, the first being classical
\emph{injective agreement} following Lowe~\cite{DBLP:conf/csfw/Lowe97a}, and
the second being an implicit agreement property.
%
Informally, injective agreement guarantees to an initiator $I$ that whenever
$I$ completes a run ostensibly with a responder $R$,
then $R$ has been engaged in the protocol as a responder,
and this run of $I$ corresponds to a unique run of $R$.
%
In addition, the property guarantees to $I$ that the two parties agree on a set
$S$ of parameters associated with the run, including, in particular, the
session key material $k$.
%
However, we will treat $k$ separately for clarity.
%
On completion, $I$ knows that $R$ has access to the session key material.
%
The corresponding property for $R$ is analogous.
%

Traditionally, the event types used to describe injective agreement are called
\emph{Running} and \emph{Commit}, but to harmonize the presentations of
authentication and secrecy in this section, we refer to these event types as
\mIStart{} and \mIComplete{} respectively for the initiator, and
\mRStart{} and \mRComplete{} for the responder.
%
For the initiator role we define injective agreement as follows:
\begin{align*}
    \mPredInjI& \triangleq
    \forall I, R, k, S, t_2\mLogicDot \mIComplete^{t_2}(I, R, k, S)
    \rightarrow\\
    &(\exists t_1\mLogicDot \mRStart^{t_1}(R, k, S) \land t_1 \lessdot t_2)
    \land\ (\forall I' R' t_1' \mLogicDot \mIComplete^{t_1'}(I' , R', k, S)
        \rightarrow t_1' \doteq t_1) \\
    &\lor(\exists t_1\mLogicDot \mRevLTK^{t_1}(R) \land t_1 \lessdot t_2).
\end{align*}
%
The property claims that for an initiator $I$, either the injective agreement
property as described above holds, or the long-term key of the believed
responder $R$ has been compromised before $I$ completed its role.
%
Note that this places no restrictions on the reveals of the ephemeral
DH half-keys, or the reveal of the initiator's long-term key.
%
For the responder role we define the analogous property by
\begin{align*}
    \mPredInjR& \triangleq\
    \forall I, R, k, S, t_2\mLogicDot \mRComplete^{t_2}(I, R, k, S)
    \rightarrow\\
    &(\exists t_1\mLogicDot \mIStart^{t_1}(I, R, k, S) \land t_1 \lessdot t_2)
    \land (\forall I' R' t_1' \mLogicDot \mRComplete^{t_1'}(I' , R', k, S)
        \rightarrow t_1' \doteq t_1)\\
    &\lor(\exists t_1\mLogicDot \mRevLTK^{t_1}(I) \land t_1 \lessdot t_2).
\end{align*}
%
%\begin{figure}[h!]
%    \begin{center}
%        \tikzset{>=latex}
%        \begin{tikzpicture}
%            % All units in pt
%            \pgfmathsetmacro{\LOFFS}{-20} % Offset between horizontal lines
%            \pgfmathsetmacro{\IS}{70}     % X-coord
%            \pgfmathsetmacro{\IC}{170}    % X-coord
%            \pgfmathsetmacro{\RS}{130}    % X-coord
%            \pgfmathsetmacro{\RC}{240}    % X-coord
%            \pgfmathsetmacro{\LTKI}{200}  % X-coord
%            \pgfmathsetmacro{\LTKR}{230}  % X-coord
%            \draw (-10 pt,10pt) node [above] {\mPredInjI};
%            % I's session ------------------------------------------------
%            % Baseline
%            \draw [->] (0pt,0pt) -- (300pt,0pt);
%            % First event
%            \draw (\IS pt,3pt) node[above] {\mIStart{}};
%            \draw (\IS pt,3pt) -- (\IS pt,-3pt);
%            % Second event
%            \draw (\IC pt,3pt) node[above] {\mIComplete{}};
%            \draw (\IC pt,3pt) -- (\IC pt,-3pt);
%            % Thick session span line
%            \draw [line width=2pt] (\IS pt,0pt) -- (\IC pt,0pt);
%            % R's session ------------------------------------------------
%            % Baseline
%            \draw [->] (0pt,\LOFFS + 0pt) -- (300pt,\LOFFS + 0pt);
%            % First event
%            \draw (\RS pt,\LOFFS + 3pt) node[above] {\mRStart{}};
%            \draw (\RS pt,\LOFFS + 3pt) -- (\RS pt,\LOFFS - 3pt);
%            % Second event
%            \draw (\RC pt,\LOFFS + 3pt) node[above] {\mRComplete{}};
%            \draw (\RC pt,\LOFFS + 3pt) -- (\RC pt,\LOFFS - 3pt);
%            % Thick session span line
%            \draw [line width=2pt] (\RS pt,\LOFFS + 0pt) -- (\RC pt,\LOFFS + 0pt);
%            % LTKRev(I,R)  -----------------------------------------------
%            % Baseline
%            \draw [->] (0pt,2*\LOFFS + 0pt) -- (300pt,2*\LOFFS + 0pt);
%            % Attack events
%            \draw (\IC pt,2*\LOFFS + 3pt) node[above] {\mRevLTK$(R)$,\mRevEph$(I,k)$};
%            \draw (\IC pt,2*\LOFFS + 3pt) -- (\IC pt,2*\LOFFS - 3pt);
%            \draw (\IS - 50 pt,2*\LOFFS + 3pt) node[above] {\mRevLTK$(I)$};
%            \draw (\IS - 50 pt,2*\LOFFS + 3pt) -- (\IS - 50 pt,2*\LOFFS - 3pt);
%            \draw (\RC pt,2*\LOFFS + 3pt) node[above] {\mRevEph$(R, k)$};
%            \draw (\RC pt,2*\LOFFS + 3pt) -- (\RC pt,2*\LOFFS - 3pt);
%            %\draw (\RS pt,2*\LOFFS + 3pt) node[above] {\mRevEph$(R, k)$};
%            %\draw (\RS pt,2*\LOFFS + 3pt) -- (\RS pt,2*\LOFFS - 3pt);
%            % InjAgree for R =============================================
%            \pgfmathsetmacro{\IS}{70}     % X-coord
%            \pgfmathsetmacro{\IC}{170}    % X-coord
%            \pgfmathsetmacro{\RS}{130}    % X-coord
%            \pgfmathsetmacro{\RC}{240}    % X-coord
%            \pgfmathsetmacro{\LTKI}{200}  % X-coord
%            \pgfmathsetmacro{\LTKR}{230}  % X-coord
%            \draw (-10 pt,4*\LOFFS+6pt) node [above] {\mPredInjR};
%            % I's session ------------------------------------------------
%            % Baseline
%            \draw [->] (0pt,4*\LOFFS + 0pt) -- (300pt,4*\LOFFS + 0pt);
%            % First event
%            \draw (\IS pt,4*\LOFFS + 3pt) node[above] {\mIStart{}};
%            \draw (\IS pt,4*\LOFFS + 3pt) -- (\IS pt,4*\LOFFS + -3pt);
%            % Second event
%            \draw (\IC pt,4*\LOFFS + 3pt) node[above] {\mIComplete{}};
%            \draw (\IC pt,4*\LOFFS + 3pt) -- (\IC pt,4*\LOFFS + -3pt);
%            % Thick session span line
%            \draw [line width=2pt] (\IS pt,4*\LOFFS + 0pt) -- (\IC pt,4*\LOFFS + 0pt);
%            % R's session ------------------------------------------------
%            % Baseline
%            \draw [->] (0pt,5*\LOFFS + 0pt) -- (300pt,5*\LOFFS + 0pt);
%            % First event
%            \draw (\RS pt,5*\LOFFS + 3pt) node[above] {\mRStart{}};
%            \draw (\RS pt,5*\LOFFS + 3pt) -- (\RS pt,5*\LOFFS - 3pt);
%            % Second event
%            \draw (\RC pt,5*\LOFFS + 3pt) node[above] {\mRComplete{}};
%            \draw (\RC pt,5*\LOFFS + 3pt) -- (\RC pt,5*\LOFFS - 3pt);
%            % Thick session span line
%            \draw [line width=2pt] (\RS pt,5*\LOFFS + 0pt) -- (\RC pt,5*\LOFFS + 0pt);
%            % LTKRev(I,R)  -----------------------------------------------
%            % Baseline
%            \draw [->] (0pt,6*\LOFFS + 0pt) -- (300pt,6*\LOFFS + 0pt);
%            % Attack events
%            \draw (\RC pt,6*\LOFFS + 3pt) node[above] {\mRevLTK$(I)$,\mRevEph$(R,k)$};
%            \draw (\RC pt,6*\LOFFS + 3pt) -- (\RC pt,6*\LOFFS - 3pt);
%            \draw (\IS - 50 pt,6*\LOFFS + 3pt) node[above] {\mRevLTK$(R)$};
%            \draw (\IS - 50 pt,6*\LOFFS + 3pt) -- (\IS - 50 pt,6*\LOFFS - 3pt);
%            \draw (\IC pt,6*\LOFFS + 3pt) node[above] {\mRevEph$(I, k)$};
%            \draw (\IC pt,6*\LOFFS + 3pt) -- (\IC pt,6*\LOFFS - 3pt);
%            %\draw (\RS pt,6*\LOFFS + 3pt) node[above] {\mRevEph$(R, k)$};
%            %\draw (\RS pt,6*\LOFFS + 3pt) -- (\RS pt,6*\LOFFS - 3pt);
%        \end{tikzpicture}
%        \caption{Injective agreement guarantee: the earliest the
%        adversary can use its advanced capabilities such that the properties
%        are still proven to be maintained.}
%        \label{fig:injAgree}
%    \end{center}
%\end{figure}
%%

%------------------------------------------------------------- Implicit auth
Unlike PFS, not all \mEdhoc{} methods enjoy the injective agreement property.
%
Hence we show for all methods a form of \emph{implicit agreement} on all the
parameters mentioned above.
%
We take inspiration from the computational model definitions of implicit
authentication, proposed by Guilhem~et~al.~\cite{DBLP:conf/csfw/GuilhemFW20}, to
modify classical injective agreement into an implicit property.
%
A small but important difference between our definition and theirs, is that
they focus on
authenticating a key and related identities, whereas we extend the more general
concept of agreeing on a set of parameters, inspired by the idea of injective
agreement~\cite{DBLP:conf/csfw/Lowe97a}.
%
We use the term \emph{implicit} in this context to denote that a party $A$
assumes that any other party $B$ who knows the session key material must be the
authenticated party, and that $B$ (if honest) will also agree on a set $S$ of
parameters computed by the protocol, one of which is the session key $k$.
%
If implicit agreement holds for both roles, upon completion, $A$ is guaranteed
that $B$ has been or is engaged in exactly one protocol run with $A$ in the
opposite role, and that $B$ has been or will be able to agree on $S$.
%
The main difference to injective agreement is that $A$ can conclude that if
$A$ sends the last message and this reaches $B$, then $A$ and $B$ have agreed
on $S$.
%
In contrast to injective agreement, this property is symmetric with respect to
the initiator and responder role.
%

%In particular, because the property only reasons about completed and
%hypothetically completed protocol runs, it can reason about parameters that the
%responder learns only after receiving the last message.
%
While almost full explicit key authentication, as defined by
Guilhem~et~al.~\cite{DBLP:conf/csfw/GuilhemFW20}, is a similar property, our
definition does not require key confirmation, so our definition is closer to
their definition of implicit authentication.
%
In the \mTamarin{} lemma we split the property into one lemma for the
$I$ and one for $R$.
%
We here show only the definition for $I$ because it is symmetric to the one
for $R$.
%
The definition for $I$ is
\begin{align*}
    \mPredImpI& \triangleq
    \forall I, R, k, S, t_1\mLogicDot \mIComplete^{t_1}(I, R, k, S)
    \rightarrow\\
      &(\forall I', R', S', t_2\mLogicDot \mRComplete^{t_2}(I', R', k, S') \rightarrow
             (I=I' \land R=R' \land S=S'))\\
      \land &(\forall I', R', S', t_1'\mLogicDot
        (\mIComplete^{t_1'}(I', R', k, S') \rightarrow t_1' \doteq t_1
        )\\
    &\lor(\exists t_0\mLogicDot \mRevLTK^{t_0}(R) \land t_0 \lessdot t_1)
    \lor(\exists t_0\mLogicDot \mRevEph^{t_0}(R, k))
    \lor(\exists t_0\mLogicDot \mRevEph^{t_0}(I, k)).
\end{align*}
%

For implicit agreement to hold for the initiator $I$, the ephemeral DH half-keys
can never be revealed.
%
However, the responder $R$'s long-term key can be revealed after $I$ completes
its run.
%
The initiator's long-term key can also be revealed at any time without affecting
$I$'s guarantee.\\
%
%\begin{figure}[h!]
%    \begin{center}
%        \tikzset{>=latex}
%        \begin{tikzpicture}
%            % All units in pt
%            \pgfmathsetmacro{\LOFFS}{-20} % Offset between horizontal lines
%            \pgfmathsetmacro{\IS}{80}     % X-coord
%            \pgfmathsetmacro{\IC}{200}    % X-coord
%            \pgfmathsetmacro{\RS}{120}    % X-coord
%            \pgfmathsetmacro{\RC}{230}    % X-coord
%            \pgfmathsetmacro{\LTKI}{200}  % X-coord
%            \pgfmathsetmacro{\LTKR}{230}  % X-coord
%            % I's session ------------------------------------------------
%            % Baseline
%            \draw [->] (0pt,0pt) -- (300pt,0pt);
%            % First event
%            \draw (\IS pt,3pt) node[above] {\mIStart{}};
%            \draw (\IS pt,3pt) -- (\IS pt,-3pt);
%            % Second event
%            \draw (\IC pt,3pt) node[above] {\mIComplete{}};
%            \draw (\IC pt,3pt) -- (\IC pt,-3pt);
%            % Thick session span line
%            \draw [line width=2pt] (\IS pt,0pt) -- (\IC pt,0pt);
%            % R's session ------------------------------------------------
%            % Baseline
%            \draw [->] (0pt,\LOFFS + 0pt) -- (300pt,\LOFFS + 0pt);
%            % First event
%            \draw (\RS pt,\LOFFS + 3pt) node[above] {\mRStart{}};
%            \draw (\RS pt,\LOFFS + 3pt) -- (\RS pt,\LOFFS - 3pt);
%            % Second event
%            \draw (\RC pt,\LOFFS + 3pt) node[above] {\mRComplete{}};
%            \draw (\RC pt,\LOFFS + 3pt) -- (\RC pt,\LOFFS - 3pt);
%            % Thick session span line
%            \draw [line width=2pt] (\RS pt,\LOFFS + 0pt) -- (\RC pt,\LOFFS + 0pt);
%            % LTKRev(I,R)  -----------------------------------------------
%            % Baseline
%            \draw [->] (0pt,2*\LOFFS + 0pt) -- (300pt,2*\LOFFS + 0pt);
%            % Attack events
%            \draw (\IC pt,2*\LOFFS + 3pt) node[above] {\mRevLTK$(R)$};
%            \draw (\IC pt,2*\LOFFS + 3pt) -- (\IC pt,2*\LOFFS - 3pt);
%            \draw (\IS - 50 pt,2*\LOFFS + 3pt) node[above] {\mRevLTK$(I)$};
%            \draw (\IS - 50 pt,2*\LOFFS + 3pt) -- (\IS - 50 pt,2*\LOFFS - 3pt);
%        \end{tikzpicture}
%        \caption{Implicit agreement guarantee for $I$: the earliest the
%        adversary can use its advanced capabilities. Ephemeral keys can never be
%        revealed.}
%        \label{fig:impAgreeI}
%    \end{center}
%\end{figure}
%%

%

%------------------------------------------------------- Agreed parameters
\runhead{Agreed Parameters}
\label{sec:agreedParams}
%\subsubsection{Agreed Parameters}
The initiator $I$ gets injective and implicit agreement guarantees on the
following partial set $S_P$ of parameters:
\begin{itemize}
    \item the roles played by itself and its peer,
    \item responder identity,
    \item session key material (which varies depending on \mEdhoc{} method),
    \item context identifiers \mCi{} and \mCr{}, and
    \item cipher suites \mSuites{}.
\end{itemize}
%
Because of \mEdhoc{} aiming to provide identity protection for $I$, there is no
injective agreement guarantee for $I$ that $R$ agrees on the initiator's
identity.
%
For the same reason, there is no such guarantee for $I$ with respect to
the \mGiy{} part of the session key material when $I$ uses a \mStat{}-based method.
%
There is, however, an implicit agreement guarantee for $I$ that $R$ agrees on
$I$'s identity and the full session key material.
%
Since $R$ completes after $I$, $R$ can get injective agreement guarantees on
more parameters, namely the initator's identity and the full session key
material for all methods.\\
%
Let the full set of parameters be $S_F = S_P \cup \{I, \mGiy\}$ when \mGiy{} is
part of the session key material and $S_P \cup \{I\}$ otherwise.\\
%

%------------------------------------------------------- Implied properties
\runhead{Inferred Properties}
%\subsubsection{Subsumed Properties}
Given the secrecy and agreement properties, and parameter set $S_P$ defined
above, other properties can be inferred to hold in our threat model
under.
%
Protocols where a party does not get confirmation on that their peer knows the
session key material are known to be at risk to
\emph{Key-Compromise Impersonation (KCI)}
attacks~\cite{DBLP:conf/ima/Blake-WilsonJM97}.
%
Attacks in this class allow an adversary in possession of a party $A$'s secret
long-term key to coerce $A$ to complete a
protocol run believing it authenticated a certain peer $B$, but where $B$ did
not engage with $A$ in a run.
%
Because both our notions of agreement above ensure agreement on identities,
roles and session key material, all methods passing verification of those are
also resistant to KCI attacks.
% KARL: I added a caveat in the first sentence.
% In a sense we have modelled some of it as explicit as it gets: entity
% authentication for instance: both parties agree on their own and on their
% peer's identity. We do show that (implicitly for all methods, and explicitly
% for some).

% \knote{Even if this is true for the formalization above, it may not hold once
%     encoded into Tamarin. Cas mentioned at IndoCrypt that there are KCI attacks
%     that Tamarin cannot discover (need reference for that).
% }
% \vnote{This is fair. I think if we move this to discussion (or even if we don't) we should make a note that we have not formalized these in Tamarin and are merely following evident inferences.}

If a party $A$ can be coerced into believing it completed a run with $B$, but
where the session key material is actually shared with $C$ instead, the 
protocol is vulnerable to an \emph{Unknown Key-Share (UKS)}
attack~\cite{DBLP:conf/ima/Blake-WilsonJM97}.
%
For the same reason as for KCI, a protocol passing verification of our agreement
properties is also resistant to UKS attacks.
%

From the injective agreement properties it follows that each party is assured
the identity of its peer upon completion.
%
Therefore, the agreement properties also capture \emph{entity authentication}.
%
%We say that \mEdhoc{} in method $m$ satisfies \emph{explicit authentication} for
%the initiator $I$ with a responder $R$, if injective agreement holds for $I$
%with $R$ on the session key $sk$, when running method $m$.
%%
%The corresponding definition for the responder is analogous.
%%
%If both parties obtain explicit authentication we refer to it as mutual explicit
%authentication (or simply explicit authentication).
%%
%A party $A$ is guaranteed explicit authentication when both parties have
%agreement on fresh session key material, each others identities and roles
%(and other parameters), when $A$ completes the protocol run.
%%
%Because they have (cryptographically justified) agreement on each others
%identities, it follows that explicit authentication implies entity
%authentication.
%%
%As we discuss later, it turned out that explicit authentication does not hold for all
%\mEdhoc{} methods, in which cases we prove \emph{implicit authentication}.

%Computational models often rely on implicit session key authentication
%(see, for example, the definition of SK-security in the Canetti-Krawczyk
%model~\cite{DBLP:conf/crypto/CanettiK02}).
%%
%Although symbolic models predominantly rely on correspondence properties
%in the style of Lowe~\cite{DBLP:conf/csfw/Lowe97a}, there are examples where
%implicit session key authentication has been used.
%%
%For example, Schmidt~et~al.~\cite{DBLP:conf/csfw/SchmidtMCB12} use a
%symbolized version of an extended Canetti-Krawzcyk model.

%%
%We say that a protocol satisfies \emph{implicit authentication} if the
%initiator and responder agree on the session key only after both parties
%successfully completes the protocol.
%%
%That is, authentication is implicit, as the
%initiator receives no confirmation that the responder has computed the same session key.
%%
%More precisely, we adapt the definition of~\cite{DBLP:journals/iacr/GuilhemFW19}
%to the symbolic model, and we prove that if an initiator $I$ and a responder $R$
%complete the protocol deriving the same session key, then $I$ believes they are
%talking to $R$ and vice versa.
%%

%\mEdhoc{} claims to support mutual authentication with consistency, aliveness
%and peer awareness (see Section~\ref{sec:claimedProperties}).
%%
%These claims appear to have been imported from \mSigma{}~\cite{sigma}, using the
%inheritance argument, because all three appear there as well.
%%
%The definition of consistency~\cite{sigma} is very close in intent to the
%definition we use for implicit authentication, except that it is posed in a
%computational setting.
%%
%We therefore consider them equal in terms of intent for the purpose of this
%paper.
%%
%Aliveness is, according to \mSigma{}~\cite{sigma}, guaranteed to a party $A$ if
%after running the protocol with $B$, $A$ knows that $B$ was alive during
%the execution, e.g., by verifying that $B$ signed a challenge.
%%
%For \mEdhoc{} this means that if we can prove mutual injective agreement, i.e.,
%explicit authentication, on \mGxy{}, then mutual aliveness follows.
%%
%Peer awareness is informally defined in~\cite{sigma} as the guarantee
%that if $A$ completes a protocol run with $B$, then not only does $A$
%know that $B$ is alive, but also that $B$ has initiated a corresponding session
%with $A$.
%\\

%\runhead{Perfect Forward Secrecy (PFS)}
%%\subsubsection{Perfect Forward Secrecy} 
%Perfect forward
%secrecy holds if, for any run in which the initiator and the responder
%agree on a session key $sk$ and any of their long-term keys are revealed after
%the run is complete, the attacker does still not learn $sk$.

 
%\subsubsection{Key-Compromise Impersonation} (KCI) This property takes the perspective of one
%of the endpoints of the protocol, say Alice running a session with Bob. A
%protocol is secure under KCI if Alice can still establish a secure session with
%Bob, even though Alice's keys are compromised at any time, and Bob's key
%material is not leaked until the end of the session.
%
% 
%\subsubsection{Post-Compromise Security} (PCS) A protocol that has
%\emph{post-compromise security} (following definitions in~\cite{cohn2016post})
%is capable of establishing a secure session even after one of the parties has
%been compromised. Cohn-Cordon et al.~\cite{cohn2016post} presents two notions of
%PCS, namely weak and strong PCS: here we focus on the latter.
%%
%A protocol guarantees \emph{weak PCS} if secrecy of any session key $sk$ holds
%between the initiator and the responder, even if the run of the protocol that
%established $sk$ happens after a \emph{limited compromise}, where the key
%material is not leaked, but the attacker is capable of impersonating both
%parties (i.e. has the ability to perform all cryptographic operations using the
%initiator's and responder's long term keys, but has not access to the long term
%keys).

 
%With the first rule we allow the protocol to decrypt the message \mT{m} if the encryption has matching key \mT{k}, authenticated data \mT{ad}, and uses the same algorithm \mT{al}.
%
%The second rule allows the attacker to decrypt the message \mT{m} with the key
%\mT{k} and without the authenticated data \mT{ad}, and hence skip the check.

%The built-in theories for XOR and Diffie-Hellman are a fair bit more complex
%than authenticated encryption, hence we refer to the original
%papers~\cite{DBLP:conf/csfw/DreierHRS18,DBLP:conf/csfw/SchmidtMCB12}
%for a full reference.
%

 
%\subsubsection{Syntactic Sugar} In the following presentation we use some syntactic
%sugar, namely
%the use of let bindings (\mT{let ... in}), which are series of
%definitions of patterns which are substituted in the rest of the rule. %Another
%prominent feature is the use of tuples (\mT{<t1, ..., tn>}) which are a
%built-in concept in \mTamarin.

%---------------------------------------------------------------------------
\subsection{\mTamarin{}}
\label{sec:tamarin}
% \knote{FOR PFS: It turns out that Tamarin's logic does not allow formulas of the form:
%     $(A \lor B) \rightarrow C$, so I cannot make the Tamarin model match the
%     formula above. The Tamarin model still uses one event type to represent that
%     either $I$ or $R$ completed.  One way of solving this is to say here in the
%     paper that we define a new event type $\mathbf{Comp} =
%     \mIComplete \lor \mRComplete$. That is exactly what is going on anyway.
% }
We chose \mTamarin{} to model and verify \mEdhoc{} in the symbolic model.
%
\mTamarin{} is an interactive verification tool in which models are specified
as multi-set rewrite rules that define a transition relation.
%
The elements of the multi-sets are called facts and represent the global system
state.
%
Rules are equipped with event annotations called actions. Sequences of actions make up the execution trace.
%
These actions allow the user to check logic formulas on models.
%
%The logic used is a fragment of temporal first order logic.
%

Multi-set rewrite rules with actions are written $ l \ifarrow[e] r $,
where $l$ and $r$ are multi-sets of facts, and $e$ is a multi-set of actions.
%
% Facts and actions are $n$-ary predicates over a term algebra, which defines a
% set of function symbols $\mathcal F$, variables $\mathcal V$ and names
% $\mathcal N$.
Facts and actions are $n$-ary predicates over a term algebra, which defines a
set of function symbols, variables and names.
%
\mTamarin{} checks equality of these terms under an equational theory $E$.
%
For example, one can write $ dec(enc(x,y),y) =_E x $
to denote that symmetric decryption reverses the encryption operation under
this theory.
%
The equational theory $E$ is fixed per model, and hence we omit the subscript.
%
\mTamarin{} supports let-bindings and tuples as syntactic sugar to simplify
model definitions.
%
It also provides built-in rules for Dolev-Yao adversaries and for
managing their knowledge.
%

We implement events using actions, and parameters associated with events using
terms of the algebra.
%

To simplify the exposition, we do not describe the model in full detail, but
it can be found in~\cite{edhocTamarinRepo}.\\
%

%\runhead{Semantics and Built-ins}
%%\subsubsection{Semantics and Built-ins} \phantom{} 
%\mTamarin{} states
%$S$, $S'$ are multisets of facts, and a semantic transition of the form $S \semarrow[E] S'$
%occurs if there is a rule $l \ifarrow[e] r$ and a substitution $\sigma$ such
%that $S \supseteq \sigma(l)$ and $S' = S \setminus \sigma(l) \uplus \sigma(r)$
%and $E = \sigma(e)$.
%
%There are a few more details, such as persistent facts which are denoted by a $!$
%and are never removed from the state.
%%
%The sorts fresh (denoted by $\sim$) and public (denoted by $\$$) denote fresh
%constants and public values known to the attacker respectively, and are both
%sub-sorts of a base sort.
%%
%Finally, \mTamarin{} has some built-in predicates ($\mIn,
%\mOut$ to represent input and output of messages with the attacker,
%and
%$\mFr$ to denote a fresh constant created in the current rule, among
%others), rules and equations that represent the attacker's knowledge
%and standard equational theories in the symbolic model,
%which we present later.

%\anote{This can go, I make a shorter note later:\\
%{Notational conventions} In the remainder of this section we present
%\mTamarin{} code as it appears in the models that we verify, in the style of
%literate programming.  Whenever possible we match the style of the protocol
%diagrams in Section~\ref{sec:edhoc} and the naming convention of the \mEdhoc{}
%\mSpec~\cite{selander-lake-edhoc-01}, so that each element of the model is
%traceable to the standard.  There are a few exceptions to this, most notably
%some variable names that we introduce for the sake of the \mTamarin{} model and are
%not present in the original \mSpec{}, which will appear in \mT{camelCase}, and
%the syntax for Diffie-Hellman exponentiation which is specific to \mTamarin{}.
%We also use \mT{xx} to name the ephemeral key for the initiator (resp. \mT{yy}
%for the responder) as to avoid confusion with \mTamarin's builtin variable
%names \mT{x} and \mT{y}.}

\runhead{Protocol Rules and Equations}
%\subsubsection{Protocol Rules and Equations}
\mTamarin{} allows users to define new function symbols and equational theories.
These user-defined objects are then translated by \mTamarin{} into rewrite
rules, which are added to the set of considered rules during verification.
For example, in our model we have a symbol to denote authenticated encryption,
for which \mTamarin{} produces the following rule:
%
\begin{lstlisting}
[!KU(k), !KU(m), !KU(ad), !KU(ai)] --> [!KU(aeadEncrypt(k, m, ad, ai))]
\end{lstlisting}
%
to denote that if the adversary knows a key \mT{k}, a message \mT{m}, the
authenticated data \mT{ad}, and an algorithm \mT{al}, then they can construct
the encryption, and thus get to know the message
\lstinline{aeadEncrypt(k, m, ad, al)}.



%-------------------------------------------------------------------------- sub
\subsection{Modeling \mEdhoc{} in \mTamarin}
\label{sec:modeling}
%In this section we detail the modeling choices that we have made for this formal
%verification effort.
%
We model all five different methods of the \mEdhoc{}
\mSpec~\cite{our-analysis-selander-lake-edhoc-01}, namely \mPskPsk,
\mSigSig, \mSigStat, \mStatSig{} and \mStatStat.
%
The methods share much common structure, and we derive
their \mTamarin{}-models from a single specification written with the aid of the
M4 macro language.
%
%Whenever possible we adhere with the variable names present in the \mSpec{} and
%in Section~\ref{sec:edhoc}.
%%
%There are a few exceptions: we use \mT{camelCase} for names introduced in the
%modeling, and we use \mT{xx} and \mT{yy} for the ephemeral keys, to avoid name clashes.
%
To keep the presentation brief, we only present the \mStatSig{} metohod, as it
shows asymmetric authentication methods simultaneously.
%
%More details on the \mPskPsk{} method and 
The full \mTamarin{} code for all models can be found at~\cite{edhocTamarinRepo}.
\\
%\anote{Fix: give a dropbox link instead of the repo}
%

%\subsubsection{Primitive Operations}
\runhead{Primitive Operations}
Our model uses the built-in theories of exclusive-or and DH operations, as
in~\cite{DBLP:conf/csfw/DreierHRS18,DBLP:conf/csfw/SchmidtMCB12}.
%
Hashing is modeled via the built-in hashing function symbol augmented
with a public constant as additional input, to model different
hash functions.
%
%Due to not modeling error messages, the selected hash function is fixed.
%
The HKDF interface is represented by \mT{expa} for the
expansion operation and \mT{extr} for the extraction operation.
%
Signatures use \mTamarin's built-in theory for \mT{sign} and \mT{verify}
operations.
%
%The XOR theory introduces the symbol \mT{XOR}, plus the necessary equational theory including associativity, commutativity, and inverse.
%
%The Diffie-Hellman theory introduces exponentiation \mT{g^y} and product of exponents \mT{x * y} as built-in symbols in the language, plus the necessary equational theory of associativity, commutativity, distributivity of exponentiation with product, and inverse.
%
For \mAead{} operations on key \mT{k}, message \mbox{\mT{m}}, additional data \mT{ad}
and algorithm identifier \mT{ai}, we use \lstinline{aeadEncrypt(k, m, ad, ai)}
for encryption.
%
Decryption with verification of the integrity is defined via the equation
\lstinline{aeadDecrypt(k, aeadEncrypt(k, m, ad, ai), ad, ai) = m}.
%
To enable the adversary to decrypt without needing to verify the integrity, we
also add the equation \lstinline{decrypt(k, aeadEncrypt(k, m, ad, ai), ai) = m}.
%
The latter equation is not used by honest parties.\\
%

%\subsubsection{Protocol Environment and Adversary Model}
\runhead{Protocol Environment and Adversary Model}
We model binding between a party's identity and a public key using the
following rules.
%
\begin{lstlisting}
rule registerLTK_SIG:
    [Fr(~ltk)] --[UniqLTK($A, ~ltk)]->
        [!LTK_SIG($A, ~ltk), !PK_SIG($A, pk(~ltk)), Out(<$A, pk(~ltk)>)]
rule registerLTK_STAT:
    [Fr(~ltk)] --[UniqLTK($A, ~ltk)]->
        [!LTK_STAT($A, ~ltk), !PK_STAT($A, 'g'^~ltk), Out(<$A, 'g'^~ltk>)]
\end{lstlisting}
%
The rules \mT{registerLTK_SIG} and \mT{registerLTK_STAT} register a public key
for the \mSig{}- and \mStat{}-based methods respectively.
%
%A similar rule \mT{registerLTK_PSK} registers pre-shared symmetric keys for
%pairs of agents.
%%
The fact \mT{Fr(~ltk)} creates a fresh term \mbox{\mT{ltk},} representing a long-term
secret key, which is not known to the adversary.
%
The fact \mbox{\mT{Out(<$A, pk(~ltk)>)}} sends the identity of the party
owning the long-term key and the corresponding public key to the adversary.
%
The event \mT{UniqLTK} together with a corresponding restriction models the fact
that the each party is associated with exactly one long-term key.
%
Consequently, an adversary cannot register additional long-term keys for an
identity.
%
% or pair of agents, as enforced by the following restriction:
% \begin{lstlisting}
% restriction uniqLTKs:
%     "All id k1 k2 #i #j. (UniqLTK(id, k1)@i & UniqLTK(id, k2)@j) ==> k1 = k2"
% \end{lstlisting}
In line with the \mEdhoc{} \mSpec{}, this models an external mechanism
ensuring that long term keys are bound to correct identities, e.g., as
managed by a certificate authority.
%

We rely on \mTamarin's{} built-in message deduction rules for a Dolev-Yao adversary.
%
To model an adversary compromising long-term keys, i.e., events of type
\mRevLTK{}, and revealing ephemeral DH half-keys, i.e., events of type
\mRevEph{}, we use standard reveal rules.
%
% \begin{lstlisting}
% rule revealLTK_SIG:
%     [!LTK_SIG($A, ~ltk)] --[LTKRev($A)]-> [Out(~ltk)]
% rule revealLTK_STAT:
%     [!LTK_STAT($A, ~ltk)] --[LTKRev($A)]-> [Out(~ltk)]
% rule revealEphKey:
%     [CompletedRevEph(~tid, u, v, sk, ~eph)] --[EphKeyRev(sk)]-> [Out(~eph)]
% \end{lstlisting}
%rule forge_SIG: [!LTK_SIG($A, ~ltk), In(xx)] --[TEE($A)]-> [Out(sign(xx, ~ltk))]
%rule exp_STAT: [!LTK_STAT($A, ~ltk), In('g'^x)] --[TEE($A)]-> [Out(('g'^x)^~ltk)]
%
The timing of these reveals as modelled by these events is important. The long-term keys are revealed on registration, before protocol execution.
%
The ephemeral DH half-key of a party is revealed when the party completes, i.e.,
at events of type \mIComplete{} and \mRComplete.~\footnote{A stronger, and perhaps more realistic model, would reveal ephemeral keys also when they are
created at the start of the protocol run, but we failed to get \mTamarin{} to
terminate on that.}
%
% \knote{Most state-of-the-art Tamarin models of real-world protocols
%         (involving DH) do not
%         reveal the actual session key used for the secure channel, e.g., Cremers
%         et~al.~\cite{DBLP:conf/ccs/CremersFKN20}, Girol
%         et~al.~\cite{DBLP:conf/uss/GirolHSJCB20}. Similarly to us, they instead
%         only leak the session state (only ephemeral keys in out case).  We had
%         serious problems getting Tamarin to terminate if leaking actual session
%         key. Is this a common problem in Tamarin?  "All" computational models
%         have session key reveal queries. CK and eCK have session state (eph-key)
%         reveal queries as well.
%     }
%These rules allow to check Perfect Forward Secrecy, Key Compromise Impersonation
%and (weak) Post Compromise Security as defined in Section~\ref{sec:desired-properties},
%by giving the attacker the ability to access to long term and session keys, or
%to the cryptographic interface, at the appropriate time.

%\subsubsection{Modeling Choices}
\runhead{Protocol Roles}
We model each method of the protocol with four rules: \mT{I1}, \mT{R2}, \mT{I3}
and \mT{R4} (with the method suffixed to the rule name).
%
Each of these represent one step of the protocol as run by the initiator $I$
and the responder $R$.
%
The rules correspond to the event types \mIStart, \mRStart, \mIComplete,  and
\mRComplete, respectively.%, and they can be traced back to the diagrams of
%Figure~\ref{fig:edhocsigstat} and the more general
%Figure~\ref{fig:edhocFramework}.
%
Facts prefixed with \mT{StI} carry state information between \mT{I1} and \mT{I3}.
%
A term unique to the current thread, \mT{tid}, links two rules to a given state fact.
%
Similarly, facts prefixed with \mT{StR} carry state information between the
responder role's rules.
%
Lines 27--28 in the \mT{R2_STAT_SIG} rule shown below illustrate one such use of state
facts.
%

%Our model differs slightly from the \mSpec{}.
%%
%In particular, for convenience we divide the \mMethod{} element into two
%elements, representing the method for the initiator and the responder without
%reducing the attacker potential.
%
We do not model the error message that $R$ can send in response to message
\mMsgone, and hence our model does not
capture the possibility for $R$ to reject $I$'s offer.
%

%We model the XOR encryption of \mT{CIPHERTEXT_2} with the key \mT{K_2e} as to
%allow recovering of part of the key for known plaintext.
%
%Hence \mT{CIPHERTEXT_2} is not a direct XOR ``encryption'' in the model, but
%rather a tuple where each field is XORed with a half-key expansion (\mT{K_2e_1}
%and \mT{K_2e_2}).
We model the XOR encryption of \mT{CIPHERTEXT_2} with the key \mT{K_2e} using
\mTamarin{}'s built in theory for XOR, and allow each term of the encrypted
element to be attacked individually.
%
That is, we first expand \mT{K_2e} to as many key-stream terms as there are
terms in the plaintext tuple using the \mHkdfExpand{} function in a counter-mode
of operation as defined by \mEdhoc{}.
%
We then XOR each term in the plaintext with its own key-stream term.
%
This models the \mSpec{} closer than if we would have XORed \mT{K_2e}, as a
single term, onto the plaintext tuple.
%
The XOR encryption can be seen in on line 18-21 in the listing of
\mT{R2_STAT_SIG} below.
%
In the latest version of the \mEdhoc{}
specification~\cite{latest-ietf-lake-edhoc-03}, the XOR-based encryption
has been changed into using the encryption algorithm from the AEAD algorithm
from \mSuites{}.
%
\begin{lstlisting}
rule R2_STAT_SIG:
  let
     agreed = <CS0, CI, ~CR> // Agreed data (and exp_sk)
     gx = 'g'^xx
     data_2 = <'g'^~yy, CI, ~CR>
     m1 = <'STAT', 'SIG', CS0, CI, gx>
     TH_2 = h(<$H0, m1, data_2>)
     prk_2e = extr('e', gx^~yy)
     prk_3e2m = prk_2e
     K_2m = expa(<$cAEAD0, TH_2, 'K_2m'>, prk_3e2m)
     protected2 = $V // ID_CRED_V
     CRED_V = pkV
     extAad2 = <TH_2, CRED_V>
     assocData2 = <protected2, extAad2>
     MAC_2 = aead('e', K_2m, assocData2, $cAEAD0)
     authV = sign(<assocData2, MAC_2>, ~ltk)
     plainText2 = <$V, authV>
     K_2e = expa(<$cAEAD0, TH_2, 'K_2e'>, prk_2e)
     K_2e_1 = expa(<$cAEAD0, TH_2, 'K_2e', '1'>, prk_2e)
     K_2e_2 = expa(<$cAEAD0, TH_2, 'K_2e', '2'>, prk_2e)
     CIPHERTEXT_2 = <$V XOR K_2e_1, authV XOR K_2e_2>
     m2 = <data_2, CIPHERTEXT_2>
     exp_sk = <gx^~yy>
  in
     [!LTK_SIG($V, ~ltk), !PK_SIG($V, pkV), In(m1), Fr(~CR), Fr(~yy), Fr(~tid)]
     --[ExpRunningR(~tid, $V, exp_sk, agreed), R2(~tid, $V, m1, m2)]->
     [StR2_STAT_SIG($V, ~ltk, ~yy, prk_3e2m, TH_2, CIPHERTEXT_2,
                    gx^~yy, ~tid, m1, m2, agreed),
      Out(m2)]
\end{lstlisting}
%

To implement events and
to bind them to parameters, we use several \mTamarin{} actions.
%
For example, the action \mT{ExpRunningR(~tid, \$V, exp_sk, agreed)} in line 26
above implements binding an event of type \mRStart{} to the parameters and session key
material.% for the purpose of injective agreement.
%

As explained in Section~\ref{sec:agreedParams}, it is not possible to show
injective agreement on session key material when it includes \mGiy.
%
Therefore, we use different actions to emit events that include \mGiy{} in the
session key material and events that do not.
%
Session key material which includes (resp. does not include) \mGiy{} is referred to as \mT{imp_sk} (resp. \mT{exp_{sk}}) in the
\mTamarin{} model. %and the material that does not is referred to as \mT{exp_sk}.
%
In the case of the \mSigSig, \mSigStat{}, and \mPskPsk{} methods, \mT{imp_sk}
is the same as \mT{exp_sk}, because there is no \mGiy.%{} to take into account.
%

%We model the session key material differently for implicit authentication and
%explicit authentication.
%%
%Specifically, when the initiator uses the \mStat{} authentication method,
%\mT{impSk} includes the semi-static key \mGiy{}, whereas \mT{expSk} does not
%include it.
%%
%The reason for this is that, when sending the second message \mMsgtwo{}, the
%responder does not yet know the identity of the initiator and hence cannot
%indicate knowledge of \mGiy{} to the initiator.
%%
%Because we want to verify strong properties such as explicit authentication
%when possible, we collect those items in the key material referred to as
%\mT{exp_k}.
%%
%When not possible we prove weaker properties for \mT{impSk}, which excludes
%key material which it is even theoretically impossible to get explicit
%authentication on.
%%

%What happens after a run of \mEdhoc{} completes is beyond the scope of our study, hence
%we have left this part out of the modeling and focus on the key material
%that forms the basis for the \mOscore{} security context.
%%

%-------------------------------------------------------------------------- sub
\subsection{Property Formalization}
\label{sec:propertyFormalization}
The properties we defined above translate directly into \mTamarin's logic,
with the mapping of events to the actions emitted from the model.
%
As an example, we show the lemma for verifying the property \mPredPfs.
%
\begin{lstlisting}
lemma secrecyPFS:
   all-traces
   "All u v sk #t3 #t2. (K(sk)@t3 & CompletedRun(u, v, sk)@t2) ==>
       ( (Ex #t1. LTKRev(u)@t1 & #t1 < #t2)
          | (Ex #t1. LTKRev(v)@t1 & #t1 < #t2)
          | (Ex #t1. EphKeyRev(sk)@t1))"
\end{lstlisting}
%
As mentioned earlier, the action \mT{CompletedRun(u, v, sk)} in line 3 is
emitted by both the rules \mT{I3} and \mT{R4}, and corresponds
to the disjunction of events $\mIComplete^{t_2} \lor \mRComplete^{t_2}$ in the
definition of \mPredPfs{} in Section~\ref{sec:secrecy}.
%
Similarly, \mT{EphKeyRev(sk)} in line 6 represents the ephemeral
DH half-key is revealed for either $I$ or $R$, or both.
%

%In this section we present our formalization of the security properties. %into\\
%%\mTamarin{} lemmas.
%%
%We refer to Section~\ref{sec:desired-properties} for a full explanation of the
%properties.
%\\

%\runhead{Explicit Authentication}
%We model explicit authentication between the initiator and the
%responder in the form of mutual injective agreement on the session key material,
%and on the roles and identities of the two parties.
%%
%We split the property into two lemmas, one for authenticating the responder to the
%initiator, and one for the other direction.
%%
%For the first case, we use the action facts \mT{ExpCommitI} and
%\mT{ExpRunningR}, and show that there is injective agreement
%between the two action facts on the parameters identities \mT{U} and \mT{V},
%their respective roles,  and the session key material \mT{expSk}.
%%
%The key material differs between \mEdhoc{} methods.
%%

%Additionally, we require that injective agreement must hold only when
%no long-term key material for the two parties has been revealed before
%the initiator completes the protocol run.
%%
%This is covered by the main disjunction in lines 10-12 on the right of
%the implication.
%%
%That is, either we have injective agreement or one of
%the three \mT{LtkRev} action facts must have been generated.
%%

%lemma authInjAgreeGuaranteeForI:
%    all-traces
%    "All tidI u v expSk #i.
%         (ExpCommitI(tidI, u, v, expSk)@i
%	     & (All #j m1. I1(tidI, u, v, m1) @ j ==> (All #k. TEE(u)@k ==> k < j) & (All #k. TEE(v)@k ==> k < j))
%         & (All tidR #j m1 m2. R2(tidR, v, m1, m2) @ j ==> (All #k. TEE(u)@k ==> k < j) & (All #k. TEE(v)@k ==> k < j)))
%          ==>
%         ( ( (Ex tidR #j. ExpRunningR(tidR, v, expSk)@j & #j < #i)
%           & not(Ex tidI2 u2 v2 #i2. ExpCommitI(tidI2, u2, v2, expSk)@i2 & not(#i = #i2) ) )
%         | (Ex #j. LTKRev(v)@j & #j < #i) )"

% Code from July 22 commit
%\begin{lstlisting}
%lemma authInjAgreeGuaranteeForI:
%     all-traces
%     "All tidI u v expSk #i.
%          ExpCommitI(tidI, u, v, expSk)@i ==>
%          ( ( (Ex tidR #j. ExpRunningR(tidR, v, expSk)@j & #j < #i)
%            & not( Ex tidI2 u2 v2 #i2. ExpCommitI(tidI2, u2, v2, expSk)@i2
%                 & not(#i = #i2)
%                 )
%            )
%          | (Ex #j. LTKRev(<u, v>)@j & #j < #i)
%          | (Ex #j. LTKRev(u)@j & #j < #i)
%          | (Ex #j. LTKRev(v)@j & #j < #i)
%          )
%     "
%\end{lstlisting}

%Note that this property \emph{does not hold} when the initiator is
%running the \mStat{} method, because the key material would then cover \mGiy{},
%and as discussed above that is not possible.
%%
%For that case we need to prove implicit authentication, as detailed in
%the next section.

%Similarly to the previous lemma, we require that injective agreement also holds
%in the reverse direction:
%%
%%\begin{lstlisting}
%%lemma authInjAgreeGuaranteeForR:
%%    all-traces
%%    "All tidR u v sk #i.
%%         (CommitR(tidR, u, v, sk)@i
%%	     & (All tidI #j m1. I1(tidI, u, v, m1) @ j ==> (All #k. TEE(u)@k ==> k < j) & (All #k. TEE(v)@k ==> k < j))
%%         & (All #j m1 m2. R2(tidR, v, m1, m2) @ j ==> (All #k. TEE(u)@k ==> k < j) & (All #k. TEE(v)@k ==> k < j)) )
%%         ==>
%%         ( ( (Ex tidI #j. ExpRunningI(tidI, u, v, sk)@j & #j < #i)
%%           & not(Ex tidR2 u2 v2 #i2. ExpCommitR(tidR2, u2, v2, sk)@i2 & not(#i = #i2)) )
%%         | (Ex #j. LTKRev(u)@j & #j < #i) )"
%%\end{lstlisting}

%% From commit on June 22
%\begin{lstlisting}
%lemma authInjAgreeGuaranteeForR:
%    all-traces
%    "All tidR u v sk #i.
%         CommitR(tidR, u, v, sk)@i ==>
%         ( ( (Ex tidI #j. RunningI(tidI, u, v, sk)@j & #j < #i)
%           & not( Ex tidR2 u2 v2 #i2. CommitR(tidR2, u2, v2, sk)@i2
%                & not(#i = #i2)
%                )
%           )
%         | (Ex #j. LTKRev(<u, v>)@j & #j < #i)
%         | (Ex #j. LTKRev(u)@j & #j < #i)
%         | (Ex #j. LTKRev(v)@j & #j < #i)
%         )
%    "
%\end{lstlisting}
%%
%The initiator is the first to complete the protocol run and confirms that
%it knows both parties identities, their roles (based on message types) and the
%session key in the third message.
%%
%In particular, the initiator knows the responder's identity, so even if the
%responder uses the \mStat{} authentication, the knowledge asymmetry that caused
%problems for injective agreement on \mGiy{} does not occur for \mGrx{}.
%%
%Therefore, the responder gets injective agreement guarantees, and hence explicit
%authentication guarantees, for the entire session key material.
%%
%Consequently, we do not need to differentiate the explicit session key from the
%implicit session key in this case and can ignore the \mT{Exp} prefix for the
%running and commit action facts (\mT{RunningI} and \mT{CommitR} respectively).
%%
%\\

%%\subsubsection{Implicit Authentication}
%\runhead{Implicit Authentication}
%The following lemma proves implicit authentication:
%% \begin{lstlisting}
%% lemma authGIYImplicitAuthGuaranteeForI:
%%     all-traces
%%     "All tidI u v impSk #i.
%%          CommitI(tidI, u, v, impSk)@i ==>
%%          ( ( (All tidR u2 v2 #j. CommitR(tidR, u2, v2, impSk)@j ==>
%%                 (u = u2  &  v = v2)
%%              )
%%            &
%%              (not Ex #k. K(impSk)@k)
%%            &
%%              (not( Ex tidR u v #j tidR2 u2 v2 #j2.
%%                       ( CommitR(tidR,  u,  v,  impSk)@j
%%                       & CommitR(tidR2, u2, v2, impSk)@j2
%%                       & not(#j = #j2)
%%                       )
%%                  )
%%              )
%%            )
%%          | (Ex #k. LTKRev(u)@k) | (Ex #k. TEE(u)@k)
%%          | (Ex #k. LTKRev(v)@k) | (Ex #k. TEE(v)@k)
%%          )
%%          "
%% \end{lstlisting}

%\begin{lstlisting}
%lemma authGIYImplicitAuthGuaranteeForI:
%    all-traces
%    "All tidI u v impSk #i.
%         CommitI(tidI, u, v, impSk)@i ==>
%         ( ( (All tidR u2 v2 #j. CommitR(tidR, u2, v2, impSk)@j ==>
%                (u = u2  &  v = v2)
%             )
%           &
%             (not Ex #k. K(impSk)@k)
%           &
%             (not( Ex tidR u v #j tidR2 u2 v2 #j2.
%                      ( CommitR(tidR,  u,  v,  impSk)@j
%                      & CommitR(tidR2, u2, v2, impSk)@j2
%                      & not(#j = #j2)
%                      )
%                 )
%             )
%           )
%         | (Ex #k. LTKRev(u)@k)
%         | (Ex #k. LTKRev(v)@k)
%         | (Ex #k. LTKRev(<u, v>)@k)
%         )
%    "
%\end{lstlisting}
%As opposed to lemma \mT{authInjAgreeGuaranteeForI}, here we prove that the two
%parties implicitly authenticate on the key \mT{impSk}.
%%
%In this lemma we show that if any two parties (\mT{u} and \mT{v2} here) 
%have completed a run of the protocol, and \mT{u} believes she is talking to
% \mT{v} and \mT{v2} believes he is talking to \mT{u2}, then their beliefs
%match, i.e., \mT{u} = \mT{u2} and \mT{v} = \mT{v2}).
%%
%Furthermore there is an injective correspondence
%between the \mT{CommitI} and \mT{CommitR} events, and the attacker does not
%learn the session key material.
%%
%\\

%%\subsubsection{Secrecy, Forward Secrecy and Session Key Independence}
%\runhead{Secrecy, Forward Secrecy and Session Key Independence}
%Finally, we prove secrecy of session key material, perfect forward secrecy
%(PFS) and session key material independence.
%%
%All these properties are verified by proving the same lemma for each method,
%as secrecy is a strictly weaker property than PFS (and hence follows
%directly), and session key independence can be proven along PFS.
%%
%This is done by allowing the attacker to reveal long-term keys after either
%the initiator or the responder have completed the protocol, and by
%allowing the attacker to reveal session key material.
%%
%Despite these additional attacker capabilities, it still holds that the session
%key material is secret for all the other runs of the protocol.
%%
%The lemma is as follows:
%\begin{lstlisting}
%lemma secrecyPFSGIYSessionKey:
%	all-traces
%    "(All tid u v sk #i #j. (K(sk)@i & CommitI(tid, u, v, sk)@j) ==>
%            ((Ex #l. LTKRev(u)@l & #l < #j) | (Ex #l. LTKRev(v)@l & #l < #j) | (Ex #l. SKRev(sk)@l))
%         )
%         &
%         (All tid u v sk #i #j. (K(sk)@i & CommitR(tid, u, v, sk)@j) ==>
%            ((Ex #l. LTKRev(u)@l & #l < #j) | (Ex #l. LTKRev(v)@l & #l < #j) | (Ex #l. SKRev(sk)@l))
%     )
%    "
%\end{lstlisting}
%% We present the lemma for the \mSigStat{} method:
%% \begin{lstlisting}
%%   lemma secrecyPFSGIYSessionKey:
%%         all-traces
%%         "(All tid u v sk #i #j. (K(sk)@i & CommitI(tid, u, v, sk)@j) ==>
%%             ((Ex #l. LTKRev(u)@l & #l < #j) | (Ex #l. LTKRev(v)@l & #l < #j) | (Ex #l. SKRev(sk)@l) | (Ex w #l. TEE(w)@l))
%%          )
%%          &
%%          (All tid u v sk #i #j. (K(sk)@i & CommitR(tid, u, v, sk)@j) ==>
%%             ((Ex #l. LTKRev(u)@l & #l < #j) | (Ex #l. LTKRev(v)@l & #l < #j) | (Ex #l. SKRev(sk)@l) | (Ex w #l. TEE(w)@l))
%%             )"
%% \end{lstlisting}

%%% Local Variables:
%%% mode: latex
%%% TeX-master: "main"
%%% End:
