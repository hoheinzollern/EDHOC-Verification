% !TEX TS-program = pdflatexmk

%\documentclass[runningheads, envcountsame, hidelinks, a4paper, draft, x11names]{llncs}
\documentclass[runningheads, envcountsame, hidelinks, a4paper, x11names]{llncs}
\usepackage{amsmath,amssymb,amsfonts}
\usepackage[scaled=0.8]{helvet}    % Less huge \textsf{functionName}
%\usepackage{enumitem}       % compacts lists and stuff
%\usepackage[subtle]{savetrees}
\usepackage[misc,geometry]{ifsym} % for letter symbol
\usepackage{soul}           % \hl for highlighting text; \st for strike-through
\usepackage{graphicx}
\usepackage[dvipsnames]{xcolor}
%\usepackage{wrapfig}
\usepackage{tikz}
\usetikzlibrary{trees,snakes,arrows}
\usetikzlibrary{shapes,chains}
\usetikzlibrary{positioning}
\usepackage{hyperref}
\usepackage[nospace]{cite}

\usepackage{ifthen}

% -- Fonts and styles for names
\newcommand{\mFunStyle}[1]{\textsf{#1}}
\newcommand{\mConstStyle}[1]{\textsf{#1}}
\newcommand{\mVarStyle}[1]{\mathit{#1}}
\newcommand{\mFactStyle}[1]{\textsf{#1}}
\newcommand{\mMethodStyle}[1]{\mConstStyle{#1}}
\newcommand{\mProtocolStyle}[1]{\text{#1}}

% -- Macros for consistent wording
\newcommand{\mSpec}{specification}  % The EDHOC spec document we are analyzing

% -- Domain specific macros
\newcommand{\mArxiv}{\texttt{arXiv}}
\newcommand{\mTamarin}{\mProtocolStyle{Tamarin}}
\newcommand{\mProverif}{\mProtocolStyle{ProVerif}}
\newcommand{\mEdhoc}{\mProtocolStyle{EDHOC}}
\newcommand{\mOscore}{\mProtocolStyle{OSCORE}}
\newcommand{\mSigma}{\mProtocolStyle{SIGMA}}
\newcommand{\mSigmaI}{\mProtocolStyle{SIGMA\nobreakdash-I}}
\newcommand{\mCbor}{\mProtocolStyle{CBOR}}
\newcommand{\mCose}{\mProtocolStyle{COSE}}
\newcommand{\mCoseEncrypt}{\mProtocolStyle{COSE\_Encrypt0}}
\newcommand{\mCoseSign}{\mProtocolStyle{Cose\_Sign1}}
\newcommand{\mHkdf}{\mProtocolStyle{HKDF}}
\newcommand{\mHkdfExtract}{\mProtocolStyle{HKDF\nobreakdash-extract}}
\newcommand{\mHkdfExpand}{\mProtocolStyle{HKDF\nobreakdash-expand}}
\newcommand{\mHmac}{\mProtocolStyle{HMAC}}
\newcommand{\mAead}{\mProtocolStyle{AEAD}}
\newcommand{\mAeadDecrypt}{\mProtocolStyle{AEAD\nobreakdash-decrypt}}
\newcommand{\mDecrypt}{\mProtocolStyle{decrypt}}
\newcommand{\mOptls}{\mProtocolStyle{OPTLS}}
\newcommand{\mNoise}{\mProtocolStyle{Noise}}
\newcommand{\mTls}{\mProtocolStyle{TLS}}
\newcommand{\mDandTls}{\mProtocolStyle{(D)TLS}}
\newcommand{\mCtls}{\mProtocolStyle{cTLS}}
\newcommand{\mCoap}{\mProtocolStyle{CoAP}}

\newcommand{\mStat}{\mMethodStyle{STAT}}
\newcommand{\mSig}{\mMethodStyle{SIG}}
\newcommand{\mPsk}{\mMethodStyle{PSK}}
\newcommand{\mStatStat}{\mMethodStyle{STAT-STAT}}
\newcommand{\mStatSig}{\mMethodStyle{STAT-SIG}}
\newcommand{\mSigStat}{\mMethodStyle{SIG-STAT}}
\newcommand{\mSigSig}{\mMethodStyle{SIG-SIG}}
\newcommand{\mPskPsk}{\mMethodStyle{PSK-PSK}}

\newcommand{\mSid}{\mConstStyle{sid}}    % session id = (u, v, s-key)

\newcommand{\mXor}{\mConstStyle{XOR}}
\newcommand{\mSuites}{\mConstStyle{Suites\_I}}
\newcommand{\mMethod}{\mConstStyle{Method}}
\newcommand{\mCi}{\mConstStyle{C\_I}}
\newcommand{\mCr}{\mConstStyle{C\_R}}
\newcommand{\mGi}{\mConstStyle{G\_I}}
\newcommand{\mGr}{\mConstStyle{G\_R}}
\newcommand{\mGiy}{\mConstStyle{G\_IY}}
\newcommand{\mGrx}{\mConstStyle{G\_RX}}
\newcommand{\mGx}{\mConstStyle{G\_X}}
\newcommand{\mGy}{\mConstStyle{G\_Y}}
\newcommand{\mGxy}{\mConstStyle{G\_XY}}
\newcommand{\mIDPsk}{\mConstStyle{ID\_PSK}}
\newcommand{\mTH}{\mConstStyle{TH}}
\newcommand{\mTHtwo}{\mConstStyle{TH\_2}}
\newcommand{\mKtwoe}{\mConstStyle{K\_2e}}
\newcommand{\mKtwom}{\mConstStyle{K\_2m}}
\newcommand{\mKtwoae}{\mConstStyle{K\_2ae}}
\newcommand{\mSign}{\mConstStyle{sign}}

\newcommand{\mKthreeae}{\mConstStyle{K\_3ae}}
\newcommand{\mKthreem}{\mConstStyle{K\_3m}}

\newcommand{\mTHthree}{\mConstStyle{TH\_3}}
\newcommand{\mhplain}{\mConstStyle{h''}}
\newcommand{\mCredi}{\mConstStyle{CRED\_I}}
\newcommand{\mCredr}{\mConstStyle{CRED\_R}}
\newcommand{\mHash}{\mConstStyle{H}}

\newcommand{\mTHfour}{\mConstStyle{TH\_4}}
\newcommand{\mAuthi}{\mConstStyle{Auth\_I}}
\newcommand{\mAuthr}{\mConstStyle{Auth\_R}}

\newcommand{\mMactwo}{\mConstStyle{MAC\_2}}
\newcommand{\mMacthree}{\mConstStyle{MAC\_3}}

\newcommand{\mSigtwo}{\mConstStyle{Sig\_2}}
\newcommand{\mSigthree}{\mConstStyle{Sig\_3}}

\newcommand{\mMsgone}{\mConstStyle{m1}}
\newcommand{\mMsgtwo}{\mConstStyle{m2}}
\newcommand{\mMsgthree}{\mConstStyle{m3}}

\newcommand{\mCipher}{\mConstStyle{cipher\_2}}

\newcommand{\mAD}{\mConstStyle{AD}}
\newcommand{\mADone}{\mConstStyle{AD\_1}}
\newcommand{\mADtwo}{\mConstStyle{AD\_2}}
\newcommand{\mADthree}{\mConstStyle{AD\_3}}

\newcommand{\mPRK}{\mConstStyle{PRK}}
\newcommand{\mPRKtwo}{\mConstStyle{PRK\_2e}}
\newcommand{\mPRKthree}{\mConstStyle{PRK\_3e2m}}
\newcommand{\mPRKfour}{\mConstStyle{PRK\_4x3m}}

\newcommand{\mIdcredi}{\mConstStyle{ID\_CRED\_I}}
\newcommand{\mIdcredr}{\mConstStyle{ID\_CRED\_R}}
\newcommand{\mLtki}{\mConstStyle{ltk\_I}}
\newcommand{\mLtkr}{\mConstStyle{ltk\_R}}
\newcommand{\mLtk}{\mConstStyle{ltk}}

% TIKZ messages and actions
\newcommand{\msg}[4]{\draw[->,thick] ([yshift=-#1]#2.south) coordinate (l1)--(l1-|#3) node[midway, above]{#4}}
\newcommand{\action}[3]{\node[draw,thick,fill=white,align=center,below={#1} of {#2}]{#3}}

% Tamarin symbols
\newcommand{\ifarrow}[1]{\ensuremath{\mathit{\,-\hspace{-2.4pt}[{#1}]\hspace{-4.95pt}\rightarrow\,}}}
\newcommand{\semarrow}[1]{\ensuremath{\mathit{\,=\hspace{-4.9pt}[{#1}]\hspace{-5pt}\Rightarrow\,}}}
\newcommand{\mIn}{\mathsf{In}}
\newcommand{\mOut}{\mathsf{Out}}
\newcommand{\mFr}{\mathsf{Fr}}
\newcommand{\mKD}{\mathsf{KD}}
\newcommand{\mKU}{\mathsf{KU}}
\newcommand{\mT}[1]{\lstinline[basicstyle=\sffamily\normalsize]{#1}} % Tamarin code inline
\usepackage[final]{listings}

\lstdefinestyle{mystyle}{
    backgroundcolor=,   
    commentstyle=\color{Green},
    identifierstyle=\color{black},
    keywordstyle=\color{ForestGreen},
    numberstyle=\color{Gray},
    stringstyle=,
    basicstyle=\sffamily\small,
    keywords={let,in,rule,restriction,axiom,lemma,all-traces,exists-trace,Ex,All,Fr,In,Out},
    breakatwhitespace=false,
    breaklines=true,
    captionpos=b,
    keepspaces=false,
    numbers=left,
    numbersep=5pt,
    showspaces=false,
    showstringspaces=false,
    showtabs=false,
    tabsize=2,
    morecomment=[l]{//},
    literate=
    {=}{$=$}{1}
    {[}{$[$}{1}%
    {]}{$]$}{1}%
    {<}{$\langle$}{1}%
    {>}{$\rangle$}{1}%
    {(}{$($}{1}%
    {)}{$)$}{1}%
    {&}{$\wedge$}{1}
    {|}{$\vee$}{1}
    {$}{\$}{1}
    {<\ \#}{$<$\ \ \#}{3}% trick here: we distinguish angle brackets from temporal comparisons because of the # for the variable on the rhs
    {--[}{$\mbox{-\hspace{-4.0pt}[}$}{3}%
    {]->}{$\mbox{]\hspace{-4.2pt}\rightarrow}$}{3}%
    {-->}{$\rightarrow$}{1}
    {==>}{$\Rightarrow$}{1}
    {XOR}{$\oplus$}{1}%
    {aeadEncrypt}{\mAead{}}{6}%
    {decrypt}{\mDecrypt}{6}
    {aeadDecrypt}{\mAeadDecrypt}{11}
    {hkdfExtract}{\mHkdfExtract}{14}
    {hkdfExpand}{\mHkdfExpand}{14}
    {All\ }{$\forall$\ }{3}
    {Ex\ }{$\exists$\ }{3}
    {h(}{H$($}{2}
    {~}{{\url{~}}}{1}
    }
\lstset{style=mystyle}


\begin{document}
\title{Formal Analysis of EDHOC Key Establishment for Constrained IoT Devices}
%
%\author{Karl Norrman\inst{1,2}\textsuperscript{(\Letter)}\orcidID{0000-0003-0164-1478}
%\and
%Vaishnavi Sundararajan\inst{3}
%\and
%Alessandro Bruni\inst{4}
%}
\author{}
%
% Suggestion for shortened title in running heads:
%    Formal Analysis of EDHOC Key Establishment
%
\authorrunning{}
%\authorrunning{K. Norrman et al.}
% First names are abbreviated in the running head.
% If there are more than two authors, 'et al.' is used.
%
%\institute{
%    KTH Royal Institute of Technology, Stockholm, Sweden \and
%    Ericsson Research, Security, Stockholm, Sweden,\\
%    \email{karl.norrman@ericsson.com} \and
%    Ericsson Research, Int. Auton. Systems, Bangalore, India,\\
%    \email{vaishnavi.sundararajan@ericsson.com} \and
%    IT University of Copenhagen, Copenhagen, Denmark,\\
%    \email{brun@itu.dk}
%}
\institute{}
%
\maketitle
%

\begin{abstract}
%\hl{250 words}
The IETF is standardizing an authenticated key establishment (AKE) protocol
named \mEdhoc{} for constrained IoT devices~\cite{selander-lake-edhoc-01}.
%
In contrast to more powerful devices like web cameras and cars, which receive a
lot of media attention, such devices operate under severe energy consumption
and message size restrictions.
%
\mEdhoc{} was first formally analyzed in 2018 by
Bruni~et~al.~\cite{DBLP:conf/secsr/BruniJPS18}.
%
Since then, the protocol has been significantly extended and now has three new
key establishment methods.
%
In this paper, we formally analyze all methods of \mEdhoc{} in a symbolic
Dolev-Yao model, using the \mTamarin{} verification tool.
%
We show that the different methods provide sensible, but also rather
heterogeneous security properties, and discuss various consequences of this.
%
Our work has also led to improvements in the design and the specification of
\mEdhoc.
%
\end{abstract}
%

% KARL: appears there are no standard keywords like in ACM and previous years
% SSR papers use very custom key words
\keywords{Formal verification \and Security \and IETF standardization \and
          Key establishment protocols \and Tamarin}

%-------------------------------------------------------------------------- sec
%\vspace{-2.5em}
\section{Introduction}
\label{sec:introduction}

%-------------------------------------------------------------------------- sub
\subsection{Background and motivation}
\label{sec:motivation}
IoT security threats involving computationally strong devices such as cars
and web-cameras receive the most attention from media and academia.
%
Securing communication between such devices can readily be done using existing
protocols like \mDandTls.
%
Constrained devices and networks, on the other hand, which operate under severe
bandwidth and energy consumption restrictions, have received much less
attention.
%
These devices may be simple sensors which only relay environment
measurements to a server every hour, but need to function autonomously without
maintenance for long periods of time.
%
To keep energy consumption down, highly specialized radio links with small
and heterogeneous frame sizes are sometimes used.
%
The IETF defined the Constrained Application Protocol (\mCoap{}) protocol for
data transmission in such situations~\cite{rfc7252}.
%
\mCoap{} does not include security protection on its own, and in some cases,
\mDandTls{} messages are too large to fit into the radio frames, or is
unsuitable to handle proxy operations with change of underlying transport
protocol.
%
These are some of the reasons the IETF standardized the Object Security for
Constrained RESTful Environments
(\mOscore{}) protocol to secure communications between constrained
devices~\cite{rfc8613}.
%

The \mOscore{} protocol requires a pre-established security context.
%
For a couple of years, requirements and mechanisms for a key
exchange protocol, named \mEdhoc~\cite{selander-lake-edhoc-01}, for
establishing \mOscore{} security contexts have been discussed in the IETF.
%
Naturally, \mEdhoc{} must work under the same constrained requirements as
\mOscore{} itself.
%
While not all use cases for \mEdhoc{} are firmly set, the overall goal is to
establish an \mOscore{} security context, under message size limitations.
%
The work is now being done in the IETF Lightweight Authenticated Key Exchange
(LAKE) working group.
%

%-------------------------------------------------------------------------- sub
\subsection{Related Work}
\label{sec:relatedWork}
%
The first incarnation of \mEdhoc{} appeared in March 2016.
%
It contained two different key establishment methods, one based on a
pre-shared Diffie-Hellman (DH) cryptographic core and a second based on a
variant of challenge-response signatures, in the style of the \mNoise{}
framework~\cite{perrin2016noise}.
%
By a cryptographic core, or simply core, we mean an academic protocol, without
encodings or application specific details required by an industrial protocol.
%
By a key establishment method we mean a core with such details added.
%

The core based on challenge-response signatures was replaced by \mSigma{}
in May 2018.
%
Now, variants using challenge-response signatures have been added again by
integrating the cryptographic core of \mOptls{}.
%
Additionally, mixed variants where one party uses a challenge-response
signature and the other a regular signature have also been added.
%
Consequently, there are now five methods in total, and formal analysis is
required to verify security guarantees for \mEdhoc.
%
This is especially important, since the \mSpec{} itself lacks a description
of the intended security model and overall security goals.
%

Three cores have been analyzed in the computational model (\mSigma{}~\cite{DBLP:conf/crypto/CanettiK02},
\mOptls{}~\cite{DBLP:conf/eurosp/KrawczykW16}, and
\mNoise{}~\cite{DBLP:conf/eurosp/KobeissiNB19}).
%
However, there are significant departures from these cores in \mEdhoc, and
these proofs do not carry over automatically
(See Section~\ref{sec:relationsToOtherProtocols}).
%
The work closest to ours is Bruni~et~al.~\cite{DBLP:conf/secsr/BruniJPS18},
formally analyzing the May 2018 version of \mEdhoc{} (with two methods) in
\mProverif~\cite{DBLP:conf/csfw/Blanchet01}.
%

%\knote{should be moved to section 3}
%
%Computational models often rely on implicit session key authentication
%(see, for example, the definition of SK-security in the Canetti-Krawczyk
%model~\cite{DBLP:conf/crypto/CanettiK02}).
%%
%Although symbolic models predominantly rely on correspondence properties
%in the style of Lowe~\cite{DBLP:conf/csfw/Lowe97a}, there are examples where
%implicit session key authentication has been used.
%%
%For example, Schmidt~et~al.~\cite{DBLP:conf/csfw/SchmidtMCB12} use a
%symbolized version of an extended Canetti-Krawzcyk model.
%%

%-------------------------------------------------------------------------- sub
\subsection{Contributions}
\label{sec:contributions}
We formally analyze the \mEdhoc{} protocol in the \mTamarin{}
tool~\cite{DBLP:conf/cav/MeierSCB13}.
%
This work is presented in Section~\ref{sec:formalization}.
%
The formal models and proven properties are based on the version as on
2020-03-01~\cite{selander-lake-edhoc-01}.
%
We give an explicit security model for the protocol and verify essential
properties, such as session key and entity authentication and perfect forward
secrecy, for all five methods.
%
We also discuss the relationship between proven properties and the claimed
security properties, and the lack of a precise security model for \mEdhoc{}.
%
In fact, the standard has already been improved based on our analysis,
which we discuss in Section~\ref{sec:discussion}.
%

%-------------------------------------------------------------------------- sec
\section{The \mEdhoc{} Protocol}
\label{sec:edhoc}
% !TEX root =  main.tex

%Description of \m\mEdhoc and main changes from last verified version

\vnote{I find the macros for the protocol, method, and tool names distracting while reading through (the font changes too much, too often). The macro for EDHOC actually inserts a line break if you start a sentence with it, because of the \texttt{hbox} (I'm not sure why the hbox exists for what is a macro to be inserted in running text). I do need to start sentence with EDHOC many times though, so this is irritating. It is also the case that ProVerif needs to be written like so, and not capitalized entirely, so at least one of them is plain wrong! For now I've left in the macros, but I'd prefer that we fixed them to be regular capitalized text or, in the case of ProVerif, in camelcase as they should be.}

\subsection{Overview}
Constrained IoT systems often deal with a lot of valuable personal and business information that ought to be kept secure. Such systems need to be assured of end-to-end protection with source authentication and perfect forward secrecy. It is often desirable to protect such devices at the application layer -- for example, in cases where transport layer security is not sufficient [\mcneed], or where multiple underlying protocols need to be accounted for. One method for providing application layer security is provided by CBOR Object Signing and Encryption (COSE) [RFC8152: \mcfix].  

In order to derive shared key material with which to proceed, communicating parties can run an Elliptic Curve Diffie-Hellman key exchange protocol with ephemeral keys. Ephemeral Diffie-Hellman Over COSE (\mEdhoc) is a lightweight key exchange protocol for such situations, and is expected to provide perfect forward secrecy and identity protection. \mEdhoc supports authentication using pre-shared keys (PSK), raw public keys (RPK), and public key certificates. After successful completion of the \mEdhoc protocol, application keys and other application specific data can be derived using the \mEdhoc-Exporter interface. 

A main use case for \mEdhoc is to establish a security context for Object Security for Constrained RESTful Environments (\mOscore) [RFC8613: \mcfix]. \mOscore is a protocol which uses COSE for application-layer protection on top of the transport-layer Constrained Application Protocol (CoAP). \mEdhoc uses COSE for cryptography, CBOR for encoding, and CoAP for transport. By reusing existing libraries, the additional code footprint can be kept very low.

\mEdhoc is designed to work in highly constrained scenarios. This makes it especially suitable for network technologies which have low throughput, low power consumption, and small frame sizes. Examples include Cellular IoT, 6TiSCH, and LoRaWAN [\mcneed].

\subsection{Background, comparison with~\cite{DBLP:conf/secsr/BruniJPS18}}
The first version of \mEdhoc was proposed in March 2016 to a working group investigating lightweight authenticated key exchange protocols [\mcneed]. There has been a focus on formally verifying that the protocol satisfies the properties expected of it right from the beginning. 

The 2018 work~\cite{DBLP:conf/secsr/BruniJPS18} by Bruni et al performed a formal verification of version 08 [\url{https://tools.ietf.org/html/draft-selander-ace-cose-ecdhe-08} \mcfix] of \mEdhoc. The protocol and properties are modelled and verified in the \mProverif tool. This version of the protocol belongs to the \mSigmaI family of protocols, and has two modes -- one with asymmetric keys, and one with pre-shared symmetric keys (PSK). Bruni et al showed that this version satisfies the requisite properties of identity protection, (perfect forward) secrecy of data, and strong authentication, upon completion of the protocol.

\mEdhoc has undergone a lot of change since version 08, as will be described in the following sections, and the formal verification of the current version, therefore, is a worthwhile exercise.

%\subsection{Overarching goal of security context establishment}
\subsection{Methods and features of \textsc{EDHOC}}
\mEdhoc can established Diffie-Hellman key exchange in one of three different ways -- using digital signatures, static Diffie-Hellman keys, or pre-shared symmetric keys. We describe each of these methods in detail below. The communicating parties must agree on the method as part of the first message.

\subsubsection{\mSigSig method}
The \mSigma (SIGn-and-MAc) family of protocols [\mcneed] has many variants. The \mSigSig method of \mEdhoc is built on \mSigmaI, a variant of the \mSigma protocol which provides identity protection for the initiator, and  implements the \mSigmaI variant as Mac-then-Sign. The current \mSigSig method corresponds (with a few minor changes) to the asymmetric key mode of \mEdhoc v08. 

The communicating parties exchange ephemeral public keys, compute the shared secret, and derive symmetric application keys from this secret.

\subsubsection{\mPskPsk method}
In this method, the initiator and responder are assumed to have a pre-shared key which is secret to them, and can be retrieved by the responder using a public part of the first message (\mIDPSK). This method corresponds to the symmetric key method of \mEdhoc v08. 

In the first message, the initiator sends a message consisting of the method name, the initiator's ephemeral key (\mGx), their connection identifier (\mCi), the \mIDPSK identifier, and (optional) plaintext (\mADone). 

The second message, sent by the responder, is composed of \mCi, the responder's ephemeral key \mGy, their connection identifier \mCr, and an AEAD encryption [\mcneed] of the transaction hash of the first message (\mTHtwo) along with an optional plaintext \mADtwo. The key used for this (\mKtwo) is derived using the EDHOC key derivation function with \mTHtwo and the pseudorandom string \mPRKtwo as input, while the associated data for the AEAD encryption is constructed by concatenating a constant string, the hash function, and \mTHtwo. 

The initiator, upon receipt of this message, sends back \mCr, followed by an AEAD encryption of the transaction hash of the second message (\mTHthree) along with a plaintext \mADthree. As earlier, the key used is derived by supplying \mTHthree and \mPRKthree as input to the KDF, and the  associated data obtained by concatenating the constant string, hash function, and \mTHthree. An abstract description is shown in Figure~\ref{fig:edhocpsk}.

\vnote{Figure numbers don't square up, check class file guidelines!}

\begin{figure}\label{fig:edhocpsk}
\centering
\includegraphics[scale=0.3]{Images/psk.png}
\caption{The PSK-PSK method of EDHOC}
\end{figure}

\subsubsection{\mStat-based methods}
\mEdhoc allows for three \mStat-based methods -- two where only one participant has a static Diffie-Hellman key (while the other uses signatures), and one where both do. This set of methods is not covered in v08 of \mEdhoc, which only has a single \mSigma asymmetric key method (corresponding to the \mSigSig method shown above). This allows one party to use a \mSigma style of authentication, while the other can use something along the lines of \mOptls.

\vnote{Need to talk about links to OPTLS and NOISE}



\subsection{Expected security properties}


%Discussion of KDF -- page 13 of EDHOC

%certain security applications may be integrated into EDHOC by transporting auxiliary data together with the messages. One example is the transport of third-party authorization information protected outside of EDHOC [I-D.selander-ace-ake-authz]. Another example is the embedding of a certificate enrolment request or a newly issued certificate. EDHOC allows opaque auxiliary data (AD) to be sent in the EDHOC messages. Unprotected Auxiliary Data (AD_1, AD_2) may be sent in message_1 and message_2, respectively. Protected Auxiliary Data (AD_3) may be sent in message_3. Since data carried in AD1 and AD2 may not be protected, and the content of AD3 is available to both the Initiator and the Responder, special considerations need to be made such that the availability of the data a) does not violate security and privacy requirements of the service which uses this data, and b) does not violate the security properties of EDHOC. 



%-------------------------------------------------------------------------- sec
\section{Formalization and Results}
\label{sec:formalization}
% !TEX root =  main.tex
 
Next we describe our approach towards formalizing the \mEdhoc{} protocol and list the properties we verify. %We used the
%symbolic (Dolev-Yao) model for verification, with \mTamarin{} for tool support.
%
%The next three subsections describe our threat model, briefly present the \mTamarin{} tool, and our modeling choices.
%
%Finally, we present the properties that we proved in this effort.
\subsection{Threat Model}\label{sec:threat-model}
We verify \mEdhoc{} in the symbolic Dolev-Yao model: as customary in this style of
modeling, we assume all cryptographic primitives to be ``perfect''. Encrypted messages can only be decrypted with the key, and no hash collisions exist. The attacker controls the
%, and hence
%only allow the attacker to encrypt and decrypt messages when they know the key,
%and exclude hash collisions, for example; the attacker is in control of the
 communication channel, and can interact with unbounded sessions of the protocol,
dropping, injecting and modifying messages at their liking.

One important point of the modeling is that we allow the attacker to impersonate
dishonest and/or compromised endpoints, by revealing their long-term and session
key material at any given point.
%
%Conversely, 
We say that a party is honest if they never reveal their
long-term key or session key material.

An important point is to define what the key material is.
    \mEdhoc{} does not result in an explicit session key, but a cryptographic
    state from which keys for \mOscore{} can be derived using \mHkdf.
    As will be seen later, depending on how the key material is defined, the
    different methods will have different authentication properties.
    %In particular, all methods except those where the initiator uses the
    %\mStat{} method provide a stronger form of authentication (injective
    %agreement) for the initiator.

\subsection{\mTamarin{}}
\label{sec:tamarin}
 
We chose \mTamarin{} to model and verify \mEdhoc{} in the symbolic model.
%
\mTamarin{} is an interactive verification tool based on multi-set rewriting rules
with event annotations, which allow the user to check Linear Temporal Logic
(LTL) formulas on these models.
%
Multi-set rewrite rules with events look like $ l \ifarrow[e] r $,
where $l$ and $r$ are multi-sets of facts, and $e$ is a multi-set of events.
Facts are $n$-ary predicates over a term algebra, which defines a set of function
symbols $\mathcal F$, variables $\mathcal V$ and names $\mathcal N$. \mTamarin{}
checks equality of these terms under an equational theory $E$. For example,
one can write $ dec(enc(x,y),y) =_E x $
to denote that symmetric decryption reverses the encryption operation under this theory.
The equational theory $E$ is fixed per model, and hence we omit the subscript.

In the presentation of the model we use some syntactic
sugar, namely
the use of let bindings (\mT{let ... in}). This is a series of
definitions of patterns which are substituted in the rest of the rule. \\

%\runhead{Semantics and Built-ins}
%%\subsubsection{Semantics and Built-ins} \phantom{} 
%\mTamarin{} states
%$S$, $S'$ are multisets of facts, and a semantic transition of the form $S \semarrow[E] S'$
%occurs if there is a rule $l \ifarrow[e] r$ and a substitution $\sigma$ such
%that $S \supseteq \sigma(l)$ and $S' = S \setminus \sigma(l) \uplus \sigma(r)$
%and $E = \sigma(e)$.
%
%There are a few more details, such as persistent facts which are denoted by a $!$
%and are never removed from the state.
%%
%The sorts fresh (denoted by $\sim$) and public (denoted by $\$$) denote fresh
%constants and public values known to the attacker respectively, and are both
%sub-sorts of a base sort.
%%
%Finally, \mTamarin{} has some built-in predicates ($\mIn,
%\mOut$ to represent input and output of messages with the attacker,
%and
%$\mFr$ to denote a fresh constant created in the current rule, among
%others), rules and equations that represent the attacker's knowledge
%and standard equational theories in the symbolic model,
%which we present later.

%\anote{This can go, I make a shorter note later:\\
%{Notational conventions} In the remainder of this section we present
%\mTamarin{} code as it appears in the models that we verify, in the style of
%literate programming.  Whenever possible we match the style of the protocol
%diagrams in Section~\ref{sec:edhoc} and the naming convention of the \mEdhoc{}
%\mSpec~\cite{selander-lake-edhoc-01}, so that each element of the model is
%traceable to the standard.  There are a few exceptions to this, most notably
%some variable names that we introduce for the sake of the \mTamarin{} model and are
%not present in the original \mSpec{}, which will appear in \mT{camelCase}, and
%the syntax for Diffie-Hellman exponentiation which is specific to \mTamarin{}.
%We also use \mT{xx} to name the ephemeral key for the initiator (resp. \mT{yy}
%for the responder) as to avoid confusion with \mTamarin's builtin variable
%names \mT{x} and \mT{y}.}

\runhead{Protocol rules and equations} 
%\subsubsection{Protocol Rules and Equations}
\mTamarin{} allows users to define new function symbols and equational theories.
These user defined objects are then translated by \mTamarin{} into rewrite
rules, which are added to the set of considered rules during verification.
For example, in our model we have a symbol to denote authenticated encryption, for which \mTamarin{} produces the following rule:
%
\begin{lstlisting}
[!KU(k), !KU(m), !KU(ad), !KU(al)] --> [!KU(aeadEncrypt(k, m, ad, al))]
\end{lstlisting}
%
to denote that if the attacker knows a key \mT{k}, a message \mT{m}, the
authenticated data \mT{ad}, and an algorithm \mT{al}, then they can construct
the encryption using these parameters, and thus get to know the message
\lstinline{aeadEncrypt(k, m, ad, al)}.

In our model we make use of
the built-in theories of exclusive-or and DH operations, as in~\cite{DBLP:conf/csfw/DreierHRS18,DBLP:conf/csfw/SchmidtMCB12}.
%
%The XOR theory introduces the symbol \mT{XOR}, plus the necessary equational theory including associativity, commutativity, and inverse.
%
%The Diffie-Hellman theory introduces exponentiation \mT{g^y} and product of exponents \mT{x * y} as built-in symbols in the language, plus the necessary equational theory of associativity, commutativity, distributivity of exponentiation with product, and inverse.
For \mAead{} operations, we add the following equations:
\begin{lstlisting}
aeadDecrypt(k, aeadEncrypt(k, m, ad, al), ad, al) = m
decrypt(k, aeadEncrypt(k, m, ad, al), al) = m
\end{lstlisting}
Both the above equations govern decryption. The first rule checks to see if the authenticated data is as expected, while the second rule skips this check.


 
\subsection{Verified Properties}
\label{sec:desired-properties}

 
%\subsubsection{Secrecy}
\runhead{Secrecy}
We say that \mEdhoc{} satisfies secrecy of the established session key $sk$
between two honest parties $I$ and $R$ if, for any run of the protocol between $I$ and
$R$, the attacker does not get to know $sk$.\\
%
%The attacker may passively observe---and actively interfere with---the
%communication, and run any number of sessions with $A$ and $B$, in either role,
%concurrently or otherwise.

 
%\subsubsection{Authentication}
\runhead{Authentication}
\label{sec:authenticationDef}
Following~\cite{DBLP:conf/csfw/Lowe97a}, we say that a protocol guarantees to an
initiator $I$ injective agreement with a responder $R$ if, whenever $I$ believes
they have completed a run of the protocol with $R$, then $R$ has previously been
running the protocol with $I$ in these particular roles, and each such run of
$I$ corresponds to a unique run of $R$.

%To define \mEdhoc{}'s authentication properties we make use of Lowe's definition
%of \emph{injective agreement}~\cite{DBLP:conf/csfw/Lowe97a}:
%\knote{Maybe we don't need to quote the definition and could paraphrase it in a
%shorter form to save space.}
%\begin{quote}
%  ``We say that a protocol guarantees to an initiator $A$ [injective] agreement
%  with a responder $B$ on a set of data items $ds$ if, whenever $A$ (acting as
%  initiator) completes a run of the protocol, apparently with responder $B$,
%  then $B$ has previously been running the protocol, apparently with $A$, and
%  $B$ was acting as responder in his run, and the two agents agreed on the data
%  values corresponding to all the variables in $ds$, and each such run of $A$
%  corresponds to a unique run of $B$.''
%\end{quote}
%
We say that \mEdhoc{} in method $m$ satisfies \emph{explicit authentication} for
the initiator $I$ with a responder $R$, if injective agreement holds for $I$
with $R$ on the session key $sk$, when running method $m$.
%
The corresponding definition for the responder is analogous.
%
If both parties obtain explicit authentication we refer to it as mutual explicit
authentication (or simply explicit authentication).

A party $A$ is guaranteed explicit authentication when both parties agree
on the session key (and other parameters), when $A$ completes the protocol
run.
%
As we discuss later, it turned out that explicit authentication does not hold for all
\mEdhoc{} methods, in which cases we prove \emph{implicit authentication}.

Computational models often rely on implicit session key authentication
(see, for example, the definition of SK-security in the Canetti-Krawczyk
model~\cite{DBLP:conf/crypto/CanettiK02}).
%
Although symbolic models predominantly rely on correspondence properties
in the style of Lowe~\cite{DBLP:conf/csfw/Lowe97a}, there are examples where
implicit session key authentication has been used.
%
For example, Schmidt~et~al.~\cite{DBLP:conf/csfw/SchmidtMCB12} use a
symbolized version of an extended Canetti-Krawzcyk model.

%
We say that a protocol satisfies \emph{implicit authentication} if the
initiator and responder agree on the session key only after a successful
execution of the protocol.
%
That is, authentication is implicit, as the
initiator receives no confirmation that the responder has computed the same session key.
%
More precisely, we adapt the definition of~\cite{DBLP:journals/iacr/GuilhemFW19}
to the symbolic model, and we prove that if an initiator $I$ and a responder $R$
complete the protocol deriving the same session key, then $I$ believes they are
talking to $R$ and vice versa.\\

\runhead{Session independence} 
%\subsubsection{Session Independence}
Session independence holds if knowing one session key does
not give the attacker any information about other sessions.  To model session
key independence of \mEdhoc, we allow leakage of session keys, and 
check security only of those sessions for which the keys have not been
directly revealed to the attacker.\\

\runhead{Perfect forward secrecy (PFS)}
%\subsubsection{Perfect Forward Secrecy} 
Perfect forward
secrecy holds if, for any run in which the initiator and the responder
agree on a session key $sk$, the attacker does not learn $sk$, even when the
long-term keys are revealed after completing the session.

 
%\subsubsection{Key-Compromise Impersonation} (KCI) This property takes the perspective of one
%of the endpoints of the protocol, say Alice running a session with Bob. A
%protocol is secure under KCI if Alice can still establish a secure session with
%Bob, even though Alice's keys are compromised at any time, and Bob's key
%material is not leaked until the end of the session.
%
% 
%\subsubsection{Post-Compromise Security} (PCS) A protocol that has
%\emph{post-compromise security} (following definitions in~\cite{cohn2016post})
%is capable of establishing a secure session even after one of the parties has
%been compromised. Cohn-Cordon et al.~\cite{cohn2016post} presents two notions of
%PCS, namely weak and strong PCS: here we focus on the latter.
%%
%A protocol guarantees \emph{weak PCS} if secrecy of any session key $sk$ holds
%between the initiator and the responder, even if the run of the protocol that
%established $sk$ happens after a \emph{limited compromise}, where the key
%material is not leaked, but the attacker is capable of impersonating both
%parties (i.e. has the ability to perform all cryptographic operations using the
%initiator's and responder's long term keys, but has not access to the long term
%keys).

 
%With the first rule we allow the protocol to decrypt the message \mT{m} if the encryption has matching key \mT{k}, authenticated data \mT{ad}, and uses the same algorithm \mT{al}.
%
%The second rule allows the attacker to decrypt the message \mT{m} with the key
%\mT{k} and without the authenticated data \mT{ad}, and hence skip the check.

%The built-in theories for XOR and Diffie-Hellman are a fair bit more complex
%than authenticated encryption, hence we refer to the original
%papers~\cite{DBLP:conf/csfw/DreierHRS18,DBLP:conf/csfw/SchmidtMCB12}
%for a full reference.
%

 
%\subsubsection{Syntactic Sugar} In the following presentation we use some syntactic
%sugar, namely
%the use of let bindings (\mT{let ... in}), which are series of
%definitions of patterns which are substituted in the rest of the rule. %Another
%prominent feature is the use of tuples (\mT{<t1, ..., tn>}) which are a
%built-in concept in \mTamarin.

%-------------------------------------------------------------------------- sub
\subsection{Modeling \mEdhoc{}}
\label{sec:modeling} 
In this section we detail the modeling choices that we have made for this formal
verification effort.
%
We model the five different methods of \mEdhoc{} from a single specification
using the M4 macro language to derive all valid combinations: \mPskPsk,
\mSigSig, \mSigStat, \mStatSig{} and \mStatStat.
%
Whenever possible we adhere with the variable names present in the \mSpec{} and
in Section~\ref{sec:edhoc}.
%
There are a few exceptions: we use \mT{camelCase} for those names the modeling
introduces, and we use \mT{xx} and
\mT{yy} for the ephemeral keys, to avoid name clashes.
%
To keep the presentation brief, we only present the \mStatSig{} mode, as it
shows both asymmetric authentication methods at the same time.
%
More details on the \mPskPsk{} method and all \mTamarin{} models can be
found through this link~\cite{edhocTamarinRepo}.
\\
\anote{Fix: give a dropbox link instead of the repo}
%

%\subsubsection{General Setup}
\runhead{General Setup}
The following rules express the registering of the long term keys for the
\mSig{}- and \mStat{}-methods, respectively.
%
\begin{lstlisting}
rule registerLTK_SIG:
 [Fr(~ltk)] --[ UniqLTK($A, ~ltk) ]->
  [!LTK_SIG($A, ~ltk), !PK_SIG($A, pk(~ltk)), Out(<!<$A, pk(~ltk)>!>)]
rule registerLTK_STAT:
 [Fr(~ltk)] --[ UniqLTK($A, 'g'^~ltk) ]->
  [!LTK_STAT($A, ~ltk), !PK_STAT($A, 'g'^~ltk), Out(<!<$A, 'g'^~ltk>!>)]
\end{lstlisting}
%
The rules \mT{registerLTK_SIG} and \mT{registerLTK_STAT} register a public key
(for signing and static DH exchange, respectively) that are tied to the
identity of an agent \mbox{\mT{A}.}
%
A similar rule \mT{registerLTK_PSK} registers pre-shared symmetric keys for
pairs of agents.
%
The event \mT{UniqLTK} together with a corresponding restriction models that
the long-term key is unique for each agent.
% or pair of agents, as enforced by the following restriction:
% \begin{lstlisting}
% restriction uniqLTKs:
%     "All id k1 k2 #i #j. (UniqLTK(id, k1)@i & UniqLTK(id, k2)@j) ==> k1 = k2"
% \end{lstlisting}
This models that there is an external mechanism ensuring that the
long term keys are bound to the correct identity, e.g., a certificate authority.
%
It also models that the attacker cannot register new public keys for an
existing identity.
%

In our model we introduce rules to give the attacker access to
long-term keys and session keys.
%and the cryptographic interface of the device.
%
\begin{lstlisting}
rule revealLTK_SIG: [!LTK_SIG($A, ~ltk)] --[LTKRev($A)]-> [Out(~ltk)]
rule revealLTK_STAT: [!LTK_STAT($A, ~ltk)] --[LTKRev($A)]-> [Out(~ltk)]
rule revealSessionKeyI: [CommitI(tid, u, v, sk)] --[SKRev(sk)]-> [Out(sk)]
rule revealSessionKeyR: [CommitR(tid, u, v, sk)] --[SKRev(sk)]-> [Out(sk)]
\end{lstlisting}
%rule forge_SIG: [!LTK_SIG($A, ~ltk), In(xx)] --[TEE($A)]-> [Out(sign(xx, ~ltk))]
%rule exp_STAT: [!LTK_STAT($A, ~ltk), In('g'^x)] --[TEE($A)]-> [Out(('g'^x)^~ltk)]
%
These are used to model long-term key compromise and session key secrecy.
\\
%These rules allow to check Perfect Forward Secrecy, Key Compromise Impersonation
%and (weak) Post Compromise Security as defined in Section~\ref{sec:desired-properties},
%by giving the attacker the ability to access to long term and session keys, or
%to the cryptographic interface, at the appropriate time.

%\subsubsection{Modeling Choices}
\runhead{Modeling Choices}
We model each method of the protocol with four rules: \mT{I1}, \mT{R2}, \mT{I3}
and \mT{R4} (with the current method suffixed to the rule name).
%
Each of these represent one step of the protocol as run by the initiator \mT{I}
and the responder \mT{R}.
%
The rules can be traced back to the diagrams of
Figure~\ref{fig:edhocsigstat} and Figure~\ref{fig:edhocstatsig}.
%

Our model differs slightly from the \mSpec{}.
%
In particular, for convenience we divide the \mMethod{} element into two
elements, representing the method for the initiator and the responder; this
does not reduce the attacker potential.
%
To make the model manageable we omit the connection identifiers \mCi{} and
\mCr{}, and represent the selected ciphersuite by the public variable
\mT{\$cSUITES0}, known to the attacker.
%
We plan to introduce the connection identifiers in our ongoing verification
effort.
%
The way we model the selected cipher suite implies that our model does not
capture the possibility for the responder to reject the initiator's offer.
%

%We model the XOR encryption of \mT{CIPHERTEXT_2} with the key \mT{K_2e} as to
%allow recovering of part of the key for known plaintext.
%
%Hence \mT{CIPHERTEXT_2} is not a direct XOR ``encryption'' in the model, but
%rather a tuple where each field is XORed with a half-key expansion (\mT{K_2e_1}
%and \mT{K_2e_2}).
We model the XOR encryption of \mT{CIPHERTEXT_2} with the key \mT{K_2e} using
\mTamarin{}'s built in theory for XOR, and allow each term of the encrypted
element to be attacked individually.
%
That is, we first expand \mT{K_2e} to as many key-stream terms as there are
terms in the plaintext tuple using the \mHkdfExpand{} function in a counter-mode
of operation.
%
We then XOR each term in the plaintext with its own key-stream term.
%
This models the \mSpec{} closer than if we would have XORed \mT{K_2e}, as a
single term, onto the plaintext tuple.
%
The XOR encryption can be seen in on line 16-19 in the listing of
\mT{R2_STAT_SIG} below.
%
\begin{lstlisting}
rule R2_STAT_SIG:
    let
          data_2 = <'g'^~yy>
          m1 = <'STAT', 'SIG', $cSUITE0, gx>
          TH_2 = h(<$cHash0, m1, data_2>)
          prk_2e = hkdfExtract('emptyStr', gx^~yy)
          prk_3e2m = prk_2e
          K_2m = hkdfExpand(<$cAEAD0, TH_2, 'K_2m'>, prk_3e2m)
          protected2 = $V // ID_CRED_V
          CRED_V = pkV
          extAad2 = <TH_2, CRED_V>
          assocData2 = <protected2, extAad2>
          MAC_2 = aeadEncrypt('emptyStr', K_2m, assocData2, $cAEAD0)
          authV = sign(<assocData2, MAC_2>, ~ltk)
          plainText2 = <$V, authV>
          K_2e = hkdfExpand(<$cAEAD0, TH_2, 'K_2e'>, prk_2e)
          K_2e_1 = hkdfExpand(<$cAEAD0, TH_2, 'K_2e', '1'>, prk_2e)
          K_2e_2 = hkdfExpand(<$cAEAD0, TH_2, 'K_2e', '2'>, prk_2e)
          CIPHERTEXT_2 = <$V XOR K_2e_1, authV XOR K_2e_2>
          m2 = <data_2, CIPHERTEXT_2>
          exp_sk = <gx^~yy>
   in
          [ !LTK_SIG($V, ~ltk)
          , !PK_SIG($V, pkV)
          , In(m1)
          , Fr(~yy)
          , Fr(~tid)
          ]
          --[ ExpRunningR(~tid, $V, exp_sk)
           , R2(~tid, $V, m1, m2)
            ]->
          [ StR2_STAT_SIG($V, ~ltk, ~yy, prk_3e2m, TH_2,
                          CIPHERTEXT_2, gx^~yy, ~tid, m1, m2)
          , Out(m2)
          ]
\end{lstlisting}

We use conventional state facts to save the internal state of a party between
executions of their rules.
%
%For instance, the fact \mbox{\mT{StI1_PSK_PSK($U, ~ltk, $V, ~xx, m1, ~tid)}}
%stores the initiator's internal state after executing the first step of the
%\mPskPsk{} method.
For instance, the fact
\mT{StR2_STAT_SIG} on lines 32 and 33
in the listing above stores the responder's internal state after executing the
second step of the \mStatSig{} method.
%

We mark some steps of the protocol with \mT{Running} action facts, e.g., line 29
above.
%
These facts represent that the party consider itself running the
protocol using certain parameters.
%
We mark other steps with \mT{Commit} action facts.
%
These facts represent that the party consider itself having
completed the protocol using a certain set of parameters.
%
For example, the fact \mT{ExpRunningR(~tid, \$V, exp_sk)} (line 29 above)
represent that a party is running a session and that they believe
they are doing so in the repsonder role, that their own identity is \mT{V}
and that \mT{exp_sk} is the session key.
%
Other facts like \mT{ExpCommitI(~tid, \$U, \$V, exp_sk)}
%
and \mT{CommitI(~tid, \$U, \$V, imp_sk)} represent that a party has completed
a session in the role of initiator, their own identity being \mT{U}, their
peer's identity being \mT{V} and the session key being \mT{exp_sk} and
\mT{imp_sk} respectively.
%
We use these action facts to model explicit and implicit authentication, as will
follow below.
%
The difference between the two \mT{Commit} action fact types is the choice of
key material on which we verify authentication (\mT{exp_sk} vs \mT{imp_sk}).
%
In the case of the \mSigSig{}, \mSigStat{} and \mPskPsk{} methods, these keys
are the same, but there will be a crucial difference when the initiator runs
the \mStat{} method.
%

We model the session key material differently for Implicit authentication and
explicit authentication.
%
Specifically, when the initiator uses the \mStat{} authentication method,
\mT{imp_sk} includes the semi-static key \mGiy{}, whereas \mT{exp_sk} does not
include it.
%
The reason for this is that, when sending the second message \mMsgtwo{}, the
responder does not yet know the identity of the initiator and can hence not
indicate knowledge of \mGiy{} to the initiator.
%
Because we want to verify strong properties such as explicit authentication
when possible, we collect those items in the key material referred to as
\mT{exp_k}.
%
When not possible we prove weaker properties for \mT{imp_sk}, which excludes
key material which it is even theoretically impossible to get explicit
authentication on.
%

What happens after \mEdhoc{} completes is beyond the scope of our study, hence
we have left this part out of the modeling and only focus on the key material
that is the bases for the \mOscore{} security context.
%

%-------------------------------------------------------------------------- sub
\subsection{Property Formalization}
\label{sec:propertyFormalization}
In this section we present how we formalized the security properties into
\mTamarin{} lemmas.
%
We refer to Section~\ref{sec:desired-properties} for a full explanation of the
properties.
\\

\runhead{Explicit Authentication}
We model explicit authentication between the initiator and the
responder in the form of mutual injective agreement on the session key material,
the roles and identities of the two parties.
%
We split the property in two lemmas, one for authenticating the responder to the
initiator, and one for the other direction.
%
For the first case, we use the events \mT{ExpCommitI} and
\mT{ExpRunningR}, and show that there is injective agreement
between the two events on the parameters \mT{tidI},
\mT{v} and the session key material \mT{expSk}
%
The key material differ between \mEdhoc{} methods.
%

Additionally, we require that injective agreement must hold only when
no long term key material for the two parties has been revealed before
the end of the protocol.
%
This is achieved by the main disjunction in lines 5-10 on the right of
the implication, requiring to reveal the long term keys (i.e. one of
the three \mT{LtkRev} events must trigger) if the responder has
not been running a matching session with the initiator.

\knote{We can remove the line "all-traces" from all listings to save space.
It is Tamarin's default. One only have to add "exists-trace" explicitly.}

%lemma authInjAgreeGuaranteeForI:
%    all-traces
%    "All tidI u v expSk #i.
%         (ExpCommitI(tidI, u, v, expSk)@i
%	     & (All #j m1. I1(tidI, u, v, m1) @ j ==> (All #k. TEE(u)@k ==> k < j) & (All #k. TEE(v)@k ==> k < j))
%         & (All tidR #j m1 m2. R2(tidR, v, m1, m2) @ j ==> (All #k. TEE(u)@k ==> k < j) & (All #k. TEE(v)@k ==> k < j)))
%          ==>
%         ( ( (Ex tidR #j. ExpRunningR(tidR, v, expSk)@j & #j < #i)
%           & not(Ex tidI2 u2 v2 #i2. ExpCommitI(tidI2, u2, v2, expSk)@i2 & not(#i = #i2) ) )
%         | (Ex #j. LTKRev(v)@j & #j < #i) )"

% Code from July 22 commit
\begin{lstlisting}
lemma authInjAgreeGuaranteeForI:
     all-traces
     "All tidI u v expSk #i.
          ExpCommitI(tidI, u, v, expSk)@i ==>
          ( ( (Ex tidR #j. ExpRunningR(tidR, v, expSk)@j & #j < #i)
            & not( Ex tidI2 u2 v2 #i2. ExpCommitI(tidI2, u2, v2, expSk)@i2
                 & not(#i = #i2)
                 )
            )
          | (Ex #j. LTKRev(<u, v>)@j & #j < #i)
          | (Ex #j. LTKRev(u)@j & #j < #i)
          | (Ex #j. LTKRev(v)@j & #j < #i)
          )
     "
\end{lstlisting}

Note that this property \emph{does not hold when} the initiator is
running the \mStat{} method.
%
For that case we need to prove implicit authentication, as detailed in
the next section.

Similarly to the previous lemma, we require that injective agreement also holds
in the reverse direction:

\begin{lstlisting}
lemma authInjAgreeGuaranteeForR:
    all-traces
    "All tidR u v sk #i.
         (CommitR(tidR, u, v, sk)@i
	     & (All tidI #j m1. I1(tidI, u, v, m1) @ j ==> (All #k. TEE(u)@k ==> k < j) & (All #k. TEE(v)@k ==> k < j))
         & (All #j m1 m2. R2(tidR, v, m1, m2) @ j ==> (All #k. TEE(u)@k ==> k < j) & (All #k. TEE(v)@k ==> k < j)) )
         ==>
         ( ( (Ex tidI #j. ExpRunningI(tidI, u, v, sk)@j & #j < #i)
           & not(Ex tidR2 u2 v2 #i2. ExpCommitR(tidR2, u2, v2, sk)@i2 & not(#i = #i2)) )
         | (Ex #j. LTKRev(u)@j & #j < #i) )"
\end{lstlisting}

As the explicit and implicit authentication always correspond for the
responder authenticating with the initiator, here we do not need the
additional \mT{Exp} prefix to the running and commit events
(\mT{CommitI} and \mT{CommitR} respectively).

 
\subsubsection{Implicit Authentication}

The following lemma proves implicit authentication:
\begin{lstlisting}
lemma authGIYImplicitAuthGuaranteeForI:
    all-traces
    "All tidI u v impSk #i.
         CommitI(tidI, u, v, impSk)@i ==>
         ( ( (All tidR u2 v2 #j. CommitR(tidR, u2, v2, impSk)@j ==>
                (u = u2  &  v = v2)
             )
           &
             (not Ex #k. K(impSk)@k)
           &
             (not( Ex tidR u v #j tidR2 u2 v2 #j2.
                      ( CommitR(tidR,  u,  v,  impSk)@j
                      & CommitR(tidR2, u2, v2, impSk)@j2
                      & not(#j = #j2)
                      )
                 )
             )
           )
         | (Ex #k. LTKRev(u)@k) | (Ex #k. TEE(u)@k)
         | (Ex #k. LTKRev(v)@k) | (Ex #k. TEE(v)@k)
         )
         "
\end{lstlisting}

As opposed to lemma \mT{authInjAgreeGuaranteeForI}, here we prove that the two
parties implicitly authenticate on the keys \mT{impSk}. %
In this lemma we show that if any two parties (\mT{u} and \mT{v2} here) complete
a run of the protocol, and \mT{u} believes she is talking to \mT{v} and \mT{v2}
believes he is talking to \mT{u2}, then their identities match (that is,
\mT{u = u2} and \mT{v = v2}). Furthermore there is an injective correspondence
between the \mT{CommitI} and \mT{CommitR} events, and the attacker does not
learn the session key material.

 
\subsubsection{Secrecy, Forward Secrecy and Session Key Independence}

Finally, we prove secrecy of session keys, perfect forward secrecy
(PFS) and session key independence.
%
All these properties are validated by a unique lemma for each method,
as secrecy is a strictly weaker property than PFS (and hence follows
directly), and session key independence can be proven along PFS.
%
This is done by allowing the revelation of long term keys after either
the initiator or the responder have completed the protocol, and by
allowing to reveal the session keys.
%
It still holds that the session keys are secret for all the other runs
of the protocol.

We present the lemma for the \mSigStat{} method:
\begin{lstlisting}
  lemma secrecyPFSGIYSessionKey:
        all-traces
        "(All tid u v sk #i #j. (K(sk)@i & CommitI(tid, u, v, sk)@j) ==>
            ((Ex #l. LTKRev(u)@l & #l < #j) | (Ex #l. LTKRev(v)@l & #l < #j) | (Ex #l. SKRev(sk)@l) | (Ex w #l. TEE(w)@l))
         )
         &
         (All tid u v sk #i #j. (K(sk)@i & CommitR(tid, u, v, sk)@j) ==>
            ((Ex #l. LTKRev(u)@l & #l < #j) | (Ex #l. LTKRev(v)@l & #l < #j) | (Ex #l. SKRev(sk)@l) | (Ex w #l. TEE(w)@l))
            )"
\end{lstlisting}

%%% Local Variables:
%%% mode: latex
%%% TeX-master: "main"
%%% End:


%-------------------------------------------------------------------------- sec
\section{Discussion}
\label{sec:discussion}
There are a few instances where \mEdhoc{} can be improved,
which we found during this work and communicated to the authors. We discuss them below.
%

%-------------------------------------------------------------------------- sub
\subsection{Security Claim Justification}
\label{sec:securityClaims}
The \mSpec{} makes detailed claims about security properties that \mEdhoc{}
enjoys, which the authors assume hold because they reuse cryptographic cores
from existing academic protocols.
%
Specifically, in Section~8.1 of~\cite{selander-lake-edhoc-01}, the authors
claim that ``EDHOC inherits its security properties from the theoretical
\mbox{SIGMA-I}''.
%
The intention is the same for the other reused cryptographic
cores based on \mOptls{} and \mNoise{}, but since the \mSpec{} is still work in
progress, it is not yet written down~\cite{personalCommunication}.
%

While it is good practice to reuse well-studied academic components in
industrial standards, it is important to justify that changes made to these
components preserve security properties.
%
Some properties have been verified~\cite{DBLP:conf/secsr/BruniJPS18} and this paper verifies more, but until some justification is provided, security
claims may benefit from a note of caution.

%-------------------------------------------------------------------------- sub
\subsection{Unclear Intended Use}
\label{sec:unclearProtocolUse}
Formal verification methodologies often clash with industrial standard
development practices; this is true also in our case.
%
The most important reason for the clash is that formal verification aims to
verify whether well-specified and detailed security goals are met, whereas
industrial standards are developed with a clear abstract goal,
but one that lacks specificity.
%
Instead the goals are there typically made more specific by listing resistance
to attacks as these are discovered throughout the work.
%
Granted, as discussed in Section~\ref{sec:claimedProperties}, the \mSpec{} lists
several specific security goals.
%
However, without knowing how the protocol is to be used,
it is not clear whether these are the most important goals for constrained IoT
devices.
%

The abstract goal of \mEdhoc{} is simple: establish an \mOscore{} security
context using few roundtrips and small messages.
%
From that, the design of \mEdhoc{} is mainly driven by what
can be achieved given the technical restrictions.
%
Focusing too much on what can be achieved within given restrictions, and paying
too little attention to the use cases where the
protocol is to be used and their specific goals, risks resulting in
sub-optimal trade-offs and design decisions.
%

\mEdhoc{} is intended to cover a variety of use cases, many of which are
difficult to predict today.
%
Just because it is difficult to predict these use cases, does not
prevent one from collecting \emph{typical} use cases and user stories
to identify more specific security goals that will be important in most cases.
%

While constructing our model, we came up with simple user stories to identify
security properties of interest.
%
Several of these revealed undefined aspects of \mEdhoc{}.
%
We informed the \mEdhoc{} authors, who then included these aspects the \mSpec{}.
%
We present a couple of examples here.\\
%

%\subsubsection{Non-repudiation.}
\runhead{Non-repudiation}
An access control solution for a nuclear power-plant may need to log who is
passing through a door, whereas it may be undesirable for, say, a coffee
machine to log a list of users along with their coffee preferences.
%
Via this simple thought experiment, we realized that the \mSpec{} did not
consider non-repudiation.
%
In response, the authors of the \mSpec{} added a paragraph about which methods
provided which types of (non)-repudiation~\cite{selander-lake-edhoc-01}. \\

%\subsubsection{Unintended Peer Authentication.}
\runhead{Unintended Peer Authentication}
Section~3.2 of the \mSpec{} states that parties must be configured
with a policy restricting the set of peers they can run \mEdhoc{} with.
%
However, the initiator is not required to verify that the \mIdcredr{} received
in the second message is the same as the one intended at initialization.
%
The following thought experiment shows why such a policy is insufficient.
%

Assume that someone has configured all devices in their home to be in the
allowed set of devices, but that one of the devices ($A$) is compromised.
%
If another device $B$, initiates a connection to a third device $C$, the
compromised device $A$ may interfere by responding in $C$'s place, blocking
the legitimate response from $C$.
%
Since $B$ does not verify that the received identity in the second message
matches the intended identity $C$, and device $A$ is part of the allowed set,
$B$ will complete and accept the \mEdhoc{} run with device $A$ instead of the
intended $C$.
%
The obvious solution is for the initiator to match \mIdcredr{} to the intended
identity provided by the application.
%
We have communicated this situation to the \mEdhoc{} authors and they are considering
how to resolve the issue.
%

%------------------------------------------------------------------------- sub
\subsection{Unclear Security Model}
When designing a security protocol, the attacker's capabilities must be
considered so that it is possible to determine whether the protocol is
sufficiently secure.
%
That is, a security model in which the protocol is deemed
secure needs to be defined at least on a high level.
%
We argue that the \mSpec{} gives too little information about what capabilities
an attacker is assumed to have, and that this leads to unclear design goals and
potentially sub-optimal design.
%
%Let us explore this via an example.
%

It is conceivable that IoT devices deployed in a hostile environments can be
hardened by equipping them with a TEE, but \mEdhoc{} is not %intentionally
designed to take advantage of this.
%
Even though \mEdhoc{} incorporates cryptographic cores from different academic
security protocols, its design does not take into account the attacker models
for which these protocols were designed.
%
For example, \mOptls{} is designed to be secure in the CK
model~\cite{DBLP:conf/crypto/CanettiK02}.
%
The CK security model explicitly separates the secure storage of long-term
credentials from storage of session state and ephemeral keys to model the 
use of TEEs.
%

The \mEdhoc{} authors indicated to us that it was
not necessary to consider compromised ephemeral session keys separately from
from compromised long-term keys.
%
The rationale is that \mSigma{} cannot protect against compromised ephemeral
keys~\cite{personalCommunication}.
%
That rationale is presumably based on the fact that the \mSigSig{} method is
closely modeled on \mSigmaI{} and that it would be preferable to obtain a
homogeneous security level among the \mEdhoc{}
methods.
%
That rationale is only true, however, if one restricts attention to session key
confidentiality of an ongoing session.
%
TEEs provide value in other ways, for example, by allowing contractions with
weak PCS guarantees.
%
It would be wasteful to not consider TEEs for long-term key storage as part of
the security, since it could otherwise make use of many good properties of the
re-used cryptographic cores, like \mOptls.
%
%For example, coming back to \mOptls{}, which was intentionally
%designed to be provably secure in the CK model.
%\vnote{Incomplete sentence?}
%

%-------------------------------------------------------------------------- sub
\subsection{Session Key Material}
\label{sec:sessionKeyMaterial}
\mEdhoc{} establishes session key material, from which session keys
can be derived using the \mEdhoc{}-Exporter.
%
The session key material is affected by \mGxy{}, and if a party uses the
\mStat{} method, also by that party's secret static long-term DH key.
%
As shown in Section~\ref{sec:formalization}, mutual injective agreement cannot
be achieved for \mGiy{}.
%
If this property is not important for constrained IoT devices which cannot use
any of the other methods, then one can simply accept that the methods have
different authentication strengths.
%
However, if it is important, this is a problem.
%

We identified three alternatives for resolving this.
%
One is to include a fourth message from responder to initiator,
carrying a MAC based on a key derived from session key
material including \mGiy{}.
%
Successful MAC verification guarantees
to the initiator that the responder injectively agrees on \mGiy{}.
%
However, our understanding is that adding an extra message is
undesirable, since~\cite{selander-lake-edhoc-01} requests as few messages as
possible.
%

Another possibility is to include \mGi{}, or its hash, in the first and
second messages.
%
This would, however, increase message sizes, a grave concern for \mEdhoc{}.
%and prevent initiator identity protection.
%

A third alternative is to not derive the session key material from \mGiy{}.
%
Doing so would destroy the protection \mOptls{} provides against compromise
of the initiator's ephemeral DH key.
%

Regardless of how this problem is handled, we have verified that all methods
share a common, but weaker, security property: mutual implicit authentication
on all of \mGxy{}, \mGiy{} and \mGrx{}.
%

%-------------------------------------------------------------------------- sub
\subsection{Cipher Suite Negotiation}
\label{sec:ciphersuiteNegotiation}
%\knote{If we are tight on space, we can consider removing this section since we
%    don't verify any negotiation. We did have impact on the \mEdhoc{} \mSpec
%    with this though.
%}
%

Cipher suite negotiation in \mEdhoc{} spans one or more executions of the
protocol.
%
As part of the first message, the initiator proposes an ordered list of cipher suites they support (see Section~\ref{sec:edhoc}, Figure~\ref{fig:edhocFramework}). The responder either accepts the highest entry in this list (if they also support that suite) or makes a counter-proposal, namely the highest entry which they do support. If there is no such entry the responder rejects the suite entirely, and the protocol does not continue. 
%
If a run terminates due to the proposed cipher suites being rejected by the
responder, the initiator maintains state and initiates a new run, proposing
an updated set of cipher suites.
%
%Our model does not cover this, and we leave it for future work.
%Maintaining state between protocol runs 
This implicitly creates a long-lived
meta-session covering multiple \mEdhoc{} sessions.
%
%Such a meta-session is presumably controlled by the underlying application.
%

However, the time for which the initiator should
remember a rejected cipher suite for a given responder is not specified.
%
From a security perspective, remembering the rejected cipher suite for the
next \mEdhoc{} run in the same meta-session would be sufficient.
%
If the responder is updated with a new cipher suite before the next such
session, this could be taken into account. On the other hand, caching the
rejected cipher suite between meta-sessions would reduce the number of
round-trips for subsequent runs, should the responder not have been updated.
%
This needs to be clearly specified, and we have conveyed this to the authors of the \mSpec.

%-------------------------------------------------------------------------- sec
\section{Conclusions and future work}
\label{sec:conclusions}
We formally modeled all five
methods of the \mEdhoc{} key establishment protocol using the \mTamarin{} tool, and
formulated and verified several important security properties in this model --
PFS; session key material independence; mutual injective agreement on roles, identities and session key material
(except for when the initiator uses the \mStat{} method); implicit
authentication on on the same parameters for all methods; consitency, aliveness
and peer awareness (the last two when mutual injective agreement holds);
entity authentication (also when mutual injective agreement holds).
%
%We also identified security properties that do not hold for all methods.
%
Most importantly, we found that injective agreement on \mGiy{} does not hold for
initiators when they use the \mStat{} method.
%
We verified each method in isolation, and leave as future work to verify whether
the methods are secure under composition.
%

Further, we identified a situation where initiators may establish an \mOscore{}
security context with a different party than the application using \mEdhoc{}
intended, and proposed a simple mitigation.
%
We discussed how the IETF may extract and better define security properties to
enable easier verification.
%
There is some work that was done as part of this effort which has not been mentioned in this paper. These lines of enquiry form the basis for future work.
%For more details, see~\cite{edhocTamarinRepo}.

We considered two more properties, namely key-compromise impersonation (KCI) resistance and weak post-compromise security (PCS). A protocol is said to have KCI resistance if a party $I$ can establish a secure session with party $R$, even if $I$'s keys are compromised at any time, and $R$'s key material is not leaked until the end of the session. The authors of \mSpec{} claim that \mEdhoc{} has KCI resistance. Weak PCS~\cite{cohn2016post} is said to hold if a session key established in a run $\rho$ between parties $I$ and $R$ continues to be secret between them, even if $\rho$ was such that the attacker could perform cryptographic operations using $I$ and $R$'s long term keys (without having access to the keys themselves). We verified that KCI resistance and weak PCS guarantees hold for the \mSigSig{} and \mPskPsk{} methods.
However, the tool does not terminate for these properties on the other methods, and therefore, we do not know whether the \mStat-based methods enjoy these properties or not. We relegate the verification of these properties for the \mStat-based methods to future work. 

We also tried to incorporate the parameters \mCi, \mCr, and \mAD{} into our model. This too resulted in non-termination for some of the methods, and further study is required to obtain conclusive results along these lines. 

Another potential extension is to incorporate the cipher suite negotiation process into the formal modeling.\\ 
%

%-------------------------------------------------------------------------- ack
% Should be a run-in heading.  subsubsection works in llncs2e document class
%\runhead{Acknowledgments} This work was partially supported by
%the Wallenberg AI, Autonomous Systems and Software Program (WASP) funded by
%the Knut and Alice Wallenberg Foundation.
%%
%We are grateful to G\"oran Selander, John Mattsson and Francesca Palombini for
%clarifications regarding the specification.
%

%\vnote{We need to fix the link for the code repository.}

%-------------------------------------------------------------------------- bib
\bibliographystyle{spmpsci}
\bibliography{refv}
\end{document}
