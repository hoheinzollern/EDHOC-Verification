\documentclass[runningheads,draft,x11names]{llncs}
\usepackage{amsmath,amssymb,amsfonts}
\usepackage[scaled=0.8]{helvet}    % Less huge \textsf{functionName}
\usepackage{enumitem}       % compacts lists and stuff
\usepackage[subtle]{savetrees}
\usepackage{soul}           % \hl for highlighting text; \st for strike-through
\usepackage{graphicx}
\usepackage[dvipsnames]{xcolor}
\usepackage{wrapfig}
\usepackage{tikz}
\usetikzlibrary{trees,snakes,arrows}
\usetikzlibrary{shapes,chains}
\usetikzlibrary{positioning}
\usepackage{url}
\usepackage{hyperref}
\hypersetup{
    final, % Uncomment to remove all links (useful for printing in black and white)
    colorlinks=true, breaklinks=true, bookmarks=false,bookmarksnumbered,
    urlcolor=blue, linkcolor=black, citecolor=green, % Link colors
    pdfkeywords={}, % PDF Keywords
    pdfcreator={PdfLaTeX}, % PDF Creator
  }

\usepackage{ifthen}

% -- Fonts and styles for names
\newcommand{\mFunStyle}[1]{\textsf{#1}}
\newcommand{\mConstStyle}[1]{\textsf{#1}}
\newcommand{\mVarStyle}[1]{\mathit{#1}}
\newcommand{\mFactStyle}[1]{\textsf{#1}}
\newcommand{\mMethodStyle}[1]{\mConstStyle{#1}}
\newcommand{\mProtocolStyle}[1]{\text{#1}}

% -- Macros for consistent wording
\newcommand{\mSpec}{specification}  % The EDHOC spec document we are analyzing

% -- Domain specific macros
\newcommand{\mArxiv}{\texttt{arXiv}}
\newcommand{\mTamarin}{\mProtocolStyle{Tamarin}}
\newcommand{\mProverif}{\mProtocolStyle{ProVerif}}
\newcommand{\mEdhoc}{\mProtocolStyle{EDHOC}}
\newcommand{\mOscore}{\mProtocolStyle{OSCORE}}
\newcommand{\mSigma}{\mProtocolStyle{SIGMA}}
\newcommand{\mSigmaI}{\mProtocolStyle{SIGMA\nobreakdash-I}}
\newcommand{\mCbor}{\mProtocolStyle{CBOR}}
\newcommand{\mCose}{\mProtocolStyle{COSE}}
\newcommand{\mCoseEncrypt}{\mProtocolStyle{COSE\_Encrypt0}}
\newcommand{\mCoseSign}{\mProtocolStyle{Cose\_Sign1}}
\newcommand{\mHkdf}{\mProtocolStyle{HKDF}}
\newcommand{\mHkdfExtract}{\mProtocolStyle{HKDF\nobreakdash-extract}}
\newcommand{\mHkdfExpand}{\mProtocolStyle{HKDF\nobreakdash-expand}}
\newcommand{\mHmac}{\mProtocolStyle{HMAC}}
\newcommand{\mAead}{\mProtocolStyle{AEAD}}
\newcommand{\mAeadDecrypt}{\mProtocolStyle{AEAD\nobreakdash-decrypt}}
\newcommand{\mDecrypt}{\mProtocolStyle{decrypt}}
\newcommand{\mOptls}{\mProtocolStyle{OPTLS}}
\newcommand{\mNoise}{\mProtocolStyle{Noise}}
\newcommand{\mTls}{\mProtocolStyle{TLS}}
\newcommand{\mDandTls}{\mProtocolStyle{(D)TLS}}
\newcommand{\mCtls}{\mProtocolStyle{cTLS}}
\newcommand{\mCoap}{\mProtocolStyle{CoAP}}

\newcommand{\mStat}{\mMethodStyle{STAT}}
\newcommand{\mSig}{\mMethodStyle{SIG}}
\newcommand{\mPsk}{\mMethodStyle{PSK}}
\newcommand{\mStatStat}{\mMethodStyle{STAT-STAT}}
\newcommand{\mStatSig}{\mMethodStyle{STAT-SIG}}
\newcommand{\mSigStat}{\mMethodStyle{SIG-STAT}}
\newcommand{\mSigSig}{\mMethodStyle{SIG-SIG}}
\newcommand{\mPskPsk}{\mMethodStyle{PSK-PSK}}

\newcommand{\mSid}{\mConstStyle{sid}}    % session id = (u, v, s-key)

\newcommand{\mXor}{\mConstStyle{XOR}}
\newcommand{\mSuites}{\mConstStyle{Suites\_I}}
\newcommand{\mMethod}{\mConstStyle{Method}}
\newcommand{\mCi}{\mConstStyle{C\_I}}
\newcommand{\mCr}{\mConstStyle{C\_R}}
\newcommand{\mGi}{\mConstStyle{G\_I}}
\newcommand{\mGr}{\mConstStyle{G\_R}}
\newcommand{\mGiy}{\mConstStyle{G\_IY}}
\newcommand{\mGrx}{\mConstStyle{G\_RX}}
\newcommand{\mGx}{\mConstStyle{G\_X}}
\newcommand{\mGy}{\mConstStyle{G\_Y}}
\newcommand{\mGxy}{\mConstStyle{G\_XY}}
\newcommand{\mIDPsk}{\mConstStyle{ID\_PSK}}
\newcommand{\mTH}{\mConstStyle{TH}}
\newcommand{\mTHtwo}{\mConstStyle{TH\_2}}
\newcommand{\mKtwoe}{\mConstStyle{K\_2e}}
\newcommand{\mKtwom}{\mConstStyle{K\_2m}}
\newcommand{\mKtwoae}{\mConstStyle{K\_2ae}}
\newcommand{\mSign}{\mConstStyle{sign}}

\newcommand{\mKthreeae}{\mConstStyle{K\_3ae}}
\newcommand{\mKthreem}{\mConstStyle{K\_3m}}

\newcommand{\mTHthree}{\mConstStyle{TH\_3}}
\newcommand{\mhplain}{\mConstStyle{h''}}
\newcommand{\mCredi}{\mConstStyle{CRED\_I}}
\newcommand{\mCredr}{\mConstStyle{CRED\_R}}
\newcommand{\mHash}{\mConstStyle{H}}

\newcommand{\mTHfour}{\mConstStyle{TH\_4}}
\newcommand{\mAuthi}{\mConstStyle{Auth\_I}}
\newcommand{\mAuthr}{\mConstStyle{Auth\_R}}

\newcommand{\mMactwo}{\mConstStyle{MAC\_2}}
\newcommand{\mMacthree}{\mConstStyle{MAC\_3}}

\newcommand{\mSigtwo}{\mConstStyle{Sig\_2}}
\newcommand{\mSigthree}{\mConstStyle{Sig\_3}}

\newcommand{\mMsgone}{\mConstStyle{m1}}
\newcommand{\mMsgtwo}{\mConstStyle{m2}}
\newcommand{\mMsgthree}{\mConstStyle{m3}}

\newcommand{\mCipher}{\mConstStyle{cipher\_2}}

\newcommand{\mAD}{\mConstStyle{AD}}
\newcommand{\mADone}{\mConstStyle{AD\_1}}
\newcommand{\mADtwo}{\mConstStyle{AD\_2}}
\newcommand{\mADthree}{\mConstStyle{AD\_3}}

\newcommand{\mPRK}{\mConstStyle{PRK}}
\newcommand{\mPRKtwo}{\mConstStyle{PRK\_2e}}
\newcommand{\mPRKthree}{\mConstStyle{PRK\_3e2m}}
\newcommand{\mPRKfour}{\mConstStyle{PRK\_4x3m}}

\newcommand{\mIdcredi}{\mConstStyle{ID\_CRED\_I}}
\newcommand{\mIdcredr}{\mConstStyle{ID\_CRED\_R}}
\newcommand{\mLtki}{\mConstStyle{ltk\_I}}
\newcommand{\mLtkr}{\mConstStyle{ltk\_R}}
\newcommand{\mLtk}{\mConstStyle{ltk}}

% TIKZ messages and actions
\newcommand{\msg}[4]{\draw[->,thick] ([yshift=-#1]#2.south) coordinate (l1)--(l1-|#3) node[midway, above]{#4}}
\newcommand{\action}[3]{\node[draw,thick,fill=white,align=center,below={#1} of {#2}]{#3}}

% Tamarin symbols
\newcommand{\ifarrow}[1]{\ensuremath{\mathit{\,-\hspace{-2.4pt}[{#1}]\hspace{-4.95pt}\rightarrow\,}}}
\newcommand{\semarrow}[1]{\ensuremath{\mathit{\,=\hspace{-4.9pt}[{#1}]\hspace{-5pt}\Rightarrow\,}}}
\newcommand{\mIn}{\mathsf{In}}
\newcommand{\mOut}{\mathsf{Out}}
\newcommand{\mFr}{\mathsf{Fr}}
\newcommand{\mKD}{\mathsf{KD}}
\newcommand{\mKU}{\mathsf{KU}}
\newcommand{\mT}[1]{\lstinline[basicstyle=\sffamily\normalsize]{#1}} % Tamarin code inline
\usepackage[final]{listings}

\lstdefinestyle{mystyle}{
    backgroundcolor=,   
    commentstyle=\color{Green},
    identifierstyle=\color{black},
    keywordstyle=\color{ForestGreen},
    numberstyle=\color{Gray},
    stringstyle=,
    basicstyle=\sffamily\small,
    keywords={let,in,rule,restriction,axiom,lemma,all-traces,exists-trace,Ex,All,Fr,In,Out},
    breakatwhitespace=false,
    breaklines=true,
    captionpos=b,
    keepspaces=false,
    numbers=left,
    numbersep=5pt,
    showspaces=false,
    showstringspaces=false,
    showtabs=false,
    tabsize=2,
    morecomment=[l]{//},
    literate=
    {=}{$=$}{1}
    {[}{$[$}{1}%
    {]}{$]$}{1}%
    {<}{$\langle$}{1}%
    {>}{$\rangle$}{1}%
    {(}{$($}{1}%
    {)}{$)$}{1}%
    {&}{$\wedge$}{1}
    {|}{$\vee$}{1}
    {$}{\$}{1}
    {<\ \#}{$<$\ \ \#}{3}% trick here: we distinguish angle brackets from temporal comparisons because of the # for the variable on the rhs
    {--[}{$\mbox{-\hspace{-4.0pt}[}$}{3}%
    {]->}{$\mbox{]\hspace{-4.2pt}\rightarrow}$}{3}%
    {-->}{$\rightarrow$}{1}
    {==>}{$\Rightarrow$}{1}
    {XOR}{$\oplus$}{1}%
    {aeadEncrypt}{\mAead{}}{6}%
    {decrypt}{\mDecrypt}{6}
    {aeadDecrypt}{\mAeadDecrypt}{11}
    {hkdfExtract}{\mHkdfExtract}{14}
    {hkdfExpand}{\mHkdfExpand}{14}
    {All\ }{$\forall$\ }{3}
    {Ex\ }{$\exists$\ }{3}
    {h(}{H$($}{2}
    {~}{{\url{~}}}{1}
    }
\lstset{style=mystyle}


\begin{document}
\title{Formal Analysis of EDHOC Key Establishment for Constrained IoT Devices}
\author{Karl Norrman\inst{1,2}\orcidID{0000-0003-0164-1478} \and
Vaishnavi Sundararajan\inst{3} \and
Alessandro Bruni\inst{4}
}
%
\authorrunning{K. Norrman et al.}
% First names are abbreviated in the running head.
% If there are more than two authors, 'et al.' is used.
%
\institute{
    KTH Royal Institute of Technology, Stockholm, Sweden \and
    Ericsson Research, Security, Stockholm, Sweden,
    \email{karl.norrman@ericsson.com} \and
    Ericsson Research, Int. Auton. Systems, Bangalore, India, \email{vaishnavi.sundararajan} \and
    IT University of Copenhagen, Copenhagen, Denmark, \email{brun@itu.dk}
}
%
\maketitle
%

\begin{abstract}
%\hl{250 words}
    IETF is standardizing a key establishment protocol named \mEdhoc{} for
constrained IoT devices~\cite{selander-lake-edhoc-01}.
%
In contrast to more powerful IoT devices, such as web cameras and cars,
which receive most attention from media, constrained devices often have severe
restrictions on energy consumption.
%
Additionally, they often use specialized wireless communication links with
demanding constraints on message sizes, which may also vary between messages.
%
\mEdhoc{} was first formally analyzed by
Bruni~et.~al.~\cite{DBLP:conf/secsr/BruniJPS18}.
%
Since then, the protocol has been significantly extended and is now a
framework with a number of cryptographic cores, called methods.
%
The initial version of \mEdhoc{} contained only two out of the current five
methods~\cite{selander-ace-cose-ecdhe-08}.
%
In this paper we formally analyze all methods of \mEdhoc{} in a symbolic
Dolev-Yao model, using the \mTamarin{} verification tool.
%
We show that the different methods provide sensible, but also rather
heterogeneous security properties, and discuss consequences of this.
%
\end{abstract}
%

%-------------------------------------------------------------------------- sec
\section{Introduction}
\label{sec:introduction}
%-------------------------------------------------------------------------- sub
\subsection{Background and motivation}
\label{sec:motivation}
IoT security threats involving cars, web-cameras and other resourceful devices
receive most attention from media and academia.
%
These devices are computationally strong with no severe bandwidth or energy
consumption restrictions.
%
Securing the communication between such devices can readily be done using
\mDandTls.
%
Constrained devices, on the other hand, on which bandwidth and
energy consumption restrictions are common, have received much less attention.
%
These devices may be simple sensors with the only task of relaying
measurements of their physical environment to a server every hour, but doing so
autonomously for a decade without maintenance.
%
To keep energy consumption down, highly specialized radio links with small
and heterogeneous frame sizes are sometimes used.
%
IETF defined the Constrained Application Protocol (\mCoap{}) protocol for data
transport in such situations~\cite{rfc7252}.
%
\mCoap{} does not include security protection on its own.
%
In some cases, \mDandTls{} messages are too large to fit into the radio frames.
%
This is one of the reasons IETF standardized the Object Security for
Constrained RESTful Environments (\mOscore{}) protocol to secure
communications between constrained devices, as a complement for when
\mDandTls{} is too heavy weight~\cite{rfc8613}.
%

The \mOscore{} protocol requires a pre-established security context.
%
For a couple of years, the IETF Lightweight Authenticated Key Exchange (LAKE)
working group has been developing requirements and mechanisms for a key
exchange protocol, named \mEdhoc~\cite{selander-lake-edhoc-01}, capable of
establishing \mOscore{} security contexts.
%
Naturally, \mEdhoc{} must work under the same constrained requirements as
\mOscore{} itself.
%

While use cases for \mEdhoc{} are not firmly set,
the overall goal of \mEdhoc{} is to establish an \mOscore{} security
context, keeping messages small being the most prominent driver for the
design.
%
Discussions on the LAKE working group mailing list explored whether a
compressed version of \mTls, named \mCtls~\cite{ietf-tls-ctls-00}, would suffice
for the same situations that
the combination of \mOscore{} and \mEdhoc{} aim for.
%
They concluded to proceed developing \mEdhoc{} in parallel to the \mTls{}
working groups efforts to develop \mCtls.
%

The \mEdhoc{} protocol has evolved significantly over time to cater for smaller
messages and more use cases.
%
The first incarnation of \mEdhoc{} appeared in March 2016.
%
It contained two different cryptographic cores, one based on a
pre-shared Diffie-Hellman (DH) key and a second following a draft of the
challenge-response signatures, in the style of the \mNoise{}
framework~\cite{perrin2016noise}.
%
The latter was then replaced by \mSigma, and this version, from May 2018, was
formally analyzed by Bruni~et.~al.~\cite{DBLP:conf/secsr/BruniJPS18}.
%
The protocol has now further evolved and variants using challenge-response
signatures have been added again by integrating the cryptographic core of
\mOptls{}.
%
On top of this, mixed variants where one party uses a challenge-response
signature and the other a regular signature have also been added.
%
Consequently, there are now five cryptographic cores in total, and it is prudent
to formally analyze them all to ensure a higher level of security assurance for
\mEdhoc.
%
This is especially important, since the \mSpec{} itself lacks a description
of the intended security model and overall security goals.
%
Discussing how this gap can be filled, and how to identify what the expected
security goals should be, is an important part of this paper.
%

%-------------------------------------------------------------------------- sub
\subsection{Contributions}
\label{sec:contributions}
We have reviewed and analyzed the \mEdhoc{} protocol, which has continuously
evolved, at times based on our feedback.
%
The analysis spanned over roughly five months, but the formal models and proven
properties are based on to the version specified in the draft as it was on
2020-03-01~\cite{selander-lake-edhoc-01}.
%

Our main contributions are the following.
\begin{itemize}
    \item We provide formalization of all five cryptographic cores of \mEdhoc{}
        using \mTamarin~\cite{DBLP:conf/cav/MeierSCB13}.
    \item We give an explicit security model for the protocol and have verified
        essential security properties, such as session key and entity
        authentication, as well as Perfect Forward Secrecy (PFS), within that
        model.
        From these explicitly proven properties other properties follow, e.g.,
        session key independence.
    \item We discuss the relation between proven properties, potentially missing
        properties and the lack of clear design goals and security model for
        \mEdhoc{}.
        Based on this we give recommendations for the future standards
        development in regards to clarity, not only in terms of under which
        technical restrictions the protocol must work, but also what its
        envisioned uses are.
    \item Our discussions with members of the IETF LAKE working group have already
        lead to improvements and clarifications of the standard \mSpec{} based
        on observations we made during the construction of our formal model.
\end{itemize}

%-------------------------------------------------------------------------- sub
\subsection{Related work}
\label{sec:relatedWork}
The work closest to ours is Bruni~et.~al.~\cite{DBLP:conf/secsr/BruniJPS18},
which used \mProverif~\cite{DBLP:conf/csfw/Blanchet01} to analyze an earlier,
two-cryptographic core, version of \mEdhoc~\cite{selander-ace-cose-ecdhe-08}.
%
We consider our work to be a sort of follow-up to that, doing a similar kind of
analysis of the most recent, and more elaborate, version of \mEdhoc{} with its
five cores.
%
We do, however, take a clean slate approach and look at the protocol as new.
%
For example, since the model has undergone so much change, the set of properties
which we verify is also different.
%
The \mTamarin{} tool has been used to verify many other protocols, perhaps
closest to our work is Cremers~et.~al's analysis of
\mTls~\cite{DBLP:conf/ccs/CremersHHSM17}.
%
Some of the cryptographic cores themselves have been analyzed in the
computational model, e.g., \mSigma{} by Canetti and
Krawczyk~\cite{DBLP:conf/crypto/CanettiK02}, \mOptls{} by Krawczyk and
Wee~\cite{DBLP:conf/eurosp/KrawczykW16}, and \mNoise{} by
Kobeissi~et.~al.~\cite{DBLP:conf/eurosp/KobeissiNB19}.
%
Computational models often rely on implicit session key authentication
(see for example the definition of SK-security in the Canetti-Krawczyk
model~\cite{DBLP:conf/crypto/CanettiK02}).
%
Although symbolic models predominantly rely on correspondence properties in the
style of Lowe~\cite{DBLP:conf/csfw/Lowe97a}, there are examples where implicit
authentication has been used.
%
For example, Schmidt~et.~al.~\cite{DBLP:conf/csfw/SchmidtMCB12} use a
symbolized version of an extended Canetti-Krawzcyk model.
%

%-------------------------------------------------------------------------- sec
\section{The \mEdhoc{} protocol}
\label{sec:edhoc}
% !TEX root =  main.tex

%Description of \m\mEdhoc and main changes from last verified version

\vnote{I find the macros for the protocol, method, and tool names distracting while reading through (the font changes too much, too often). The macro for EDHOC actually inserts a line break if you start a sentence with it, because of the \texttt{hbox} (I'm not sure why the hbox exists for what is a macro to be inserted in running text). I do need to start sentence with EDHOC many times though, so this is irritating. It is also the case that ProVerif needs to be written like so, and not capitalized entirely, so at least one of them is plain wrong! For now I've left in the macros, but I'd prefer that we fixed them to be regular capitalized text or, in the case of ProVerif, in camelcase as they should be.}

\subsection{Overview}
Constrained IoT systems often deal with a lot of valuable personal and business information that ought to be kept secure. Such systems need to be assured of end-to-end protection with source authentication and perfect forward secrecy. It is often desirable to protect such devices at the application layer -- for example, in cases where transport layer security is not sufficient [\mcneed], or where multiple underlying protocols need to be accounted for. One method for providing application layer security is provided by CBOR Object Signing and Encryption (COSE) [RFC8152: \mcfix].  

In order to derive shared key material with which to proceed, communicating parties can run an Elliptic Curve Diffie-Hellman key exchange protocol with ephemeral keys. Ephemeral Diffie-Hellman Over COSE (\mEdhoc) is a lightweight key exchange protocol for such situations, and is expected to provide perfect forward secrecy and identity protection. \mEdhoc supports authentication using pre-shared keys (PSK), raw public keys (RPK), and public key certificates. After successful completion of the \mEdhoc protocol, application keys and other application specific data can be derived using the \mEdhoc-Exporter interface. 

A main use case for \mEdhoc is to establish a security context for Object Security for Constrained RESTful Environments (\mOscore) [RFC8613: \mcfix]. \mOscore is a protocol which uses COSE for application-layer protection on top of the transport-layer Constrained Application Protocol (CoAP). \mEdhoc uses COSE for cryptography, CBOR for encoding, and CoAP for transport. By reusing existing libraries, the additional code footprint can be kept very low.

\mEdhoc is designed to work in highly constrained scenarios. This makes it especially suitable for network technologies which have low throughput, low power consumption, and small frame sizes. Examples include Cellular IoT, 6TiSCH, and LoRaWAN [\mcneed].

\subsection{Background, comparison with~\cite{DBLP:conf/secsr/BruniJPS18}}
The first version of \mEdhoc was proposed in March 2016 to a working group investigating lightweight authenticated key exchange protocols [\mcneed]. There has been a focus on formally verifying that the protocol satisfies the properties expected of it right from the beginning. 

The 2018 work~\cite{DBLP:conf/secsr/BruniJPS18} by Bruni et al performed a formal verification of version 08 [\url{https://tools.ietf.org/html/draft-selander-ace-cose-ecdhe-08} \mcfix] of \mEdhoc. The protocol and properties are modelled and verified in the \mProverif tool. This version of the protocol belongs to the \mSigmaI family of protocols, and has two modes -- one with asymmetric keys, and one with pre-shared symmetric keys (PSK). Bruni et al showed that this version satisfies the requisite properties of identity protection, (perfect forward) secrecy of data, and strong authentication, upon completion of the protocol.

\mEdhoc has undergone a lot of change since version 08, as will be described in the following sections, and the formal verification of the current version, therefore, is a worthwhile exercise.

%\subsection{Overarching goal of security context establishment}
\subsection{Methods and features of \textsc{EDHOC}}
\mEdhoc can established Diffie-Hellman key exchange in one of three different ways -- using digital signatures, static Diffie-Hellman keys, or pre-shared symmetric keys. We describe each of these methods in detail below. The communicating parties must agree on the method as part of the first message.

\subsubsection{\mSigSig method}
The \mSigma (SIGn-and-MAc) family of protocols [\mcneed] has many variants. The \mSigSig method of \mEdhoc is built on \mSigmaI, a variant of the \mSigma protocol which provides identity protection for the initiator, and  implements the \mSigmaI variant as Mac-then-Sign. The current \mSigSig method corresponds (with a few minor changes) to the asymmetric key mode of \mEdhoc v08. 

The communicating parties exchange ephemeral public keys, compute the shared secret, and derive symmetric application keys from this secret.

\subsubsection{\mPskPsk method}
In this method, the initiator and responder are assumed to have a pre-shared key which is secret to them, and can be retrieved by the responder using a public part of the first message (\mIDPSK). This method corresponds to the symmetric key method of \mEdhoc v08. 

In the first message, the initiator sends a message consisting of the method name, the initiator's ephemeral key (\mGx), their connection identifier (\mCi), the \mIDPSK identifier, and (optional) plaintext (\mADone). 

The second message, sent by the responder, is composed of \mCi, the responder's ephemeral key \mGy, their connection identifier \mCr, and an AEAD encryption [\mcneed] of the transaction hash of the first message (\mTHtwo) along with an optional plaintext \mADtwo. The key used for this (\mKtwo) is derived using the EDHOC key derivation function with \mTHtwo and the pseudorandom string \mPRKtwo as input, while the associated data for the AEAD encryption is constructed by concatenating a constant string, the hash function, and \mTHtwo. 

The initiator, upon receipt of this message, sends back \mCr, followed by an AEAD encryption of the transaction hash of the second message (\mTHthree) along with a plaintext \mADthree. As earlier, the key used is derived by supplying \mTHthree and \mPRKthree as input to the KDF, and the  associated data obtained by concatenating the constant string, hash function, and \mTHthree. An abstract description is shown in Figure~\ref{fig:edhocpsk}.

\vnote{Figure numbers don't square up, check class file guidelines!}

\begin{figure}\label{fig:edhocpsk}
\centering
\includegraphics[scale=0.3]{Images/psk.png}
\caption{The PSK-PSK method of EDHOC}
\end{figure}

\subsubsection{\mStat-based methods}
\mEdhoc allows for three \mStat-based methods -- two where only one participant has a static Diffie-Hellman key (while the other uses signatures), and one where both do. This set of methods is not covered in v08 of \mEdhoc, which only has a single \mSigma asymmetric key method (corresponding to the \mSigSig method shown above). This allows one party to use a \mSigma style of authentication, while the other can use something along the lines of \mOptls.

\vnote{Need to talk about links to OPTLS and NOISE}



\subsection{Expected security properties}


%Discussion of KDF -- page 13 of EDHOC

%certain security applications may be integrated into EDHOC by transporting auxiliary data together with the messages. One example is the transport of third-party authorization information protected outside of EDHOC [I-D.selander-ace-ake-authz]. Another example is the embedding of a certificate enrolment request or a newly issued certificate. EDHOC allows opaque auxiliary data (AD) to be sent in the EDHOC messages. Unprotected Auxiliary Data (AD_1, AD_2) may be sent in message_1 and message_2, respectively. Protected Auxiliary Data (AD_3) may be sent in message_3. Since data carried in AD1 and AD2 may not be protected, and the content of AD3 is available to both the Initiator and the Responder, special considerations need to be made such that the availability of the data a) does not violate security and privacy requirements of the service which uses this data, and b) does not violate the security properties of EDHOC. 



%-------------------------------------------------------------------------- sec
\section{Formalization and results}
\label{sec:formalization}
Next we describe our approach at formalizing the \mEdhoc protocol. We use the
symbolic (Dolev-Yao) model for verification, using \mTamarin for tool support.
%
The next three subsections describe our threat model, briefly present the
\mTamarin tool, and our modeling choices.
%
Finally, we present the properties that we proved in this effort.

\subsection{Threat model}
We verify \mEdhoc in the symbolic Dolev-Yao model: as customary in this style of
modeling, we assume all cryptographic primitives to be ``perfect'', and hence
only allow the attacker to encrypt and decrypt messages when they know the key,
and exclude hash collisions, for example; the attacker is also in control of the
communication channel, and can interact with unbounded sessions of the protocol,
dropping, injecting and modifying messages at their liking.

One important point here is that we allow the attacker to impersonate malicious
endpoints, by revealing their long-term and session key material at any given
point.
%
\mEdhoc should remain secure under these assumptions, as we detail below.

\paragraph{Post-compromise security}
Post-compromise security is a class of properties that relate to the ability of
a protocol to maintain security of previous and future sessions under a
compromise.
%
Cohn-Gordon et al.~\cite{cohn2016post} present a formal characterization of this
class of properties.
%
In this paper we adopt their terminology, hence we refer to them for a detailed
explanation.

Briefly speaking,~\cite{cohn2016post} defines the \emph{classical adversary
  model} as the model that considers only sessions ran by honest parties; other
parties may be dishonest and thus reveal their private information to the
attacker, however the honest parties remain honest.
%
Under the classical model security must be maintained for the parties
considered.
%
\emph{Perfect forward secrecy} (PFS) drops the assumption that the parties
involved in the sessions considered shall remain honest: hence long-term key
material may be leaked after the run of a session, but such session must remain
secure.
%
Further, \emph{key-compromise impersonation} (KCI) takes the perspective of one
of the endpoints of the protocol, say Alice running a session with Bob. A
protocol is secure under KCI if Alice can still establish a secure session with
Bob, even though Alice's keys are compromised at any time, and Bob's key
material is not leaked until the end of the session.
%
Finally, \emph{weak} and \emph{strong post-compromise security} (PCS) consider
only those sessions run by parties that are not under control of the
attacker. These same parties can be under the attacker's control even before and
after the session.
%
A protocol with \emph{weak PCS} must guarantee security under a limited
compromise, where the key material is not leaked, but the attacker has access to
the compormised party to perform cryptographic operations (e.g. through
interfacing with a trusted computing module).
%
A protocol with \emph{strong PCS} gurantees security even if the attacker has
complete access to the state of both parties involved, up until before the
execution of the session under consideration and even after.
%
This is usually achieved by maintaining some state and performing key rotations,
hence the attacker must be able to observe the internal state of the device as
the protocol runs.
%
\emph{We do not check strong PCS} for \mEdhoc, as it is not secure under these
assumptions.

\paragraph{Session independence}
Finally we model \emph{session independence} of \mEdhoc, that is, we allow
leakage of session key material, and additionally check security only of those
sessions for which the session key material has not been revealed. We check this
in conjunction with PCS properties.

\subsection{Tamarin}
We chose \mTamarin to model and verify \mEdhoc in the Symbolic model.
%
\mTamarin is an interactive verification tool based on multi-set rewriting rules
with event annotations, which allows to check LTL temporal formulas on these
models.
%
Multi-set rewrite rules with events take the form:
%
\[ l \ifarrow[e] r \]
%
where $l$ and $r$ are multi-sets of facts, and $e$ is a multi-set of events.
%
Facts are $n$ary predicates over a term algebra, which defines a set of function
symbols $\mathcal F$, variables $\mathcal V$ and names $\mathcal N$. Tamarin
checks equality of thse terms under an equational theory $E$, hence one can
write that
%
\[ dec(enc(x,y),y) =_E x \]
%
to denote that symmetric decryption reverses the encryption operation, and so
forth. All operations on terms are defined under $E$, hence we omit the
subscript from now on as the equational theory is fixed per model.

\paragraph{Semantics and built-ins} On a first approximation, \mTamarin states
$S$, $S'$ are multisets of facts, and a semantic transition $S \semarrow[E] S'$
occurs if there is a rule $l \ifarrow[e] r$ and a substitution $\sigma$ such
that $S \supseteq \sigma(l)$ and $S' = S \setminus \sigma(l) \uplus \sigma(r)$
and $E = \sigma(e)$.

There are a few more details, such as persistent facts that are denoted by a $!$
and are never removed from the state.
%
The sorts fresh (denoted by $\sim$) and public (denoted by $\$$) denote fresh
constants and public values known to the attacker respectively, and are both
sub-sorts of a base sort.
%
Finally, \mTamarin has some built-in predicates ($\mIn,
\mOut$ to represent input and output of messages with the attacker, among
others), rules and equations that represent the attacker's knowledge and
standard equational theories in the symbolic model, plus syntactic sugar, all of
which we introduce as we see necessary.

For example in our model we have a symbol to denote authenticated encryption and
hence \mTamarin produces the rule:
%
\[ !\mKU( k ), !\mKU( m ), !\mKU( ad ), !\mKU( al ) \ifarrow !\mKU( \mAeadEncrypt(k, m, ad, al) ) \]
%
to denote that if the attacker knows a key $k$, a message
$m$, the authenticated data $ad$, and an algorithm
$al$, then they can construct the encryption using these parameters, hence get
to know the message $\mAeadEncrypt(k, m, ad, al)$.
%
There are two built-in persistent facts to denote attacker knowledge,
$!\mKU$ and
$!\mKD$, but we ignore their distinction for now as it is only necessary for
\mTamarin's termination properties.

In our model we introduce a theory for authenticated encryption, plus the
built-in theories of XOR and Diffie-Hellmann.
%
Authenticated encryption, which is encryption with authentication data as
detailed in~\cite{aead}, has the following two equations:
\begin{align*}
  \mAeadDecrypt(k, \mAeadEncrypt(k, m, ad, al), ad, al) = m\\
  \mDecrypt(k, \mAeadEncrypt(k, m, ad, al), al) = m
\end{align*}
With the first rule we allow the protocol to decrypt the message $m$ if the
encryption has matching key $k$, authenticated data $ad$, and uses the same
algorithm $al$.
%
The second rule allows the attacker to decrypt the message $m$ with the key $k$
and without the authenticated data $ad$, and hence skip the check.

The built-in theories for XOR and Diffie-Hellman are a fair bit more complex
than authenticated encryption, hence we refer to the original
papers~\cite{xorTamarin,dhTamarin} for a full reference.
%
Suffices to say that the XOR theory introduces the symbol $\oplus$, for
expressing XOR operations $x \oplus y$, plus the necessary equational theory
including associativity, commutativity, and inverse.
%
The theory for Diffie-Hellman introduces exponentiation $x^y$ and product
$x \cdot y$ as a built-in symbols in the language, plus the necessary equational
theory of associativity, commutativity, distributivity of exponentiation with
product, and inverse.

\subsection{Modeling \mEdhoc}
In this section we detail the modeling choices that we have made for this formal
verification effort.
%
We model the five different modes of \mEdhoc from a single specification
according to the five different combinations of authentication methods:
\mPskPsk, \mSigSig, \mSigStat, \mStatSig and \mStatStat.
%
We use the M4 macro language to derive these different modes from a single
description of the protocol, thus enforcing uniformity in the presentation.
%


\subsection{Properties}

%-------------------------------------------------------------------------- sec
\section{Discussion}
\label{sec:discussion}
%-------------------------------------------------------------------------- sub
There are a few instances where things are ill-specified in the \mSpec{}, which we found as part of this work. We discuss them below. 

\subsection{Security claims}
\label{sec:securityClaims}
The \mSpec{} makes detailed claims about security properties that \mEdhoc{} enjoys. In particular, in Section~8.1 of~\cite{selander-lake-edhoc-01}, the authors claim that ``EDHOC inherits its security properties from the theoretical SIGMA-I''. While it is good practice to reuse well-studied academic components to design a standard, it is important to justify that any changes made still preserve the security properties enjoyed by the component. It is inadvisable for the \mSpec{} to make claims about security properties without any analysis to back this up.   

\subsection{Lack of clear use-cases}
\label{sec:unclearProtocolUse}
Formal verification aims to verify whether well-specified security goals hold for particular security models, defined in terms of attacker capabilities, assumptions on storage and processing by parties etc. Without a clear picture of the intended uses of the protocol, it cannot be determined which properties are the most important ones for constrained IoT applications.

Much like \mDandTls{}, \mEdhoc{} is intended for a variety of use cases,
many of which are not even known today. However, one can still collect a variety of \emph{typical} use cases and user stories. Having a better understanding of the security properties required by these use cases would help define a clearer security model, and thus guide the design of \mEdhoc{}. For example, one might be able to reduce the number of methods to only a few that are needed.

While constructing our model, we came up with simple user stories to identify security properties of interest. Several of these revealed undefined aspects of \mEdhoc{} that were then included in the \mSpec. We present below a couple of examples.

\subsubsection{Credentials stored in a trusted execution environment (TEE)}
Does one need to consider applications of \mEdhoc{} where the device is
deployed in a hostile environment, and might be equipped with a TEE? We received indications from the \mSpec{} authors (personal correspondence) that it was not necessary to consider revealing of the ephemeral secret keys, i.e. not necessary to consider compromised TEEs. 

The rationale is that \mSigma{} cannot protect against such an attack (presumably based on the fact that the \mSigSig{} method is closely based on \mSigmaI{} and that it would be preferable to obtain some kind of homogeneity among the \mEdhoc{} methods when it comes to what security properties they provide). \vnote{Speculation?}
%
That argument is only true, however, if one restricts attention to session key confidentiality of an ongoing session. TEEs do provide value by, for example, allowing weak PCS guarantees. Furthermore, in the \mStat{}-based methods, a static long-term key and an ephemeral one are mixed to obtain the session key material. This design is intentional, especially in the \mOptls{} case, where the protocol is designed to be provably secure in the Canetti-Krawczyk model. \vnote{So?}
%

\subsubsection{Non-repudiation}
An access control solution for a nuclear power-plant may need to log who is passing through a door, whereas it may be undesirable for, say, a coffee machine to log a list of users along with their coffee preferences. Via this simple thought experiment, we realized that the \mSpec{} did not
consider non-repudiation. In response, the authors of the \mSpec{} added a paragraph about which methods provided which types of (non)-repudiation.

\subsubsection{Peer authentication}
Section 3.2 of the \mSpec{} states that parties must be configured
with a policy restricting the set of peers they can run \mEdhoc{} with. However, the initiator is not required to verify that the \mIdcredr{} received in message 2 is the same as the one intended at initialization. The following thought experiment shows why such a policy is insufficient.

Assume that someone has configured all devices in their home to be in the allowed set of devices, but that one of the devices ($A$) has since been compromised. If another device $B$, unaware of the compromise, initiates a connection to a third device $C$, the compromised device $A$ may interfere. $A$ may respond in $C$'s place, blocking the legitimate response from $C$. Since $B$ does not verify $C$'s identity in message 2, and device $A$ is part of the allowed set, $B$ will complete and accept the \mEdhoc{} run with device $A$ instead of the intended $C$. The obvious solution is for the initiator to match \mIdcredr{} to the intended identity provided by the application. \vnote{Has this been pointed out to the authors of the \mSpec?}

%-------------------------------------------------------------------------- sub
\subsection{Session key material}
\label{sec:sessionKeyMaterial}
\mEdhoc{} establishes a session key state as output, from which session keys for \mOscore{} can be derived using \mHkdf{}. As discussed in Section~\ref{sec:threat-model}, there are various possibilities for defining which parameters from the \mEdhoc{} run should be included in the session key state.

While we show bidirectional injective agreement on \mGx{} and \mGy{}
for all methods, initiators cannot obtain injective agreement on \mGiy{} when using the \mStat{} method themselves. Therefore, \mGiy{} cannot be considered part of the key material if mutual injective agreement is considered important. Excluding \mGiy{} would deviate from \mOptls{}, which is the cryptographic core used for the \mStat-based methods. Specifically, the proof of session key confidentiality in the face of long-term key compromise for \mOptls{} cannot be adapted for \mEdhoc{}. Other security properties of \mOptls{} may also fail to be true if
\mGiy{} were excluded. \textbf{Conjecture.}
%

This can be avoided by including a fourth message from responder to initiator carrying a message authentication code using a key derived from session key material including \mGiy{}. This would give the initiator sufficient guarantees about the same \mGiy{} being used by the responder as well. However, the cost is an additional message, and our understanding is that this is unacceptable (personal correspondence with the \mSpec{} authors).
%
In some use cases there could be a such as message as part of
the subsequent \mOscore{} protocol run. It would, however, create an additional dependency between the two protocols, increasing complexity and the risk of insecure implementations.

Yet another option is to include \mGi{}, or a hash of it, in messages one and two. This would, however, increase message sizes, a grave concern for \mEdhoc{}, and would also prevent initiator identity protection. While identity protection is important in general, neither the \mSpec{} nor the requirements document give any rationale why one identity (the intiator's) is deemed more important (and needs protection) than the other (the responder's). So perhaps this method can be used to obtain injective agreement on \mGiy. 

%
%The \mSpec{} states that the initiator's identity is protected against active attacks and the responder's against passive attacks, but that the roles should be assigned to the \mCoap{} peers to protect the most sensitive identity the most (see Section~8~\cite{selander-lake-edhoc-01}).
%
%Roles could hence be swapped, so identity protection is not a strong argument against this option.
%
In this work, we have shown implicit agreement on \mGx{}, \mGy{} for all methods, and also on \mGiy{} and \mGrx{} when these are used, see Section~\ref{sec:formalization}. If this weaker security guarantee is sufficient, both ephemeral and long-term keys can be part of the session key material. However, this also depends on the intended use of \mEdhoc{}.
%

%-------------------------------------------------------------------------- sub
\subsection{Cipher suite negotiation}
\label{sec:ciphersuiteNegotiation}
%
Cipher suite negotiation in \mEdhoc{} spans two or more executions of the protocol. If a run terminates due to the proposed cipher suites being rejected by the responder, the initiator maintains state and initiates a new run, proposing an updated set of cipher suites (see Section~\ref{sec:edhoc}). Our model does not cover this, and we leave it for future work.

Maintaining state between protocol runs implicitly creates a long-lived meta-session covering multiple \mEdhoc{} sessions. Such a meta-session is presumably controlled by the underlying application. However, the \mSpec{} does not specify the time for which the initiator should remember a rejected cipher suite for a given responder.

%
From a security perspective, remembering the rejected cipher suite for the
next \mEdhoc{} run in the same meta-session would be sufficient. If the responder is updated with a new cipher suite before the next such session, this could be taken into account. On the other hand, caching the rejected cipher suite between meta-sessions would reduce the number of round-trips for subsequent runs, should the responder not have been updated. This needs to be clearly specified. 

%-------------------------------------------------------------------------- sec
\section{Conclusions}
\label{sec:conclusions}
We have formally modeled all the five
methods of the \mEdhoc{} key establishment protocol using the \mTamarin{} tool.
%
Using the model, we identified and defined several important security properties
and verified these.
%
We also identified security properties that does not hold for all
methods.
%
Most importantly, injective agreement on \mGiy{} does not hold for
initiators when they use the \mStat{} method, so this property cannot be claimed
for the entire key material in those situations.
%
Further, we identified a situation where initiators may establish an \mOscore{}
security context with a different party than the application using \mEdhoc{}
intended, and proposed a simple mitigation.
%
We discussed possible actions that IETF may take to extract and better define
security properties to make better use of verification techniques.
%

%-------------------------------------------------------------------------- ack
% Should be a run-in heading.  subsubsection works in llncs2e document class
\subsubsection*{Acknowledgments} This work was partially supported by
the Wallenberg AI, Autonomous Systems and Software Program (WASP) funded by
the Knut and Alice Wallenberg Foundation.
%
We are grateful to G\"oran Selander, John Mattsson and Francesca Palombini for
clarifications regarding the specification.
%

%-------------------------------------------------------------------------- bib
\bibliographystyle{plain}
\bibliography{ref}
\end{document}
