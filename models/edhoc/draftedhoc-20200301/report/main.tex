\documentclass[a4paper,11pt]{article}
% BEGIN IACR
\topmargin=-5mm
\evensidemargin=0cm
\oddsidemargin=0cm
\textwidth=16cm
\textheight=22cm
\addtolength{\headheight}{1.6pt}
% END IACR

\usepackage{amsmath,amssymb,amsfonts,amsthm}
\usepackage{stmaryrd}       % Semantic brackets \llbracket and \rrbracket
\usepackage{mathpartir}     % inference rules
\usepackage[scaled=0.86]{helvet}    % Less huge \textsf{functionName}
\usepackage{enumitem}       % compacts lists and stuff

\theoremstyle{plain}
\newtheorem{theorem}{Theorem}%{Theorem}
\newtheorem{lemma}{Lemma}
\newtheorem{proposition}{Proposition}
\theoremstyle{plain}
\newtheorem{definition}{Definition}

\usepackage{graphicx}
\usepackage{xcolor}
\usepackage{tikz}
\usepackage{environ}  % For tikz fig spanning two columns
\usetikzlibrary{positioning}
\usepackage{url}

\def\BibTeX{{\rm B\kern-.05em{\sc i\kern-.025em b}\kern-.08em
    T\kern-.1667em\lower.7ex\hbox{E}\kern-.125emX}}

\usepackage{algorithm}
\usepackage{algpseudocode}         % https://en.wikibooks.org/wiki/LaTeX/Algorithms#Typesetting_using_the_algorithmic_package

\usepackage{draftwatermark}
\SetWatermarkText{DRAFT}
\SetWatermarkScale{2}
\SetWatermarkColor[gray]{.95}

\usepackage[numbers, sort]{natbib}
\usepackage[T1]{fontenc} % E.g., for small caps in section headings
\usepackage{xspace}
\usepackage[lambda,advantage,operators,adversary,landau,probability,notions,logic,
ff,mm,primitives,events,complexity,asymptotics,sets,keys]{cryptocode}

% http://tug.ctan.org/tex-archkeIVe/macros/latex/contrib/todonotes/todonotes.pdf
\usepackage{todonotes}
\newcommand{\knote}[1]{\todo[inline,
                             color=orange,
                             size=\footnotesize
                         ]{\textsf{\textbf{Karl:} #1}}}
%\newcommand{\knote}[1]{}

% -- Fonts and styles for names
\newcommand{\mItemStyle}[1]{\ensuremath{#1}}
\newcommand{\mSetStyle}[1]{\ensuremath{\mathrm{\mathbf{#1}}}}
\newcommand{\mFunStyle}[1]{\text{\textup{\textsf{#1}}}}
\newcommand{\mConstStyle}[1]{\text{\textup{\textsf{#1}}}}
\newcommand{\mVarStyle}[1]{\mathit{#1}}
\newcommand{\mFactStyle}[1]{\text{\textsf{#1}}}
\newcommand{\mRuleStyle}[1]{\ensuremath{\mathbf{#1}}}
\newcommand{\mDomainStyle}[1]{\ensuremath{\mathcal{#1}}}



% Solution for letting a tikz figure to span two columns by Ulrike Fischer
% https://tex.stackexchange.com/questions/6388/how-to-scale-a-tikzpicture-to-textwidth
\makeatletter
\newsavebox{\measure@tikzpicture}
\NewEnviron{scaletikzpicturetowidth}[1]{%
  \def\tikz@width{#1}%
  \def\tikzscale{1}\begin{lrbox}{\measure@tikzpicture}%
  \BODY
  \end{lrbox}%
  \pgfmathparse{#1/\wd\measure@tikzpicture}%
  \edef\tikzscale{\pgfmathresult}%
  \BODY
}
\makeatother
%
\begin{document}
\title{EDHOC - ProVerif}
\author{{\color{red} DRAFT: \today}}
%
\maketitle
%

\begin{abstract}
Status of Proverif model of EDHOC.
\end{abstract}
%

%-------------------------------------------------------------------------- sec
\section{Methods}
EDHOC is a framework that can run five separate cryptographic cores, called
\textbf{methods}.
%
One method is a Diffie-Hellman key exchange authenticated by symmetric key
authentication and the other are based on asymmetric key authentication and
various Diffie-Hellman based mechanisms.
%
The asymmetric key methods allow the initiator and responder to use
regular signatures for authentication or static key/hash based signatures.
%
Any combination of these asymmetric methods are possible.
%
The symmetric method is not mixed with any of the other.
%

We have modeled all five methods.
%

%-------------------------------------------------------------------------- sec
\section{Properties}
%-------------------------------------------------------------------------- sub
\subsection{Secrecy}
%-------------------------------------------------------------------------- sub
\subsubsection{Sessions}
%
We use $U, V, g^{xy}$ as \textbf{session identifier}.
%
Because $x$ and $y$ are large and drawn from a uniform distribution at every
protocol run, this identifier is unique with high probability.
%
In our model, $x$ and $y$ are represented by new names and therefore guaranteed
to be unique.
%
\knote{We may want to include $g^{Iy}$ and $g^{Rx}$ also when static keys are
    involved, in case the adversary manages to make some party accept a correct
    $g^{x,y}$ with a wrong $g^{Iy}$ or $g^{Rx}$, or vice versa.}
%


We are interested in \textbf{session key independence}, but
the way we have defined session key secrecy, makes it unclear what that means.
%
See session key below.
%

%-------------------------------------------------------------------------- sub
\subsubsection{Session key material}
%
By \textbf{session key material}, or session keys for short,  we mean the
keying material derived by an initiator or responder during a run of the
protocol in presence of an active adversary.
%

As session key material for the PSK method we consider $g^{xy}$.
%
The reason is that to show key agreement using correspondence properties, we
cannot include \mConstStyle{TH\_4}, which covers the hash of the third message.
%
So, when the initiator completes, there is not yet a corresponding event in
the responder trace which includes \mConstStyle{TH\_4}.
%
Because binding in \mConstStyle{TH\_4} does not add any further secrecy, there
is no harm in focusing on $g^{xy}$.
%

The same argument goes for the method where both parts authenticate via
signatures, but not when one or both parties use static DH key authentication.
%
When the Initiator is using a static DH key, the session key material is
$g^{xy}$, $g^{Iy}$ and $g^{Rx}$.
%
\knote{This is because EDHOC copied it from OPTLS. However, OPTLS was designed
    for the CK-model and differentiates between compromise and session state
    reveal queries. EDHOC is not as structurally designed and don't care about
separating the two and it may be sufficient to use only $g^{xy}$ in that case.
However: I suspect the authentication in OPTLS comes from the signatures
Krawczyk designed for HMQV, and in that case the signatures \emph{are} the key.
I need to investigate this.
}
%
%-------------------------------------------------------------------------- sub
\subsubsection{Property: secrecy}
%
In parallel with running the protocol, we leak the long-term keys for Initator
and Responder.
%
This means there will be runs where their long-term keys are leaked and there
will be corresponding runs that are exactly the same, but where the long-term
keys are note leaked.
%
When Initiator and when Responder completes, we attempt to read the session
key from the attacker.
%
We say that the EDHOC provides session key secrecy if:
$$
\forall U, V, k.
    \mConstStyle{attacker}(k) \rightarrow
        (\mConstStyle{LTKReveal}(U) \lor \mConstStyle{LTKReveal}(V))
$$
Note, this does not provide session key independence, like the secrecy
definition leaking the session key at the end of the run does.
%
The reason for not using that definition is that we run in to termination
problems with that.
%

%-------------------------------------------------------------------------- sub
\subsubsection{Property: session key independence}
%
If adversary knows a session key that should not help him attacking other
session keys.
%
This can be shown using correspondence properties by leaking session keys and
verifying that this does not help the adversary attack secrecy of other session
keys.
%
As noted, we do not do this because of termination problems.
%

Because we do not verify secrecy of session keys when we \emph{know} the
adversary has access to other session keys, we cannot claim any guarantees for
session key independence.
%

%-------------------------------------------------------------------------- sub
\subsection{Authentication}
%-------------------------------------------------------------------------- sub
\subsubsection{Session key authentication}
%
When both parties use signatures, there is no $g^{Iy}$ or $g^{Rx}$, but instead
the key and entity authentication is based on that each party explicitly signs
the data they are supposed to agree on.
%
So in that case the session key material established consists of $g^{xy}$
only.
% 
This means that the initiator will, when receiving message 2, get a signature
covering $g^{xy}$ and $V$, and this is the main point: the Initiator can
verify this signature.
%
So when the Initiator completes, it knows:
%
\begin{itemize}
    \item that it communicated with party $V$,
    \item that $V$ has access to the session key material $g^{xy}$.
\end{itemize}
%
That is, the Initiator explicitly authenticated $V$ and the session key
material.
%
The corresponding holds for the Responder when it completes after receiving
message 3.
%
Therefore, when each party has completed their respective protocol
runs, they know that they agree on each others' identities and on which session
key material has been established.
%
 
When the Initiator is using a static DH key, the session key material is
$g^{xy}$, $g^{Iy}$ and $g^{Rx}$.
%
When sending message 2, the Responder does not know who the Initiator is, so
he cannot perform a computation that depends on the public key $g^{I}$ of the
Initiator and include that in message 2.
%
Specifically, when sending message 2, the Responder does not know $g^{Iy}$, so
he cannot inform the Initiator which value he will use for that part of the
session key and he cannot commit to it.
%
When the Initiator sends message 3 he performs a computation based on $g^I$
(i.e., resulting in $g^{Iy}$), sends this to the Responder and then completes.
%
So when the Initiator completes the protocol run, he has not received any
confirmation from the Responder which session key material they should use,
i.e., they have not agreed on it and hence the Responder has not authenticated
the key material to the Initiator.
%
But: The Initiator knows that \emph{if} his message 3 reaches the Responder
correctly, \emph{then} only if the Responder is $V$, will the Responder have
access to the session key material $g^{xy}, g^{Iy}, g^{Rx}$.
%
Without confirmation from $V$ in the Responder role, the Initiator cannot know
that they actually have agreed on the session key material when he completes
the run.
%
 
This situation does not occur in the OPTLS paper, because Krawczyk is not
taking client authentication into account.
%
Without client authentication in EDHOC, the corresponding session key material
would be only $g^{xy}$, $g^{Rx}$ and both the client/Initiator and
server/Responder have a access to that when processing message 2.
%
 
So, on the top of my head there are three options when the Initiator uses a
static DH-key (unless I missed something above):
\begin{enumerate}
    \item Accept implicit authentication,
    \item Sacrifice Initiator identity protection and include
        \mConstStyle{ID\_CRED\_I} in message 1, and, e.g., include
        \mConstStyle{ID\_CRED\_I} and/or \mConstStyle{CRED\_I}
        in \mConstStyle{TH\_2} or in \mConstStyle{AAD\_2},
    \item Include a fourth confirmation message from Responder to Initiator
        proving to the Initiator that the Responder knows $g^{Iy}$ (a MAC using
        the session key is enough).
\end{enumerate}
% 
\knote{John find 1 acceptable and would also like to use only $g^{xy}$ and
    session key for the STATIC methods. I would prefer not to use only $g^{xy}$,
    because Krawzcyk's proof for OPTLS is based on $g^{Rx}$ and uses $g^{xy}$
    only to get PFS. Note that responder can only compute $g^{Rx}$ if he has
    access to his private key, so the link to $g^{Rx}$ is stronger. However,
    indirectly it affects also the $g^{xy}$ via the \mConstStyle{MAC\_3}, so it
    may be OK still. But, using $g^{xy}$ to protect against session state
    reveal queries in addition to compromise queries revealing $g^{Rx}$ is
    useful with TPMs (c.f., the signature operation using the private key is
    done in the TPM for the SIG-SIG method).
}
There does not seem to be a similar issue with $g^{Rx}$.
%
When receiving message 2, the Initiator verifies \mConstStyle{MAC\_2}, which is
based on a key derived from $g^{Rx}$.
%
The \mConstStyle{MAC\_2} also covers \mConstStyle{CRED\_R} and
\mConstStyle{ID\_CRED\_R}, so the Initiator knows after verifying
\mConstStyle{MAC\_2} that $g^{Rx}$ is known to and coming from $V$.
%
An attacker would have to compute the discrete log of $g^{x}$ from the first
message to be able to forge that MAC.
% 

%-------------------------------------------------------------------------- sub
\subsubsection{Property: Session key authentication}
%
TBD
%

For the cases where the Initiator uses a static DH key, we prove implicit
authentication of Responder to initiator:
$$
\forall l. \mConstStyle{accept(l)} ==>
    \forall l'. \mConstStyle{sameKey}(l, l') ==> l'.pid = l.id
$$
which is an adaption from Delpech de Saint Guilheim et al.
%
The term $l.id$ is the identity of $l$, and $l.pid$ is the identity of $l$'s
peer.
%
The Initiator can still be explicitly authenticated to the Responder using
normal running/commit technique.
%

%-------------------------------------------------------------------------- sub
\subsubsection{Authentication/Agreement of other parameters}
%
In the same fashion we prove agreement on the session key material we can show
agreement of other parameters.
%
In particular we show that $U$ and $V$ agree on:
\begin{itemize}
    \item \mConstStyle{METHOD}
    \item \mConstStyle{C\_I, C\_R}
    \item cipher suite
    \item $U$ and $V$ (entity authentication)
    \item $\ldots$
\end{itemize}

Note that we only model one run of the protocol, so agreement on ciphersuite
does not include re-negotiation by a new run of the protocol.
%
This means that we only show agreement when the Responder either accepts the
proposal from Initiator, or terminates its execution.
%






%-------------------------------------------------------------------------- sub
\subsection{Identity privacy}
%
Identity privacy does not apply to the PSK method.
%
For asymmetric methods, identity privacy means hiding \mConstStyle{ID\_CRED} and
\mConstStyle{ID\_CRED\_X}.

%-------------------------------------------------------------------------- sub
\subsubsection{Property identity privacy}
TBD

\subsection{Reflection (attacks)}
%-------------------------------------------------------------------------- sub
Section 8.6 of the specification discusses a worry for reflection attacks.
%
We could have modeled and verified/falsified whether it is possible for
$U$ to be "tricked" into communicating with $U$.
%
This caused termination issues though, so we do not do this at the moment.
%

It also questionable whether it is a feature or bug that $U$ can communicate
with $U$.
%
The intention of the protocol in the specification has to be the final judge.
%

%-------------------------------------------------------------------------- sec
\section{Primitives}
%-------------------------------------------------------------------------- sub
\section{Encryption}
%
We model authenticated encryption with additional data (AEAD) in the standard
way.
%
We consider an AEAD transform as a single primitive providing both integrity
protection and encryption.
%

Message 2 is however encrypted using an XOR-pad that is generated using the KDF.
%
Even though the protocol intends for message to to enjoy integrity protection by
means of the signature, we cannot consider the signature in combination with
XOR as a single primitive.
%
Nonetheless, we model this XOR in the same way as we model AEAD, which is not very truthful
to the specification, since XOR does not provide integrity protection.
%
There are difficulties modeling XOR encryption in Proverif.
%


%\bibliographystyle{plain}
%\bibliography{ref}

%\printbibliography

\end{document}
