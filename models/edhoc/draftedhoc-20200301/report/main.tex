\pdfoutput=1
\documentclass[runningheads]{llncs}
\usepackage{amsmath,amssymb,amsfonts}
%\usepackage{stmaryrd}       % Semantic brackets \llbracket and \rrbracket
%\usepackage{mathpartir}     % inference rules
\usepackage[scaled=0.8]{helvet}    % Less huge \textsf{functionName}
\usepackage{enumitem}       % compacts lists and stuff
\usepackage[subtle]{savetrees}
\usepackage{soul}           % \hl for highlighting text; \st for strike-through

\usepackage{graphicx}
\usepackage{xcolor}
\usepackage{tikz}
\usepackage{environ}  % For tikz fig spanning two columns
\usetikzlibrary{positioning}
% Used for displaying a sample figure. If possible, figure files should
% be included in EPS format.
%
% If you use the hyperref package, please uncomment the following line
% to display URLs in blue roman font according to Springer's eBook style:
\usepackage{hyperref}
%\renewcommand\UrlFont{\color{blue}\rmfamily}
% not the way to do it for arXiv though...
\usepackage{url}
\hypersetup{
    final, % Uncomment to remove all links (useful for printing in black and white)
    colorlinks=true, breaklinks=true, bookmarks=false,bookmarksnumbered,
    urlcolor=blue, linkcolor=black, citecolor=green, % Link colors
%   pdftitle={title of paper}, % PDF title
%   pdfauthor={auth1, auth2}, % PDF Author
%   pdfsubject={}, % PDF Subject
    pdfkeywords={}, % PDF Keywords
    pdfcreator={PdfLaTeX}, % PDF Creator
%   pdfproducer={Pandoc, XeLaTeX with hyperref} % PDF producer
}

\def\BibTeX{{\rm B\kern-.05em{\sc i\kern-.025em b}\kern-.08em
    T\kern-.1667em\lower.7ex\hbox{E}\kern-.125emX}}

%\usepackage{algorithm}
%\usepackage{algpseudocode}         % https://en.wikibooks.org/wiki/LaTeX/Algorithms#Typesetting_using_the_algorithmic_package

\usepackage{draftwatermark}
\SetWatermarkText{\textbf{DRAFT}}
\SetWatermarkScale{2}
\SetWatermarkColor[gray]{.92}


\usepackage[numbers, sort]{natbib}
\usepackage[T1]{fontenc} % E.g., for small caps in section headings
\usepackage{xspace}
\usepackage[lambda,advantage,operators,adversary,landau,probability,notions,logic,
ff,mm,primitives,events,complexity,asymptotics,sets,keys]{cryptocode}

% http://tug.ctan.org/tex-archkeIVe/macros/latex/contrib/todonotes/todonotes.pdf
\usepackage{todonotes}
\newcommand{\knote}[1]{\todo[inline,
                             color=orange,
                             size=\footnotesize
                         ]{\textsf{\textbf{Karl:} #1}}}
%\newcommand{\knote}[1]{}

% -- Fonts and styles for names
\newcommand{\mItemStyle}[1]{\ensuremath{#1}}
\newcommand{\mSetStyle}[1]{\ensuremath{\mathrm{\mathbf{#1}}}}
\newcommand{\mFunStyle}[1]{\text{\textup{\textsf{#1}}}}
\newcommand{\mConstStyle}[1]{\text{\textup{\textsf{#1}}}}
\newcommand{\mVarStyle}[1]{\mathit{#1}}
\newcommand{\mFactStyle}[1]{\text{\textsf{#1}}}
\newcommand{\mRuleStyle}[1]{\ensuremath{\mathbf{#1}}}
\newcommand{\mDomainStyle}[1]{\ensuremath{\mathcal{#1}}}



% Solution for letting a tikz figure to span two columns by Ulrike Fischer
% https://tex.stackexchange.com/questions/6388/how-to-scale-a-tikzpicture-to-textwidth
\makeatletter
\newsavebox{\measure@tikzpicture}
\NewEnviron{scaletikzpicturetowidth}[1]{%
  \def\tikz@width{#1}%
  \def\tikzscale{1}\begin{lrbox}{\measure@tikzpicture}%
  \BODY
  \end{lrbox}%
  \pgfmathparse{#1/\wd\measure@tikzpicture}%
  \edef\tikzscale{\pgfmathresult}%
  \BODY
}
\makeatother
%
\begin{document}
\title{EDHOC - Tamarin modeling paper}
\author{First Author\inst{1}\orcidID{0000-1111-2222-3333} \and
Second Author\inst{2,3}\orcidID{1111-2222-3333-4444} \and
Third Author\inst{3}\orcidID{2222--3333-4444-5555}
{\color{red} DRAFT: \today}
}
%
\authorrunning{F. Author et al.}
% First names are abbreviated in the running head.
% If there are more than two authors, 'et al.' is used.
%
\institute{
    KTH Royal Institute of Technology, SE-100 44, Stockholm, Sweden \and
    Ericsson Security Research, SE-164 83, Stockholm, Sweden,
    \email{karl.norrman@ericsson.com} \and
    ITU Copenhagen $\ldots$ \and
    Ericsson Intell. Autonomous sys$\ldots$
}
%
\maketitle
%

\begin{abstract}
\hl{250 words}
IETF is standardizing a key establishment protocol, named \mEdhoc, for
constrained IoT devices.
%
In contrast to more powerful IoT devices, such as video cameras and cars,
which receive most attention from media, constrained devices often have severe
restrictions on energy consumption.
%
Additionally, they often use specialized wireless communication links with tough
constraints on message sizes, which may vary from message to message.
%
\mEdhoc was first formally analyzed by Bruni
et.al.~\cite{DBLP:conf/secsr/BruniJPS18}.
%
Since then, IETF has significantly extended the protocol, which is now a
framework with a number of cryptographic cores, called methods, where the
initial version of \mEdhoc is now just one out of five possibilities.
%
In this paper we formally analyze all methods of \mEdhoc in a symbolic model,
using the Tamarin verification tool.
%
We show that the different methods provide sensible, but also rather
heterogeneous security properties, and discuss consequences of this.
%
\end{abstract}
%
{\color{red}
    Time plan (we keep weekly Thursday meetings for sync):
\begin{itemize}
    \item \textbf{Target conference:} Security Standardization Research (SSR),
                               \hl{Submit DL: 24/8},
                               Notification: 1/10,
                               Camera ready: 12/10,
                               20 pages including bib, LCNS template
    \item \textbf{18/6} Outline and work division proposed by Karl; Karl and
        Vaishnavi start writing our parts
    \item \textbf{25/6} Alessandro can start writing his parts
    \item \textbf{\hl{9/7}} \mArxiv version ready for upload (minus formatting issues)
    \item \textbf{13/7} \mArxiv formatted version sent out for review (at least to
        Mads Dam, IETF team and internal Eri review).
    \item \textbf{17/7} Possible comments from review fixed, and upload to
        \mArxiv. This DL is needed. Otherwise people at the IETF meeting will
        not be able to comment from the IETF meeting.
    \item \textbf{---} Vacation(...)
    \item \textbf{20/8} Reformatted to SSR 2020 LNCS; Possibly added properties
        to model and paper if something useful came up during vacation
    \item \textbf{24/8} SSR submission deadline
\end{itemize}
}

%-------------------------------------------------------------------------- sec
{\color{blue}
\section*{Outline}
This section is only for the outline and will be removed in the final
    paper. Don't worry about the language or style. Only the content matters.
\vspace{10pt}
\knote{Please do use and define macros in the file \texttt{macros.tex}!}
\knote{Notes like this one are defined for the macros
    \texttt{\textbackslash knote},
    \texttt{\textbackslash vnote} and
    \texttt{\textbackslash anote}}

\begin{itemize}
    \item \textbf{Section: Situation and setting (\hl{Karl})}
        \begin{itemize}
            \item Elaborate on the story line above
            \item Explain the exploratory and iterative nature of industrial
                  standards and why this has resulted in the new \mStat methods.
            \item Contrast with \mTls. Motivate why \mEdhoc is needed in
                addition to \mTls. Little analysis of \mEdhoc so far.
                Motivate that most security analysis and media attention is on
                larger IoT devices. Give examples and cite use cases whit
                constrained devices. Paint the (true) picture that this will
                become a serious security risk in the future if not a secure
                \mEdhoc is created.
            \item Subsection: Contributions: model, verified properties,
                guidance on how various methods can be used (for std and
                implementations)
            \item Subsection: Related work: \mOptls, \mNoise,
                Bruni~\cite{DBLP:conf/secsr/BruniJPS18}, Cremers' et. al. \mTls
                analysis with \mTamarin.
        \end{itemize}
    \item \textbf{Section: \mEdhoc description (\hl{Vaishnavi?})}
        \begin{itemize}
            \item Describe evolved protocol from last \mEdhoc verification,
            \item Describe the overarching goal of \mEdhoc of security context
                establishment for \mEdhoc.
            \item Describe that \mEdhoc is a framework based on three different
                crypto-cores (\mPsk-based DH, \mSigma and \mOptls/\mNoise).
                Some inital thoughts on \mNoise here~\ref{sec:mail-notes-noise}.
                Describe how one party may use a \mSigma-like authentication
                whereas the other party may use \mOptls. The session key is
                the combination of the corresponding key material.
                Describe that this essentially makes
                \mEdhoc a framework where it is possible to run one out of 5
                different protocols.
            \item Describe \mSigSig method and relate it to original \mEdhoc and
                \mSigma. This can be brief. Identify and focus on potential
                differences.
            \item Describe \mPskPsk method.
            \item Main focus of this section. Describe \mStat based methods, which
                  have very different properties compared to original \mEdhoc,
                  which only had the single \mSigma method.
                  Draw a figure (use tikz for easiest handling), showing how the
                  protocols look like (\mStatStat, \mStatSig, \mSigStat are
                  sufficient, the \mPsk and \mSigSig methods are simpler/old and
                  less interesting).
                  Explain the difference in signatures: "Normal" vs. challenge
                  response signatures of \mStat-based method.
                  Explain how \mStatStat is basically running two interleaved
                  \mOptls protocols runs. Compare how \mStat based methods relate
                  to \mOptls and to relevant instance from \mNoise framework.
                  Make mapping as precise as possible, down to key derivation
                  inputs. If the mapping of protocol elements and inputs to
                  key derivations and other processing is sufficiently close,
                  then point to security proof for Noise and cite it. Otherwise
                  point out the differences.
             \item Describe additional components of \mEdhoc we've modeled,
                  e.g., \mCi and \mCr etc.
                  Explain relevance of \mCi and \mCr to \mOscore.
             \item Describe algorithm negotiation.
             \item Describe negotiation of method.
             \item Describe transmission of aux data and expected security
                 properties.
        \end{itemize}
    \item \textbf{Section: \mEdhoc security model and formalization (\hl{Alessandro?})}
        \begin{itemize}
            \item Subsection: threat model. We assume secure binding between
                credentials and the identity. We assume adversary cannot
                register another key for a name for an uncorrupted party. We
                assume adversary has Dolev-Yao style access to communication
                between agents. We assume adversary can compromise session keys,
                but is then not allowed to "test" that session key (in
                computational terminology). We assume adversary can compromise
                parties, but is not allowed to "test" them or their partners if
                he has done so. Two peers are partnered if they agree on: Their
                identities, and the session key material (except \mGiy in the
                relevant cases of \mStat). That is the session identifier and it
                is unique per run, which the injective agreement lemmas show.
                Define what is the key material for each method
                (see the \mTamarin model).
            \item Subsection: Short description of \mTamarin. It is known by
                now, but a short overview does not hurt.
            \item Subsection: Explain details of model, which properties we have
                modeled, how, which trade-offs were made and why.
                Explain how we modeled inj-agree (a sentence or two). Explain
                why it fails when $I$ uses \mStat-method and the implicit auth
                lemma we showed instead
                (cite~\cite{DBLP:journals/iacr/GuilhemFW19} here where I got the
                basic idea. It is then extended with the reveal queries in our
                model; obviously: if you know about other sources for this
                please cite them too). Include code/lemmas as you see fit.
            \item Subsection: Clear bullet-list of what we have shown, including
                some statics of running time code size perhaps.
                Scan draft-selander and check what props are claimed there; see
                what we have covered and what we haven't.
                Already shown:
                \begin{itemize}
                    \item Mutual injective agreement on session key, method and identities for
                        $R \rightarrow I$ and $I \rightarrow R$ for all methods
                        except when $I$ uses STAT. When $I$ uses STAT, $R \rightarrow I$ holds,
                        but not $I \rightarrow R$.
                    \item Injective implicit agreement on session key.
                        Only really needed for $I \rightarrow R$, because in all other cases we have
                        the stronger injective agreement property. But show it now for all
                        methods to having a common (and reasonable) level of security that
                        holds no matter which method user chooses.
                    \item PFS on session key for all methods.
                    \item Key secrecy (follows from PFS).
                    \item Session key independence (follows from how we modeled PFS).
                    \item Entity authentication - we show this in the same lemmas as we show
                          inj-agree on session key.  We don't have this property when $I$ uses
                          STAT method because $I$ then gets no confirmation of
                          Pre-specified peer model.
                    \item Key confirmation (except for when $I$ uses \mStat
                        method, then $I$ gets no key confirmation on \mGiy)
                \end{itemize}
            \item See~\ref{sec:mail-notes-encr} for how encryption is modeled.
            \item Caveat: we have not modeled running all methods in parallel,
                so we don't know if an adversary can somehow trick $I$ into
                running one method and $R$ another, and by that causing some
                attack.
        \end{itemize}
    \item \textbf{Section: Discussion (\hl{Karl})}
        \begin{itemize}
            \item Discuss that the security properties varies between methods
                and that this may confuse implementers and user of libraries.
            \item Discuss the overlayed \mOptls construction and how an
                additional \mOscore message $R \rightarrow I$ would give key
                confirmation and hence explicit inj-agree also for $I$.
            \item Discuss Identities and use-cases with the neighbours printer
                (now included in draft, but only after we mentioned it),
                See~\ref{sec:mail-notes-identity}.
            \item Discuss Transcipt hashes that lag behind one message and that
                it is supposed to "cover as much as possible", but sometimes
                does not (check tamarin model also for inconsistencies between
                methods). Unclear design except "as much as possible".
            \item Discuss unclearity regarding security model (trusted execution
                environment or not?). \mOptls has clear separation and therefore
                use both $g^{xy}$ and $g^{iy}$ etc. \mEdhoc is not as cleanly
                designed and according to IETF guys they don't care about
                trusted execution environments (see mail in txt-file).
                Would \mEdhoc be OK with skipping $g^{iy}$ from the session key
                material and only use $g^{xy}$ in that model? Probably yes, and
                then we can get inj-agree on the session key material, but then
                extending \mEdhoc implementations with a trusted execution
                environment for LTK operations is useless.
            \item Discuss practice of listing claimed properties (likely copied
                from some academic paper) without justification in IETF drafts.
            \item Discuss algorithm negotiation.
            \item Discuss session key authentication and the problem of \mGiy,
                see~\ref{sec:mail-notes-session-key-auth}
                and~\ref{sec:mail-notes-session-key-imp-auth}
            \item Discuss what should be used as session key and what the
                consequences of the choices are,
                see~\ref{sec:mail-notes-secrecy-sessions} and
                \ref{sec:mail-notes-session-key-mtrl}.
            \item Discuss non-repudiation that was not included until we pointed
                it out.
        \end{itemize}
\end{itemize}
}

%-------------------------------------------------------------------------- sec
\section{Introduction}
\label{sec:introduction}
%-------------------------------------------------------------------------- sub
\subsection{Motivation}
\label{sec:motivation}
IoT security threats involving cars, web-cameras and other resourceful devices
receive most attention from media and academia.
%
These devices are computationally strong with no severe bandwidth or energy
consumption restrictions.
%
Securing the communication between such devices can readily be done using
\mDandTls.
%
Constrained devices, on the other hand, where such restrictions are common,
have received much less attention.
%
These devices may be simple sensors with the only task of relaying
measurements of their physical environment to a server every hour, and doing so
autonomously for a decade without maintenance.
%
Constrained devices therefore often have limitations on energy consumption.
%
To keep energy consumption down, highly specialized radio links with small
and heterogeneous frame-sizes are sometimes used.
%
In some cases, \mDandTls messages are too large to fit into the radio frames.
%
This is one of the reasons IETF standardized the \mOscore protocol to secure
communications between constrained devices, as a complement for when \mDandTls
is too heavy weight~\cite{rfc8613}.
%

The \mOscore protocol requires a pre-established security context.
%
The IETF Lightweight Authenticated Key Exchange (LAKE) working group
currently develops requirements and a key exchange protocol capable of
establishing \mOscore security contexts.
%
The key establishment protocol is named \mEdhoc~\cite{selander-lake-edhoc-01}.
%
Naturally, \mEdhoc must work under the same constrained requirements as
\mOscore itself.
%

While neither the requirements nor the use-cases for \mEdhoc are firmly set,
the overarching design goal of \mEdhoc is to establish an \mOscore security
context with decent security while keeping the messages small.
%
The LAKE working group is currently discussing whether a compressed version of
\mTls, named \mCtls~\cite{ietf-tls-ctls-00}, would solve the same use-cases as
\mOscore and \mEdhoc combined.
%

The \mEdhoc protocol has evolved significantly over time to cater for smaller
messages and more use-cases.
%
The first incarnation of \mEdhoc appeared in March 2016.
%
It contained two different cryptographic cores, one based on a
pre-shared key Diffie-Hellman and a second following a draft of the
then emerging \mbox{\mTls v1.3} standard~\cite{ietf-tls-tls13-11}, using
challenge-response signatures.
%
The latter was then replaced by \mSigma, and this version, from May 2018, was
formally analyzed by Bruni et. al.~\cite{DBLP:conf/secsr/BruniJPS18}.
%
The protocol has now further evolved and variants using challenge-response
signatures have now been added again.
%
On top of this, mixed variants where one party uses a challenge-response
signature and the other a regular signature have also been added.
%
Consequently, there are now five cryptographic cores in total, and it is prudent
to formally analyze them all to ensure a higher level of security assurance for
\mEdhoc.
%
This is especially important, since the standard itself lacks description of the
intended security model and overall security goals.
%
Filling this gap and deriving circumstances under which \mEdhoc can be
securely used to achieve certain goals is an important part of this paper.
%

%-------------------------------------------------------------------------- sub
\subsection{Contributions}
\label{sec:contributions}
Our main contributions are the following.
\begin{itemize}
    \item We provide formalization of \mEdhoc's all five cryptographic cores
        using \mTamarin~\cite{DBLP:conf/cav/MeierSCB13}.
    \item We give an explicit security model for the protocol and have verified
        essential security properties, such as session key and entity
        authentication, as well as Perfect Forward Secrecy (PFS), within that
        model.
    \item We provide consequence analysis of the verified properties to
        establish which types of use-cases \mEdhoc may be suitable for and
        to give recommendations for the future standards development.
    \item Our discussions with IETF LAKE working group members have already
        lead to improvements and clarifications of the standard based on
        observations we made during the construction of our formal model.
\end{itemize}

%-------------------------------------------------------------------------- sub
\subsection{Related work}
\label{sec:relatedWork}
The work closest to ours is Bruni et. al.~\cite{DBLP:conf/secsr/BruniJPS18},
which used \mProverif~\cite{DBLP:conf/csfw/Blanchet01} to analyze an earlier,
two cryptographic core, version of \mEdhoc.
%
We consider our work to be a sort of follow-up to that, doing a similar kind of
analysis of the most recent, and more elaborate, version of \mEdhoc using five
cores.
%
We also verify slightly different properties.
%
For instance, we include session key independence in our properties.
%
The \mTamarin tool has been used to verify many other protocols, perhaps closest
to our work is Cremers et.al's analysis of
\mTls~\cite{DBLP:conf/ccs/CremersHHSM17}.
%
Some of the cryptographic cores themselves have been analyzed in the
computational model, e.g., \mSigma by Canetti and
Krawczyk~\cite{DBLP:conf/crypto/CanettiK02} and \mOptls by Krawczyk and
Wee~\cite{DBLP:conf/eurosp/KrawczykW16}.
%

%-------------------------------------------------------------------------- sub
\subsection{Structure}
\label{sec:structure}
\knote{Only include this if we have space to spare.}


%-------------------------------------------------------------------------- ack
% Should be a run-in heading.  subsubsection works in llncs2e document class
\subsubsection*{Acknowledgments} This work was partially supported by
the Wallenberg AI, Autonomous Systems and Software Program (WASP) funded by
the Knut and Alice Wallenberg Foundation.
%

%-------------------------------------------------------------------------- bib
\bibliographystyle{plain}
\bibliography{ref}

%\printbibliography

\appendix
%-------------------------------------------------------------------------- sec
\section*{Appendix}

%-------------------------------------------------------------------------- sec
\section{Methods}
\knote{The text that follows was written during the model construction. The
    intention is to keep it here fore reference until we have lifted up what is
    needed to the real report/paper and then delete this annex.
}
%-------------------------------------------------------------------------- sec
\section{Properties}
%-------------------------------------------------------------------------- sub
\subsection{Secrecy}
%-------------------------------------------------------------------------- sub
\subsubsection{Sessions}
\label{sec:mail-notes-secrecy-sessions}
%
We use $U, V, g^{xy}$ as \textbf{session identifier}.
%
Because $x$ and $y$ are large and drawn from a uniform distribution at every
protocol run, this identifier is unique with high probability.
%
In our model, $x$ and $y$ are represented by new names and therefore guaranteed
to be unique.
%
\knote{We may want to include $g^{Iy}$ and $g^{Rx}$ also when static keys are
    involved, in case the adversary manages to make some party accept a correct
    $g^{x,y}$ with a wrong $g^{Iy}$ or $g^{Rx}$, or vice versa.}
%


We are interested in \textbf{session key independence}, but
the way we have defined session key secrecy, makes it unclear what that means.
%
See session key below.
%

%-------------------------------------------------------------------------- sub
\subsubsection{Session key material}
\label{sec:mail-notes-session-key-mtrl}
%
By \textbf{session key material}, or session keys for short,  we mean the
keying material derived by an initiator or responder during a run of the
protocol in presence of an active adversary.
%

As session key material for the PSK method we consider $g^{xy}$.
%
The reason is that to show key agreement using correspondence properties, we
cannot include \mConstStyle{TH\_4}, which covers the hash of the third message.
%
So, when the initiator completes, there is not yet a corresponding event in
the responder trace which includes \mConstStyle{TH\_4}.
%
Because binding in \mConstStyle{TH\_4} does not add any further secrecy, there
is no harm in focusing on $g^{xy}$.
%

The same argument goes for the method where both parts authenticate via
signatures, but not when one or both parties use static DH key authentication.
%
When the Initiator is using a static DH key, the session key material is
$g^{xy}$, $g^{Iy}$ and $g^{Rx}$.
%
\knote{This is because EDHOC copied it from OPTLS. However, OPTLS was designed
    for the CK-model and differentiates between compromise and session state
    reveal queries. EDHOC is not as structurally designed and don't care about
separating the two and it may be sufficient to use only $g^{xy}$ in that case.
However: I suspect the authentication in OPTLS comes from the signatures
Krawczyk designed for HMQV, and in that case the signatures \emph{are} the key.
I need to investigate this.
}
%
%-------------------------------------------------------------------------- sub
\subsection{Authentication}
%-------------------------------------------------------------------------- sub
\subsubsection{Session key authentication}
\label{sec:mail-notes-session-key-auth}
%
When both parties use signatures, there is no $g^{Iy}$ or $g^{Rx}$, but instead
the key and entity authentication is based on that each party explicitly signs
the data they are supposed to agree on.
%
So in that case the session key material established consists of $g^{xy}$
only.
% 
This means that the initiator will, when receiving message 2, get a signature
covering $g^{xy}$ and $V$, and this is the main point: the Initiator can
verify this signature.
%
So when the Initiator completes, it knows:
%
\begin{itemize}
    \item that it communicated with party $V$,
    \item that $V$ has access to the session key material $g^{xy}$.
\end{itemize}
%
That is, the Initiator explicitly authenticated $V$ and the session key
material.
%
The corresponding holds for the Responder when it completes after receiving
message 3.
%
Therefore, when each party has completed their respective protocol
runs, they know that they agree on each others' identities and on which session
key material has been established.
%
 
When the Initiator is using a static DH key, the session key material is
$g^{xy}$, $g^{Iy}$ and $g^{Rx}$.
%
When sending message 2, the Responder does not know who the Initiator is, so
he cannot perform a computation that depends on the public key $g^{I}$ of the
Initiator and include that in message 2.
%
Specifically, when sending message 2, the Responder does not know $g^{Iy}$, so
he cannot inform the Initiator which value he will use for that part of the
session key and he cannot commit to it.
%
When the Initiator sends message 3 he performs a computation based on $g^I$
(i.e., resulting in $g^{Iy}$), sends this to the Responder and then completes.
%
So when the Initiator completes the protocol run, he has not received any
confirmation from the Responder which session key material they should use,
i.e., they have not agreed on it and hence the Responder has not authenticated
the key material to the Initiator.
%
But: The Initiator knows that \emph{if} his message 3 reaches the Responder
correctly, \emph{then} only if the Responder is $V$, will the Responder have
access to the session key material $g^{xy}, g^{Iy}, g^{Rx}$.
%
Without confirmation from $V$ in the Responder role, the Initiator cannot know
that they actually have agreed on the session key material when he completes
the run.
%
 
This situation does not occur in the OPTLS paper, because Krawczyk is not
taking client authentication into account.
%
Without client authentication in EDHOC, the corresponding session key material
would be only $g^{xy}$, $g^{Rx}$ and both the client/Initiator and
server/Responder have a access to that when processing message 2.
%
 
So, on the top of my head there are three options when the Initiator uses a
static DH-key (unless I missed something above):
\begin{enumerate}
    \item Accept implicit authentication,
    \item Sacrifice Initiator identity protection and include
        \mConstStyle{ID\_CRED\_I} in message 1, and, e.g., include
        \mConstStyle{ID\_CRED\_I} and/or \mConstStyle{CRED\_I}
        in \mConstStyle{TH\_2} or in \mConstStyle{AAD\_2},
    \item Include a fourth confirmation message from Responder to Initiator
        proving to the Initiator that the Responder knows $g^{Iy}$ (a MAC using
        the session key is enough).
\end{enumerate}
% 
\knote{John find 1 acceptable and would also like to use only $g^{xy}$ and
    session key for the STATIC methods. I would prefer not to use only $g^{xy}$,
    because Krawzcyk's proof for OPTLS is based on $g^{Rx}$ and uses $g^{xy}$
    only to get PFS. Note that responder can only compute $g^{Rx}$ if he has
    access to his private key, so the link to $g^{Rx}$ is stronger. However,
    indirectly it affects also the $g^{xy}$ via the \mConstStyle{MAC\_3}, so it
    may be OK still. But, using $g^{xy}$ to protect against session state
    reveal queries in addition to compromise queries revealing $g^{Rx}$ is
    useful with TPMs (c.f., the signature operation using the private key is
    done in the TPM for the SIG-SIG method).
}
There does not seem to be a similar issue with $g^{Rx}$.
%
When receiving message 2, the Initiator verifies \mConstStyle{MAC\_2}, which is
based on a key derived from $g^{Rx}$.
%
The \mConstStyle{MAC\_2} also covers \mConstStyle{CRED\_R} and
\mConstStyle{ID\_CRED\_R}, so the Initiator knows after verifying
\mConstStyle{MAC\_2} that $g^{Rx}$ is known to and coming from $V$.
%
An attacker would have to compute the discrete log of $g^{x}$ from the first
message to be able to forge that MAC.
% 

%-------------------------------------------------------------------------- sub
\subsubsection{Property: Implicit session key authentication}
\label{sec:mail-notes-session-key-imp-auth}
%

For the cases where the Initiator uses a static DH key, we prove implicit
authentication of Responder to initiator:
$$
\forall l. \mConstStyle{accept(l)} ==>
    \forall l'. \mConstStyle{sameKey}(l, l') ==> l'.pid = l.id
$$
which is an adaption from Delpech de Saint Guilheim
et.~al.~\cite{DBLP:journals/iacr/GuilhemFW19}.
%
The term $l.id$ is the identity of $l$, and $l.pid$ is the identity of $l$'s
peer.
%
The Initiator can still be explicitly authenticated to the Responder using
normal running/commit technique.
%



%-------------------------------------------------------------------------- sub
\subsection{Identity privacy}
%
Identity privacy does not apply to the PSK method.
%
For asymmetric methods, identity privacy means hiding \mConstStyle{ID\_CRED} and
\mConstStyle{ID\_CRED\_X}.

%-------------------------------------------------------------------------- sub
\subsubsection{Property identity privacy}
TBD

%-------------------------------------------------------------------------- sec
\section{Primitives}
%-------------------------------------------------------------------------- sub
\section{Encryption}
\label{sec:mail-notes-encr}
%
We model authenticated encryption with additional data (AEAD) in the standard
way.
%
We consider an AEAD transform as a single primitive providing both integrity
protection and encryption.
%

Message 2 is however encrypted using an XOR-pad that is generated using the KDF.
%
Even though the protocol intends for message to to enjoy integrity protection by
means of the signature, we cannot consider the signature in combination with
XOR as a single primitive.
%
We model XOR encryption using \mTamarin's built in xor operation, with one
xor-pad-term for each term in the message.
%
%-------------------------------------------------------------------------- sub
\section{Mixed issues}
%
%-------------------------------------------------------------------------- sub
\subsection{Unclear/underspecified identity handling}
\label{sec:mail-notes-identity}
%
This is a follow-up to my whining about that is unclear what an identity is
for the purpose of EDHOC, and what EDHOC is actually authenticating.
%
Casually, an AKE must guarantee that the session key is established with a
certain identity.
%
That is, a party must authenticate to the other: its identity and that it is
the one who shares the session key (explicitly or implicitly).
%
I think there is a "not-authenticated-whom-I-intended-attack".
%
It is there because:
\begin{itemize}
    \item It is not clear what the intention is with identities and
            authentication.
    \item The spec mixes up the concepts of roles (initiator/responder) and
        identities (parties who acts in a role).
\end{itemize}
%
The Initiator is assumed to know the identity V it is initiating the protocol
run with.
%
One identity V may have more than one \mConstStyle{CRED\_R}.
%
That is, the spec only RECOMMENDs that a one-to-one mapping between
\mConstStyle{CRED\_R}
and identity exists (Section 4.1), so that means the protocol must be secure
also when no one-to-one mapping exists.
%
Section 3.2 of EDHOC states:
\begin{itemize}
    \item Identity when PKI is used: name in certificate
    \item Identity when no PKI: the public key (bound to subject name?)
\end{itemize}

I did not find any text describing that the Initiator verifies that
\mConstStyle{ID\_CRED\_R} matches the intended identity in the responder role.
%
It only describes that the initiator should verify that the identity of the
responder is within an allowed set.
%
The latter is what I asked for earlier, so thanks for that.
%
But it is not enough.
%
Attack:
%
\begin{itemize}
    \item I configure my phone, printer and toaster as allowed set of parties
        to communicate with in my home network.
    \item Assume my phone sends message 1 to the intended responder printer.
    \item My toaster is hacked and replies with a message 2 and its
        own \mConstStyle{ID\_CRED\_R}.
    \item My phone verifies that the \mConstStyle{ID\_CRED\_R} is in the allowed set and
        continues.
    \item EDHOC completes and my phone is securely connected to my toaster,
        but that was not its intention.
\end{itemize}
%
I did not find any text that the Initiator verifies that the identity carried
in the \mConstStyle{ID\_CRED\_R} actually is the intended one.
%
It only describes that the initiator should verify that the identity of the
responder is within an allowed set.
%
The latter is what I asked for earlier, so thanks for that.
%
But it is not enough.
%
Attack:
%
\begin{itemize}
    \item I configure my phone, printer and toaster as allowed set of parties
        to communicate with in my home network.
    \item Assume my phone sends message 1 to the intended responder printer.
    \item My toaster is hacked and replies with a message 2 and its
        own \mConstStyle{ID\_CRED\_R}.
    \item My phone verifies that the \mConstStyle{ID\_CRED\_R} is in the allowed set and
        continues.
    \item EDHOC completes and my phone is securely connected to my toaster,
        but that was not its intention.
\end{itemize}
%
One could argue that the application should have configured the allowed set to
be only \mConstStyle{ID\_CRED\_R}s that matches the printer.
%
But this is not stated in the spec that I found.
%

For the responder role this does not matter, because the responder is reactive
and does not have an intention to communicate with a specific party.
%

Depending on what the intended use cases are for EDHOC I see two options:
\begin{itemize}
    \item The spec says that the allowed set must only contain \mConstStyle{ID\_CRED\_R}s
            that map to the same identity. Or
    \item The spec says that the allowed set may contain \mConstStyle{ID\_CRED\_R}s that map
            to different identities. This may be useful if, for example, I
            have two printers and my phone is OK with connecting to any of
            them. In the latter case, the spec need to state the Initiator
            does not know which identity it is initiating the connection to,
            but that it is connecting to one of the identities in the
            allowed set. This is some kind of anycast.
\end{itemize}
%
Even if the spec uses the strategy to not specify which is the case to
"allow for all possibilities" it needs to explain what to do in each of the
cases.
%
I don't believe leaving it undefined is a good idea.
%

%-------------------------------------------------------------------------- sec
\section{Comparison with existing frameworks}
\label{sec:mail-notes-noise}
It can be seen that the first two messages in the Static-DH mode of EDHOC correspond perfectly to the first two messages of the XX pattern of the NOISE framework. However, in the third message, the XX pattern requires the initiator to send their static key, followed by an encrypted payload using a key derived by a combination of the static key and an ephemeral key. In EDHOC, the static key and the payload are both encrypted in the same key, one depending on the key used for the second message, and on the static key of the initiator. This perhaps diverges from the XX pattern of NOISE, and one can no longer directly claim that EDHOC enjoys the same properties as XX. 


\end{document}
