% !TEX root =  main.tex

%Description of \m\mEdhoc and main changes from last verified version

\subsection{About the protocol}
%\kedit{Since \mEdhoc is a key-exchange protocol, it is expected to provide
%    perfect forward secrecy and identity protection. For providing application
%    layer security, \mEdhoc makes use of \mCbor Object Signing and Encryption
%(\mCose)}{}~\cite{rfc8152}.
%\kedit{This works well in cases where transport layer security is not sufficient, or where multiple underlying protocols need to be accounted for.
%}
{The purpose of \mEdhoc{} is to establish an authenticated security context
for the \mOscore{} security protocol.
}
%
%\knote{Sorry for the mess. The edit macro can't contain citations it seems.}
Parties in the \mEdhoc{} protocol can authenticate
themselves to peers using one out of three authentication methods -- digital
signatures
(\mSig), challenge-response signatures based on static Diffie-Hellman keys
(\mStat) or pre-shared symmetric keys (\mPsk).
We will refer to these authentication methods simply as methods below.

Four combinations of the \mSig{} and \mStat{} authentication methods are possible for the two parties, while the \mPsk{} authentication method can only be run by {a party when the other party is} also authenticating with a \mPsk. We refer to these combinations also as methods, denoted \mSigSig, \mSigStat, \mStatStat, \mStatSig{} and \mPskPsk. {We refer to a method where at least one party uses \mSig{} as a \mSig-based method and similarly for \mStat{} and \mPsk.}
Although the terminology is overloaded, it should be clear from the context whether a single agent authentication method or a combined method is intended.

We describe each of these methods in detail later in this section.
%\knote{I removed paragraph break here. The above sentence could start off the next paragraph.}
%The communicating parties must agree on the method and cipher suite used for encryption as part of the first message. The parties exchange ephemeral public keys, compute the shared secret, and derive symmetric application keys from this secret.
The \mSig-based methods of \mEdhoc{} {are derived from} \mSigmaI, a variant of the \mSigma{} protocol which provides identity protection for the initiator. However, \mEdhoc{} deviates from \mSigmaI{} in several aspects, e.g., \mEdhoc{} uses Mac-then-Sign like \mTls{} instead of Sign-then-Mac. The \mStat-based methods use challenge-response signatures using static Diffie-Hellman keys, and proceed along the lines of \mOptls. The challenge-response signatures use ephemeral Diffie-Hellman keys as challenges and an \mAead{} MAC to protect the data to be signed~\cite{aead,rfc5116,DBLP:conf/eurosp/KrawczykW16}.

%\mEdhoc allows the initiator and responder to run different methods, combining \mSig and \mStat -- for example, the initiator might run a \mSig-based method, while the responder is running a \mStat-based method. This set of methods is not covered in previous versions of \mEdhoc, which only had a single \mSigma asymmetric key method (corresponding to the \mSigSig method shown in Section~\ref{sec:methods}). This allows one party to use a \mSigma style of authentication using signatures, while the other can use static DH keys. %In Figure~\ref{fig:edhocasym}, a template for all the asymmetric key methods of \mEdhoc is shown. 
%
%\begin{figure}[!h]
%\centering
%\includegraphics[scale=0.3]{Images/asym.png}
%\caption{A template for the asymmetric key methods of \mEdhoc}
%\label{fig:edhocasym}
%\end{figure}

%\subsection{Background, comparison with~\cite{DBLP:conf/secsr/BruniJPS18}}
%\knote{This section is again repetition from the introduction. Related work on
%    analyzing \mEdhoc should be in the related work section.
%    This section should describe \emph{what} \mEdhoc is IMHO.}
%The first version of \mEdhoc was proposed in March 2016 to a working group investigating lightweight authenticated key exchange protocols [\mcneed]. There has been a focus on formally verifying that the protocol satisfies the properties expected of it right from the beginning. 
%
%The 2018 work~\cite{DBLP:conf/secsr/BruniJPS18} by Bruni et al performed a formal verification of version 08 (\url{https://tools.ietf.org/html/draft-selander-ace-cose-ecdhe-08} \mcfix) of \mEdhoc. The protocol and properties are modelled and verified in the \mProverif tool. This version of the protocol belongs to the \mSigmaI family of protocols, and has two modes -- one with asymmetric keys, and one with pre-shared symmetric keys. Bruni et al showed that this version satisfies the requisite properties of identity protection, (perfect forward) secrecy of data, and strong authentication, upon completion of the protocol.
%
%\mEdhoc has undergone a lot of change since version 08, as will be described in the following sections, and the formal verification of the current version, therefore, is a worthwhile exercise.

\subsection{Relations to \mSigma, \mOptls{} and \mNoise{}}
\subsubsection{\mSigma{}}
As noted above, the \mSigSig{} method of \mEdhoc{} is a very close
derivative of \mSigmaI{} and the version from \mEdhoc{} v08 was carefully analyzed by
Bruni~et.~al.~\cite{DBLP:conf/secsr/BruniJPS18}, as was the \mPskPsk{} method of that version of \mEdhoc{}. Since both of these methods are close in spirit to the versions already analyzed in that paper, we will not focus on those in the reminder of this paper. We do however model them to the same extent and
verify their properties with the same care as the other methods.

\subsubsection{\mOptls{}}
%\knote{BTW, I silently added \{\} to all macros I found that needed it for trailing space}
\mOptls~\cite{cryptoeprint:2015:978} is a key-exchange protocol developed {with the intention to serve as a basis for
the} handshake protocol in \mTls\ 1.3. {\mOptls{} is a one round-trip
protocol providing server to client authentication, but not vice versa.}
The server has a
secret static key $s$ and a public static DH key $g^{s}$. The client
first sends to the server a {challenge} message composed of a fresh nonce and some negotiated parameters, along with an {ephemeral DH} key $g^{x}$. The server responds with a nonce of its own, some other negotiated parameters, an {ephemeral DH} key $g^{y}$, and a MAC of all these values computed using a key derived from $g^{xs}$. At the end of the protocol, a session key is derived {by each party} from $g^{xs}$ and $g^{xy}$. The former ensures secrecy for as long as $s$ {remains secret}, and the latter provides perfect forward secrecy in case $s$ is compromised. {The core idea is that, when receiving the response message, the client deduces that only the intended server could have computed $g^{xs}$ with the knowledge of $s$,  and from $g^{xs}$ and $g^{xy}$ derived the MAC key.  Under this assumption, the client further deduces that anything covered by the MAC originated from the intended server. This construction is sometimes referred to as a challenge-response signature. It is very efficient in terms of the number of transmitted bits.

{In} \mEdhoc, a party authenticating with with the \mStat{} method essentially acts as an \mOptls{} client and the other party as an \mOptls{} server. There are differences, e.g., that \mEdhoc{} reuses
the $g^x$ as a nonce, while \mOptls{} makes an explicit point about keeping these as separate elements.  The \mStatStat{} method can be thought of as interleaving two \mOptls{} sessions{, where the second \mEdhoc{} message carries both the response message of the first \mOptls{} session and the request message of the second \mOptls{} session.}

\subsubsection{\mNoise{}}
For the \mStatStat{} method, since both parties are running a \mStat{} method,
there is an obvious analogue in terms of the \mNoise{}
framework~\cite{perrin2016noise}. \mNoise{} is a framework for a certain class
of security protocols where all parties {authenticate} using
static DH keys, and whose structure also meets certain requirements of encryption etc. 

The closest \mNoise{} ``pattern'' to the \mStatStat{} method of \mEdhoc{} is the
XX pattern. It can be seen that the first two messages in this method of
\mEdhoc{} correspond perfectly to the first two messages of the XX pattern of
the \mNoise{} framework. However, in the third message, the XX pattern requires
the initiator to send their static key, followed by an encrypted payload using a
key derived by a combination of the static key and an ephemeral key. In
\mEdhoc{}, the static key and the payload are both encrypted in the same key, one depending on the key used for the second message, and on the static key of the initiator.
\knote{There is something odd with the sentence above. What does it mean that
    the static key and payload are encrypted in the same key?
}
\vnote{The payload is \mADthree. The static key is \mCredi. Both of these are encrypted using \mKthreeae, which depends on \mKtwoe (obviously) -- I agree that the sentence about the key depending on the static key of the initiator is wrong. Unless I'm mistaken in my understanding?}
This perhaps diverges from the XX pattern of \mNoise{}, and one can no longer
directly claim that \mEdhoc{} enjoys the same properties as XX. As a potential
direction for future work, we relegate the further investigation of this pattern
of the \mNoise{} framework to {determine any} correspondence with the
\mStatStat{} method.

\section{Framework}
\label{sec:edhocFramework}
\knote{I read the Noise spec. There are a lot more similarities to \mEdhoc{}
    than I understood earlier. For instance the idea of overlaying two protocols
    is already in \mNoise. Also the \mPskPsk{}-method may also be mapping the
    a PSK-pattern in \mNoise. This section will describe the \mEdhoc{} framework
    because it helps frame the detailed descriptions of the methods that follow.
    I plan to also point out more similarities to \mNoise{} at the same time.
}
%\section{Methods and features of \mEdhoc}\label{sec:methods}
We are now ready to dive into the details of the various methods supported by
\mEdhoc. Note that there are some fundamental components that are common to all
the \mEdhoc methods. These components are the aforementioned \mCose{} objects,
\mAead{} encryption~\cite{aead}, and, perhaps most importantly, the key
derivation function (KDF) utilized to generate pseudorandom strings and
encryption/decryption keys. We give a quick introduction to the \mCose{} objects and the \mAead{} algorithm for the sake of completeness, and we then flesh out the workings of the KDF, before moving on to discuss the {overall structure and the details of the various} methods.
\knote{I plan to add some text (possibly a new subsection) giving an overview of
    the general framework in which each of the methods can be put in. I only
    scanned the method descriptions, but got the impression they go straight into
details on the particular method. I should have it done by Monday.}
\vnote{Yes, they do. Please go ahead if you want to add such text.}

\subsection{Algorithms and objects reused by \mEdhoc{}}
\mEdhoc{} uses \mCose{} for cryptographic operations and \mAead{} as an algorithm for encryption. We briefly describe both in this section.

\subsection{\mCose}
To achieve small message sizes and reuse implementations often supported by constrained IoT devices~\cite{Lorawan1,Lorawan2}, \mEdhoc{} uses Concise Binary Object Representation (\mCbor) for data encoding and \mCbor{} Object Signing and Encryption (\mCose) for cryptographic operations. 

\vnote{To Karl: What citation should I put here? You've asked for one, but I assume the only citation possible is the specification itself, which I don't think we need to refer to yet again.}
\knote{I was thinking of something that shows that \mCbor{} and \mCose{} are
    widely implemented. I did a quick google but didn't find anything good.
    Propsal: Check if the specification gives a reference (they should since
    this is a driving force for the design), and if there is none, ask
G\"oran if he has some good link.}
\vnote{I went through the specification, but there is never a citation for such a claim. The specification only links to the RFCs for \mCbor{} and \mCose{}. Technically you don't need a citation to say that reuse will keep computation low, but you do need a citation to say that these two are often supported by IoT devices. Let me see what I can find from Google Scholar, otherwise we can ask G\"oran. UPDATE: Turned out I had added these citations in the earlier ``overview'' subsection -- I think the two Lorawan papers should do?}

\mCbor{} is a data format designed for small code size and small message size. \mCose{} is used to create and process signatures, MACs, and encryption using \mCbor{} to encode and decode input and output objects.

All \mCose{} objects (also called structures) are built using the \mCbor{} array type. All objects share in common the first three elements of the array. These are the set of protected header parameters (these are the parameters that ought to be cryptographically protected), the set of unprotected header parameters, and the content (plaintext or ciphertext, depending on the type of \mCose{} object) of the message. Header parameters include the cryptographic algorithm to be used, content type, parameters for identifying a cryptographic key etc. The algorithm to be used might need to be authenticated, and therefore put in protected parameters, while content type indicates the type of data in the payload or ciphertext fields, and can remain unprotected.

The other elements of the \mCbor{} array may vary, depending on the kind of \mCose{} structure being considered (determined by the operation performed, and algorithm used). \mCose{} supports many different kinds of structures, for signatures, encryption etc. We will encounter only two kinds of \mCose{} objects as part of the description of \mEdhoc, namely the \mCoseEncrypt{} (the 0 indicates that the recipient is implicitly identified by the key) and \mCoseSign{} (the 1 indicates that there can only be one signer) objects.

A \mCoseEncrypt{} structure is a \mCbor{} array with (as mentioned above) a field for protected data, a field for externally supplied application data, and a field for the ciphertext. It also contains a text field which contains the string ``Encrypt0'' to denote the use of the method with implicit recipient. All these fields store the corresponding bitstrings.

Similarly, a \mCoseSign{} structure contains protected data, unprotected data, and the plaintext to be signed. It also contains a text field which contains the string ``Signature1'', and a field for the actual signature.

One of the features offered in the \mCose{} document is the ability for applications to provide additional data to be authenticated, which is not carried as part of the \mCose{} object. Applications that use this feature need to define how such externally supplied authenticated data is to be constructed. Therefore, both these objects can also carry some such external data along with (but separate from) their other contents.

\vnote{I've rewritten this section, but left the older one commented in the TeX file. Let me know if this reads better and makes more sense?}

%\subsubsection{\mCose{}}
%\kedit{}{To achieve small message sizes and reuse implementations often
%    supported by constrained IoT devices (\mcneed), \mEdhoc{} uses Concise
%    Binary Object Representation (\mCbor) for data encoding and \mCbor{} Object
%    Signing and Encryption (\mCose) for cryptographic operations.
%}
%\mCbor{} is a data format designed for small code size and small message size.
%\kedit{The \mCbor Object Signing and Encryption (\mCose) protocol}\mCose{} is
%used to create and process signatures, MACs, and encryption using \mCbor{} to
%encode and decode input and output objects.
%
%A \mCose{} object contains bitstrings corresponding to (cryptographically)
%protected parameters, unprotected parameter\kedit{}{s}, and the
%payload\kedit{}{s} to be signed. This \kedit{}{object} can be
%signed/encrypted/MACed, to obtain a resultant \mCose{} object containing the bitstring corresponding to the operation.
%\knote{The above is unclear to me. I think I can understand what it means
%    because I already know a bit about \mCose, but the "protected" and
%    "unprotected" parameters of the object for instance will not be clear to
%anyone who has not read the \mCose{} RFC. Maybe you could elaborate a bit more
%on this?}
%The respective algorithms may also be fed some externally supplied data, which
%is carried along with the \mCose{} object, but is not part of it.
%\knote{Unless we explain why the external data above could be or how it is used,
%the sentence will not help the reader I think.}
%\mEdhoc{} only uses the \kedit{sxignature}{signature} and encryption objects of \mCose.
%\knote{It has not been explained that these two types of objects exists yet. The
%    only thing the reader knows at this point is that cose objects can be signed
%    and encrypted, but that there is a specific object that has
%"encryption-type" or "sinature-type" has not been explained.}
%For encryptions, \mCose{} supports two different methods -- one where the
%recipient is not needed because the key is known implicitly, and one for all
%other cases. \mEdhoc{} only uses the former. 
%\knote{Here it is the same issue with the text. I know what receipient means in
%    terms of \mCose, but a reader who does not get much out of the above
%    sentence I belive. The last sentence states that \mEdhoc{} only supports the
%    former: maybe the above two sentences can simply be removed then?
%}
%
%A \mCoseEncrypt{} structure as used in \mEdhoc{} is a \mCbor{} array with a text field which contains the string ``Encrypt0'' to denote the use of the method with implicit recipient, a field for protected data, a field for externally supplied application data, and a field for ciphertext. All these fields store the corresponding bitstrings.
%\knote{This is the first time the \mCoseEncrypt{} structure is mentioned. How
%    does it relate to \mCose objects?  Maybe some more explanation is needed.
%}
%\vnote{I think this section needs a full overhaul. I'll rewrite it.}

\subsubsection{\mAead{}}
Authenticated encryption with associated data, or \mAead{}, {is, as the name suggests, an operation that provides
both data authentication and encryption}~\cite{aead}. The {encryption} algorithm takes in a
key, a unique nonce, a plaintext, and some associated data, and outputs a
ciphertext. The plaintext is both authenticated and encrypted, while the
associated data is merely authenticated. The ciphertext is at least as long as
the plaintext. When the plaintext is empty, the \mAead{} \kedit{}{encryption}
algorithm acts as a MAC on the associated data.

The associated data {field contains} information {which it might be desirable to authenticate, but} must be left unencrypted to allow the system to function properly.%\knote{Is the two previous sentences not saying the same thing?} %\vnote{Sure, I've removed one.} 
Authentication {of associated data} is provided without {including it in the ciphertext.} {The \mAead{} decryption operation returns the decrypted ciphertext if   the MAC verifies. If the MAC fails to verify, the operation returns
nothing.}

\subsection{\mEdhoc's key derivation function (KDF)}
\knote{This section (or another section) could describe the key hierarchy used
    by \mEdhoc, i.e., with session key material at the top (derived from
    $g^{xy}$ and possible static DH-keys, and from this, the PRKs are derived
    and from those the AEAD keys for each message and so on. Such a description
    and perhaps a figure, would make it easier to read the method sections
    later.
}
\vnote{I've restructured the text and added a couple of paragraphs. I don't think a figure will do very much, but I can try to add one if you want.}
\knote{This is great. One thing though: the hierarchy itself is kind of part of
    \mEdhoc{} itself and not a building block like the AEAD transform, the KDF
    and so on. It would be better to break the hierarchy out into a section of
    its own, so that we get a separation between the components that \mEdhoc{}
    reuses and which components it constructs itself.
    A simile: If I were to implement \mEdhoc there are some parts I would import
    as libraries: COSE, HKDF, AEAD transforms. Then there are other parts I need
    to code myself: the key hierarchy, what to put in messages and so on. These
    parts are different in my mind and should be described separately to make it
    clear what is new and what is old.
}
\vnote{This is a great point, and one I had thought about, but forgot to fix after I had added the figure. Fixed now.}
One of the most important building blocks for \mEdhoc{} is the key derivation function based on \mHkdf~\cite{rfc5869}, which is used to generate the pseudorandom strings and keys for the encryption operations in the communicated messages. The three pseudorandom strings (\mPRKtwo, \mPRKthree, and \mPRKfour) are derived using the \mHkdfExtract{} function, while the keys  are generated using the \mHkdfExpand{} function. Both these functions are based on \mHmac~\cite{rfc2104}, which is a hashing system for message authentication.

%The \mHkdfExtract function is run with a salt and some input keying material (IKM) as input. It produces as output a pseudorandom string. For the \mPskPsk method alone, the salt is the key pre-shared between the initiator and the responder, while for the other methods, it is empty. The IKM for all \mEdhoc methods is the ECDH shared secret $g^{xy}$. 

There is a natural hierarchy obeyed by these pseudorandom strings. We start with the IKM $g^{xy}$. The pseudorandom string \mPRKtwo is generated by running the \mHkdfExtract{} function on the IKM (along with the pre-shared key as salt, in the case of the \mPskPsk method -- for the other methods, the salt is empty). For the \mPskPsk and \mSigSig methods, only one pseudorandom string is used (i.e. $\mPRKtwo = \mPRKthree = \mPRKfour$). For the three \mStat-based methods, however, we use at least two pseudorandom strings. \mPRKthree can either be equal to \mPRKtwo, or obtained by running \mHkdfExtract{} on \mPRKtwo and one other input. This other input is obtained by an exponentiation operation between a long term key and an ephemeral key -- the exact inputs to this operation depend on the role and the method being run, and will be described in detail later. Similarly, \mPRKfour can either be equal to \mPRKthree, or obtained by running \mHkdfExtract{} on \mPRKthree and one other such input. Thus, \mPRKtwo is used as input to the KDF to generate \mPRKthree, which is in turn used to generate \mPRKfour.

The keys used for the \mAead{} encryptions are augmented with the suffix `e', for encrypting the ciphertext, or with the suffix `m', for generating the MAC (MACs only feature in the methods involving asymmetric keys). There are four keys -- \mKtwom, \mKtwoe, \mKthreem, and \mKthreeae. The suffixes in the names of the pseudorandom strings indicate which keys they are used to generate. \mPRKtwo is used to generate the key \mKtwoe along with the appropriate transaction hash. \mPRKthree is used for the keys \mKtwom and \mKthreeae, while \mPRKfour is used to generate \mKthreem.

\begin{figure}[htp]
\centering
% !TEX root =  edhocProtocol.tex

% Start the picture
\begin{tikzpicture}[%
    >=latex,              % Nice arrows; your taste may be different
    start chain=going below,    % General flow is top-to-bottom
    node distance=6mm and 60mm, % Global setup of box spacing
    every join/.style={norm},   % Default linetype for connecting boxes
    ]
% ------------------------------------------------- 
% A few box styles 
% <on chain> *and* <on grid> reduce the need for manual relative
% positioning of nodes
\tikzset{
terminput/.style={rounded corners, text width=6em},
term/.style={rounded corners},
  base/.style={draw, thick, on chain, on grid, align=center, minimum height=6ex},
  dhkbox/.style={draw=Goldenrod1, fill=Goldenrod1!25, rectangle},
  dhk/.style={base, dhkbox},
  prkbox/.style={draw=Red3, fill=Red3!25, rectangle},
  prk/.style={base, prkbox},
  %hkdfext/.style={base, draw=Green3, fill=Green3!25, isosceles triangle, isosceles triangle apex angle=60, anchor=base, shape border rotate=-90, text width=6em},
  hkdfext/.style={base, draw=Green3, fill=Green3!25, rectangle},
  %hkdfexp/.style={base, draw=orange, fill=orange!50, isosceles triangle, isosceles triangle apex angle=60, anchor=base, shape border rotate=-90, text width=6em},
  hkdfexp/.style={base, draw=orange, fill=orange!50, rectangle, text width=8em},
  keybbox/.style={draw=Blue3, fill=Blue3!25, rectangle},
  keyb/.style={base, keybbox, text width=4em},
  % -------------------------------------------------
  norm/.style={->, draw, Blue3},
  cond/.style={base, draw=black, fill=white, diamond},
  txt/.style={base, draw=none, fill=none}
  }
% -------------------------------------------------
% Start by placing the nodes
\node [dhk] (p0) {$\mGxy$};
%\node [hkdfext, join] (h1) {\mHkdfExtract};
\node [hkdfext, right=2.2cm of p0, join] (h1) {\mHkdfExtract};
\node [prk, join] (p2) {\mPRKtwo};
\node [cond, join] (c1) {$m \in \{1, 3\}$};

\node [prk, join, below=8mm of c1.south] (p3) {\mPRKthree};
\node [cond, join, below=10mm of p3.south] (c2) {$m \in \{2, 3\}$};
\node [prk, join, below=8mm of c2.south] (p4) {\mPRKfour};

\node [hkdfext, right=3cm of c1] (h3) {\mHkdfExtract};
\node [hkdfext, right=3cm of c2] (h5) {\mHkdfExtract};

\node [hkdfexp, shape border rotate=180, left= 3cm of p4] (h6) {\mHkdfExpand};
\node [keyb, join, left=3cm of h6] (k3) {\mKthreem};
\node [hkdfexp, shape border rotate=180, below= 1cm of h6] (h9) {\mHkdfExpand};
\node [txt, join, left=1cm of h9.west] (t4) {EDHOC-Exporter()};

\node [hkdfexp, shape border rotate=180, left= 3cm of p3] (h4) {\mHkdfExpand};
\node [keyb, join, left=3cm of h4] (k2) {\mKtwom};

\node [hkdfexp, shape border rotate=180, left= 3cm of p2] (h2) {\mHkdfExpand};
\node [keyb, join, left=3cm of h2] (k1) {\mKtwoe};

\node [hkdfexp, shape border rotate=180, below= 1cm of h4] (h8) {\mHkdfExpand};
\node [keyb, below=1cm of k2] (k2b) {\mKthreeae};

\node [txt, left=1cm of k1.west] (t1) {Enc (XOR) in m2};
\node [txt, left=1cm of k2.west] (t2) {\mMactwo\\ (signed if $m \in \{0,2\}$)};
\node [txt, left=1cm of k2b.west] (t2b) {\mAead\ in m3};
\node [txt, left=1cm of k3.west] (t3) {\mMacthree\\ (signed if $m \in \{0,1\}$)};

\draw [->, norm] (k1.west) -- (t1.east);
\draw [->, norm] (k2.west) -- (t2.east);
\draw [->, norm] (k2b.west) -- (t2b.east);
\draw [->, norm] (k3.west) -- (t3.east);

\draw [->, norm] (p3.south) -- ++(0,-0.58) -- (h8);
\draw [->, norm] (h8) -- (k2b);
\draw [->, norm] (p2) -- (h2); 
\draw [->, norm] (c1) -- (h3);
\draw [->, norm] (h3.south) -- ++(0,-1) -- ++(-3,0);
\draw [->, norm] (p3) -- (h4); 
\draw [->, norm] (c2) -- (h5);
\draw [->, norm] (h5.south) -- ++(0,-1) -- ++(-3,0);
\draw [->, norm] (p4) -- (h6);
\draw [->, norm] (p4.west) ++(-0.25,-0) -- ++(0,-1) -- (h9.east);

\node [terminput, right = 1cm of h1] (u1) {Seed};
\draw [->, dotted, shorten >=1mm] (u1) -- (h1);


\node [dhk, above = 1.5cm of h3] (u2) {$\mGrx$};
\draw [->, norm] (u2.south) -- (h3.north);

\node [dhk, above = 1.5cm of h5] (u3) {$\mGiy$};
\draw [->, norm] (u3.south) -- (h5.north);


\node [term, above = 1.4cm of h4] (u5) {\mTHtwo};
\draw [->, dotted, shorten >=1mm] (u5) -- (h4);
\draw [->, dotted, shorten >=1mm] (u5) -- (h2);

\node [term, above = 1.2cm of h6] (u6) {\mTHthree};
\draw [->, dotted, shorten >=1mm] (u6) -- (h6);
\draw [->, dotted, shorten >=1mm] (u6) -- (h8);

%\node [term, below= 1cm of h9] (u7) {\mTHfour};
\node [term, below=0.5cm of p4] (u7) {\mTHfour};
%\draw [->, dotted, shorten >=1mm] (u7) -- (h9.south);
\draw [->, dotted, shorten >=1mm] (u7.west) -- ([yshift=-0.5em] h9.east);

\matrix [draw, ultra thick, double, below=3.5em of t1.east] {
  \node [dhkbox, semithick, label=right:DH key] {}; \\
  \node [prkbox, semithick, label=right:Intermediate key material] {}; \\
  \node [keybbox, semithick, label=right:Key for \mAead{} or \mXor] {}; \\
};

%
% ------------------------------------------------- 
% 
%\path (h2.east) to node [near start, yshift=1em] {$n$} (c3); 
%  \draw [o->,lccong] (h2.east) -- (p8);
%\path (p3.east) to node [yshift=-1em] {$k \leq 0$} (c4r); 
%  \draw [o->,lcnorm] (p3.east) -- (p9);
% -------------------------------------------------
% A last flourish which breaks all the rules
%\draw [->,MediumPurple4, dotted, thick, shorten >=1mm]
%  (p9.south) -- ++(5mm,-3mm)  -- ++(27mm,0) 
%  |- node [black, near end, yshift=0.75em, it]
%    {(When message + resources available)} (p0);
% -------------------------------------------------
\end{tikzpicture}
% =================================================

\caption{A diagram to illustrate the KDF key hierarchy}
\label{fig:kdfdiagram}
\end{figure}

In order to generate these keys, the \mHkdfExpand{} function is used. This is run with a pseudorandom string, an info string, and the length of the output keying material (OKM) as input. The pseudorandom string is generated using \mHkdfExtract{} as above. The info string contains details of the \mAead{} encryption algorithm used, length of the OKM, and the transaction hash as used for the specific method (a detailed discussion follows, for each method). In case the length $l$ of the OKM is shorter than that of the transaction hash, the OKM is obtained by taking the first $l$ bits of the result of running \mHmac{} on the pseudorandom string, and a concatenation of 0x01 and the info strings. The KDF hierarchy is illustrated in Figure~\ref{fig:kdfdiagram}.



We now discuss each of the \mEdhoc{} methods in detail over the next few sections.

\subsection{\mPskPsk{} method}
In this method, the initiator and responder are assumed to have a pre-shared key (\mPsk) which is secret to them, and can be retrieved by the responder using a public part of the first message (\mIDPsk). This method corresponds to the symmetric key method of \mEdhoc{} v08. 

In the first message, the initiator sends a message consisting of parameters identifying the method and the correlation technique used (\mMethod), the cipher suites ranked in order of preference (we will come back to this later, in Section~\ref{sec:ciphersuite}), the initiator's ephemeral key (\mGx), their connection identifier (\mCi), the \mIDPsk identifier, and (optional) auxiliary data (\mADone). The responder, upon receipt of this message, must verify that the selected cipher suite is supported (see Section~\ref{sec:ciphersuite} for a description of how this is done),  and pass \mADone to the security application which needs it. Note that this first message is common to all \mEdhoc{} methods.

The second message, sent by the responder, is composed of \mCi, the responder's ephemeral key \mGy, their connection identifier \mCr, and a \mCose{} object. This contains as external data the transaction hash of the first message (\mTHtwo), along with an \mAead{} encryption of the (optional) auxiliary data \mADtwo. The key used for this (\mKtwoae) is derived using the \mEdhoc{} key derivation function with \mTHtwo and the pseudorandom string \mPRKtwo as input, while the associated data for the \mAead{} encryption is constructed by concatenating a constant string, plaintext \mhplain, and \mTHtwo. Recall that the PRKs in this method are constructed by running \mHkdfExtract{} using \mPsk{} as salt. 

The initiator, upon receipt of this message, sends back \mCr, followed by a \mCose{} object containing an \mAead{} encryption of auxiliary data \mADthree, along with the hash of \mTHtwo and the encryption object received from the responder (which is referred to as \mTHthree) as external data. There is no protected data for this encryption. As earlier, the key used (\mKthreeae) is derived by supplying \mTHthree and \mPRKthree (which is the same as \mPRKtwo, for this method) as input to the KDF, and the associated data is \mTHthree. An abstract description is shown in Figure~\ref{fig:edhocpsk}.

\begin{figure}[!h]
\centering
%\includegraphics[scale=0.3]{Images/psk.png}
\scalebox{.7}{
\tikzset{>=latex, every msg/.style={draw=thick}, every node/.style={fill=white,text=black}}
\begin{tikzpicture}
    \node (ini) at (0, 0) {Initiator};
    \draw [very thick] (0, -0.5) -- (0,-11);
    \draw [very thick] (11.5, -0.5) -- (11.5,-11);
    \node[below of=ini,fill=white,text=black] () {Knows $g,\ \mPsk,\ \mIDPsk,\ \mADone,\ \mADthree$};
    \node (res) at (11.5,0) {Responder};
    \node[below of=res] () {Knows $g,\ \mPsk,\ \mIDPsk,\ \mADtwo$};
    \action{4em}{ini}{Generates \mMethod,\ \mSuites,\ \mCi,\ $x$\\$\mGx = g^{x}$};
    \msg{8em}{ini}{res}{\mMsgone: \mMethod, \mSuites, \mGx, \mCi, \mIDPsk, \mADone};
    \action{9em}{res}{$
      \begin{array}{c}
        \text{Generates } \mCr,\ $y$\\
        \mGy = g^{y}\\
        \mTHtwo = \mHash(\mMsgone, g^{y})\\
        \mPRKtwo = \mHkdfExtract(\mPsk, g^{xy}) \\
        \mKtwoae = \mHkdfExpand(\mPRKtwo, \mTHtwo)
      \end{array}$};
    \msg{18em}{res}{ini}{\mMsgtwo: \mCi, \mGy, \mCr, $\overbrace{\mAead(\mKtwoae; \mTHtwo, \mADtwo)}^{\mCipher}$};
    \action{19em}{ini}{$
      \begin{array}{c}
       \mTHtwo = \mHash(\mMsgone, \mGy)\\
       \mPRKtwo = \mHkdfExtract(\mPsk, g^{xy}) \\
        \mKtwoae = \mHkdf(\mPRKtwo, \mTHtwo)\\
        \mTHthree = \mHash(\mTHtwo, \mCipher)\\
        \mPRKthree = \mPRKtwo \\
        \mKthreeae = \mHkdfExpand(\mPRKthree, \mTHthree)
      \end{array}$};
    
    \msg{28em}{ini}{res}{\mMsgthree: \mCr, \mAead(\mKthreeae; \mTHthree; \mADthree)};
    \action{29em}{res}{$
    \begin{array}{c}
    	\mPRKthree = \mPRKtwo \\
        \mKthreeae = \mHkdfExpand(\mPRKthree, \mTHthree)
    \end{array}$};
    \draw [line width=2mm] (-2,-11) -- (2,-11);
    \draw [line width=2mm] (9.5,-11) -- (13.5,-11);
    \end{tikzpicture}}
\caption{The \mPskPsk{} method of \mEdhoc. $\mAead(x; y; z)$ is used to denote \mAead{} encryption where $x$ is the key, $y$ is a tuple of the protected and external data, and $z$ is the plaintext.}
\label{fig:edhocpsk}
\end{figure}


\subsection{\mStat-based methods}
\mEdhoc{} allows for three \mStat-based methods -- two where only one participant has a static Diffie-Hellman key (while the other uses signatures), and one where both do. 

In the first message, the initiator includes an identifier for the method, a preference-ordered list of cipher suites, their ephemeral key \mGx, their connection identifier \mCi, and some optional plaintext \mADone. This message is common to all the four methods below involving asymmetric keys. 

The responder, upon receipt, verifies the cipher suites, passes any \mADone to the application that needs it, and proceeds to construct and send the second message. This message contains (not necessarily all of) \mCi, \mGy, \mCr, and an encrypted term. The initiator, upon getting this message, sends out a message containing an encrypted term. Depending on the methods being run by the initiator and responder, the exact contents of these messages may vary. We will now describe each of these methods and the second and third messages therein in detail.

\subsubsection{\mSigStat}
The initiator runs a \mSigma-based method, with signatures, while the responder operates with a static Diffie-Hellman key and MACs. Both the initiator and the responder generate ephemeral secrets $x$ and $y$, and also have their respective long term public-private authentication keypairs. The public authentication keys have identifiers for retrieval as well.

The responder generates an ephemeral secret $y$, and an identifier \mCr. Then, it generates the pseudorandom strings. \mPRKtwo is obtained, as mentioned earlier, by running the KDF algorithm on the empty salt and the shared secret $g^{xy}$. Then, \mPRKthree is obtained by running the KDF algorithm on \mPRKtwo and \mGrx -- this is the result of raising the initiator's ECDH ephemeral public key \mGx to the responder's private authentication key \mLtkr. These pseudorandom strings will be used to generate encryption keys for this message.

The encrypted term for the second message is constructed as follows. First, we build the \mCose{} object that is the inner MAC, since the responder runs \mStat. The protected part of this object is an identifier for retrieving the responder's public authentication key, namely \mIdcredr~\footnote{\vedit{}{The \mIdcredr parameter may be used by the initiator to look up the responder's static public key. While there need not be a unique correspondence between \mIdcredr and the actual \mCredr used by the responder, the initiator should be able to go through the list of matches to look up and obtain the ``correct'' \mCredr. Similarly for the responder with \mIdcredi and \mCredi. However, in our modelling, we choose to model the map from \mIdcredr to \mCredr (and that from \mIdcredi to \mCredi) as unique.}}. The externally supplied data is the transaction hash \mTHtwo (obtained by hashing the first message and \mGy),  the responder's public authentication key \mCredr, and (optional) auxiliary data \mADtwo. The plaintext is empty. The key used for the encryption \mKtwom is the output of the KDF on being fed as input the pseudorandom string \mPRKthree and \mTHtwo. The resulting encrypted object is now referred to as the ciphertext \mMactwo. 

For the outer encryption object, we consider the plaintext formed by concatenating the bitstrings corresponding to the identifier for \mCredr, \mMactwo, and \mADtwo (if any). The ciphertext is obtained by performing an \mXor{} operation on this plaintext with the key \mKtwoe, which is obtained by using the KDF on \mPRKtwo and the transaction hash \mTHtwo. The responder therefore sends to the initiator \mGy, \mCr, and this ciphertext as the second message.

First the initiator generates \mPRKtwo, using the same arguments to the KDF as the responder did above. For \mPRKthree, however, the initiator sends to the KDF \mPRKtwo and \mGrx -- this \mGrx is obtained by exponentiating \mCredr, the public authentication key of the responder, to the initiator's ephemeral secret $x$. Note that this will be used to generate a key to decrypt the inner \mMactwo sent by the responder, and therefore is asymmetric in nature to the \mGrx used by the responder above. \mPRKfour is set equal to \mPRKthree. 

It then sends the third message, consisting of \mCr, and an encrypted object. Here, again, we first build the inner \mCose{} object. This process is analogous to that employed by the responder for the second message. The \mCose{} object has as protected data an identifier for retrieving the initiator's public authentication key \mCredi. The externally supplied data is the transaction hash \mTHthree of the second message and \mTHtwo,  the initator's public authentication key \mCredi, and (optional) auxiliary data \mADthree. The key used is \mKthreem, obtained by inputing the pseudorandom string \mPRKthree and \mTHthree. The resulting encrypted object is now referred to as the ciphertext \mMacthree. 

Since the initiator is running \mSig, this \mCose{} object needs to be signed. To the signing algorithm, the initiator sends as protected data the identifier for retrieving \mCredi, as external data the concatenation of \mTHthree, \mCredi, and any \mADthree, and the payload \mMacthree. This is signed using the private authentication key of the initiator.  

We now construct the encryption for the third message, by passing to an \mAead{} encryption algorithm a \mCose{} object with no protected data, external data \mTHthree, and a plaintext obtained by concatenating the identifier for retrieving \mCredi, the signed object described above, and \mADthree. The key used for encryption is \mKthreeae, obtained by running the KDF on the pseudorandom string \mPRKthree and \mTHtwo. Thus, the message sent by the initiator to the responder in the third step is \mCr accompanied by this encrypted object. This method is illustrated in Figure~\ref{fig:edhocsigstat}.

\begin{figure}[!h]
\centering
%\includegraphics[scale=0.3]{Images/asym.png}
\scalebox{.7}{
\tikzset{>=latex, every msg/.style={draw=thick}, every node/.style={fill=white,text=black}}
\begin{tikzpicture}
    \node (ini) at (0, 0) {Initiator};
    \draw [very thick] (0, -0.5) -- (0,-15);
    \draw [very thick] (9, -0.5) -- (9,-15);
    \node[below of=ini,fill=white,text=black] {$
    \begin{array}{c}
    \text{Knows}\ $g$,\ \mCredi,\ \mLtki,\ \mIdcredi,\\
    \mIdcredr, \mADone,\ \mADthree
    \end{array}
    $};
    \node (res) at (9,0) {Responder};
    \node[below of=res] {$
    \begin{array}{c}
    \text{Knows}\ $g$,\ \mCredr,\ \mLtkr, \ \mIdcredr,\\
    \mIdcredi, \mADtwo
    \end{array}$};
    \action{5em}{ini}{Generates \mMethod,\ \mSuites,\ \mCi,\ $x$\\$\mGx = g^{x}$};
    \msg{10em}{ini}{res}{\mMsgone: \mMethod, \mSuites, \mGx, \mCi, \mADone};
    \action{11em}{res}{$
      \begin{array}{c}
        \text{Generates } \mCr,\ $y$\\
        \mGy = g^{y}\\
        \mTHtwo = \mHash(\mMsgone, \langle \mCi, \mGy, \mCr \rangle)\\
        \mPRKtwo = \mHkdfExtract(\textrm{``\phantom{}''}, g^{xy}) \\
        \mGrx = \mGx^{\mLtkr} \\
        \mPRKthree = \mHkdfExtract(\mPRKtwo, \mGrx) \\
        \mKtwom = \mHkdfExpand(\mPRKthree, \mTHtwo) \\
        \mMactwo = \mAead(\mKtwom; \langle \mIdcredr, \mTHtwo, \mCredr, \mADtwo \rangle; \textrm{``\phantom{}''}) \\
        \mKtwoe = \mHkdfExpand(\mPRKtwo, \mTHtwo)
      \end{array}$};
    \msg{26em}{res}{ini}{\mMsgtwo: \mCi, \mGy, \mCr, $\overbrace{\mKtwoe\ \mXor\ \langle \mIdcredr, \mMactwo, \mADtwo \rangle}^{\mCipher}$};
    \action{27em}{ini}{$
      \begin{array}{c}
        %\mTHtwo = \mHash(\mMsgone, \langle \mCi, \mGy, \mCr \rangle) \
        \mPRKtwo = \mHkdfExtract(\textrm{``\phantom{}''}, g^{xy}) \\
        %\mKtwoe = \mHkdfExpand(\mPRKtwo,\mTHtwo)\\
        \mGrx = \mCredr^{x} \\
        \mPRKfour = \mPRKthree = \mHkdfExtract(\mPRKtwo, \mGrx) \\
        %\mKtwom = \mHkdfExpand(\mPRKthree, \mTHtwo) \\
        \mKthreeae = \mHkdfExpand(\mPRKthree, \mTHtwo) \\
        \mTHthree = \mHash(\mTHtwo, \mCipher, \mCr)\\
        \mKthreem = \mHkdfExpand(\mPRKfour, \mTHthree) \\
        \mMacthree = \mAead(\mKthreem; \langle \mIdcredi, \mTHthree, \mCredi, \mADthree \rangle; \textrm{``\phantom{}''}) \\
        \mSigthree = \mSign(\mLtki; \langle \mIdcredi, \mTHthree, \mCredi, \mADthree \rangle, \mMacthree \rangle)
      \end{array}$};
    \msg{39em}{ini}{res}{$\mMsgthree: \mCr, \mAead(\mKthreeae; \mTHthree; \langle \mIdcredi, \mSigthree, \mADthree \rangle$)};
    \action{40em}{res}{$
    \begin{array}{c}
        \mTHthree = \mHash(\mTHtwo, \mCipher, \mCr)\\
        \mKthreem = \mHkdfExpand(\mPRKthree, \mTHthree) \\
        \mKthreeae = \mHkdfExpand(\mPRKthree, \mTHthree)
    \end{array}$};
    \draw [line width=2mm] (-2,-15) -- (2,-15);
    \draw [line width=2mm] (7,-15) -- (11,-15);
    \end{tikzpicture}}
\caption{The \mSigStat{} method of \mEdhoc. \mCredi and \mLtki, and \mCredr and \mLtkr are two public-private key pairs. \mCredi and \mLtki must be signature keys in the \mSig{} method. $\mSign(x; y)$ is used to denote the signing of message $y$ using key $x$.}
\label{fig:edhocsigstat}
\end{figure}

\subsubsection{\mStatStat}
In this method, both the initiator and the responder run the \mStat{} method. The responder's message looks exactly the same as in the previous subsection, for \mSigStat. The initiator's message, however, need not be signed anymore, since the initiator too is running the \mStat{} method. Thus, we skip the signature process in the steps described above, and instead of creating and passing \mSigthree to the \mAead{} encryption algorithm, the initiator passes \mMacthree itself. 

As concerns the pseudorandom strings, \mPRKtwo and \mPRKthree are generated exactly as in the \mSigStat method. However, now, since the initiator also runs a \mStat method, \mPRKfour is not equal to \mPRKthree as earlier, but computed by passing to the KDF the following parameters -- as salt, \mPRKthree and as IKM, \mGiy. \mGiy is obtained by exponentiating \mGy (received from the responder in the second message) to the initiator's private authentication key. The encryption and decryption keys are computed based on these values of the pseudorandom strings. All messages are constructed using these values of the pseudorandom strings and keys.

\subsubsection{\mStatSig}
Here, the responder runs the \mSig{} method, while the initiator runs the \mStat{} method. Thus, the second message, sent by the responder to the initiator needs an extra layer of signing. 

The pseudorandom strings are now generated in a slightly different manner, since the responder is running \mSig. \mPRKtwo is still generated by running the KDF on the empty salt and $g^{xy}$. However, \mPRKthree is now equal to \mPRKtwo. \mKtwom and \mKtwoe follow the same guidelines as earlier, but with these values of \mPRKtwo and \mPRKthree. The responder constructs \mMactwo as earlier.

Once \mMactwo has been constructed, the responder runs the signature algorithm with a \mCose{} object. This object has \mIdcredr as protected data, a concatenation of \mTHtwo, \mCredr, and \mADtwo as external data, and the payload \mMactwo. This is signed using the responder's private authentication key to obtain \mSigtwo.

The outer encryption object is constructed by considering a plaintext consisting of \mCredr, \mSigtwo (instead of \mMactwo), and \mADtwo. The key \mKtwoae is \mXor-ed with this plaintext to get a ciphertext, and the responder sends \mGy, \mCr, and this ciphertext to the initiator as the second message.

The initiator generates the pseudorandom string \mPRKtwo by running the KDF on empty salt and $g^{xy}$, and sets \mPRKthree equal to \mPRKtwo. These are used for generating the \mKtwom and \mKtwoe for decrypting the message received from the responder. The pseudorandom string \mPRKfour is generated by giving as input to \mHkdfExtract{} \mPRKthree as salt, and \mGiy as IKM -- \mGiy is obtained as in the \mStatStat method.

Now, to generate the encryption keys, the initiator generates \mKthreem by passing \mPRKfour and \mTHthree to the KDF. \mKthreeae is generated by running the KDF on \mPRKthree and \mTHtwo. There is no signature, and the message is constructed exactly as in the \mStatStat case, except with these values of \mKthreem and \mKthreeae.

In order to decrypt the message received from the initiator, the responder needs to generate keys. \mKthreeae is straightforward, but in order to generate \mKthreem, the responder needs \mPRKfour. To generate this pseudorandom string, the responder runs \mHkdfExtract{} on \mPRKthree as salt, and uses as IKM \mGiy, which is obtained by exponentiating the initiator's public authentication key to the responder's ephemeral secret $y$. As earlier, this is an asymmetric key used for ``decrypting'' the \mGiy used by the initiator to construct its \mPRKfour. This method is illustrated in Figure~\ref{fig:edhocstatsig}.

\begin{figure}[!h]
\centering
%\includegraphics[scale=0.3]{Images/asym.png}
\scalebox{.7}{
\tikzset{>=latex, every msg/.style={draw=thick}, every node/.style={fill=white,text=black}}
\begin{tikzpicture}
    \node (ini) at (0, 0) {Initiator};
    \draw [very thick] (0, -0.5) -- (0,-15.2);
    \draw [very thick] (9, -0.5) -- (9,-15.2);
    \node[below of=ini,fill=white,text=black] {$
    \begin{array}{c}
    \text{Knows}\ $g$,\ \mCredi,\ \mLtki,\ \mIdcredi,\\
    \mIdcredr, \mADone,\ \mADthree
    \end{array}
    $};
    \node (res) at (9,0) {Responder};
    \node[below of=res] {$
    \begin{array}{c}
    \text{Knows}\ $g$,\ \mCredr,\ \mLtkr,\ \mIdcredr,\\
    \mIdcredi, \mADtwo
    \end{array}$};
    \action{5em}{ini}{Generates \mMethod,\ \mSuites,\ \mCi,\ $x$\\$\mGx = g^{x}$};
    \msg{10em}{ini}{res}{\mMsgone: \mMethod, \mSuites, \mGx, \mCi, \mADone};
    \action{11em}{res}{$
      \begin{array}{c}
        \text{Generates } \mCr,\ $y$\\
        \mGy = g^{y}\\
        \mTHtwo = \mHash(\mMsgone, \langle \mCi, \mGy, \mCr \rangle)\\
        \mPRKthree = \mPRKtwo = \mHkdfExtract(\textrm{``\phantom{}''}, g^{xy}) \\
        \mKtwom = \mHkdfExpand(\mPRKthree, \mTHtwo) \\
        \mMactwo = \mAead(\mKtwom; \langle \mIdcredr, \mTHtwo, \mCredr, \mADtwo \rangle; \textrm{``\phantom{}''}) \\
        \mSigtwo = \mSign(\mLtkr; \langle \langle \mIdcredr, \mTHtwo, \mCredr, \mADtwo \rangle, \mMactwo \rangle)\\
        \mKtwoe = \mHkdfExpand(\mPRKtwo, \mTHtwo)
      \end{array}$};
    \msg{25em}{res}{ini}{\mMsgtwo: \mCi, \mGy, \mCr, $\overbrace{\mKtwoe\ \mXor\ \langle \mIdcredr, \mSigtwo, \mADtwo \rangle}^{\mCipher}$};
    \action{26em}{ini}{$
      \begin{array}{c}
        \mPRKthree = \mPRKtwo = \mHkdfExtract(\textrm{``\phantom{}''}, g^{xy}) \\
        \mGiy = \mGy^{\mLtki} \\
        \mPRKfour = \mHkdfExtract(\mPRKthree, \mGiy) \\
        \mTHthree = \mHash(\mTHtwo, \mCipher, \mCr)\\
        \mKthreem = \mHkdfExpand(\mPRKfour, \mTHthree) \\
        \mMacthree = \mAead(\mKthreem; \langle \mIdcredi, \mTHthree, \mCredi, \mADthree \rangle; \textrm{``\phantom{}''}) \\
        \mKthreeae = \mHkdfExpand(\mPRKthree, \mTHtwo) \\
      \end{array}$};
    \msg{37em}{ini}{res}{$\mMsgthree: \mCr, \mAead(\mKthreeae; \mTHthree; \langle \mIdcredi, \mMacthree, \mADthree \rangle$)};
    \action{38em}{res}{$
    \begin{array}{c}
       \mGiy = \mCredi^{y} \\
       \mPRKfour = \mHkdfExtract(\mPRKthree, \mGiy) \\
       \mTHthree = \mHash(\mTHtwo, \mCipher, \mCr)\\
        \mKthreem = \mHkdfExpand(\mPRKfour, \mTHthree) \\
        \mKthreeae = \mHkdfExpand(\mPRKthree, \mTHtwo)
    \end{array}$};
    \draw [line width=2mm] (-2,-15.2) -- (2,-15.2);
    \draw [line width=2mm] (7,-15.2) -- (11,-15.2);
    \end{tikzpicture}}
\caption{The \mStatSig{} method of \mEdhoc}
\label{fig:edhocstatsig}
\end{figure}

\subsection{\mSigSig{} method}
In this method, both parties run the \mSig{} method, and therefore, both the second and third messages need to be signed before being encrypted via \mAead. The pseudorandom string \mPRKtwo is generated as usual, by sending the empty salt and the shared secret to the KDF, and both \mPRKthree and \mPRKfour are set equal to \mPRKtwo. The second message looks like the one from the \mSigStat{} method described above, while the third looks like the one from the \mStatSig{} method.

\subsection{Negotiating a cipher suite and method and correlation parameters}
\label{sec:ciphersuite}
Recall that we mentioned that the first message contains a list of cipher suites, ranked according to the preference of the initiator. What does a cipher suite actually contain? An \mEdhoc{} cipher suite consists of an ordered set of \mCose{} algorithms: an \mAead{} algorithm, a hash algorithm, an ECDH curve, a signature algorithm, a signature algorithm curve, an application \mAead{} algorithm, and an application hash algorithm from the \mCose{} Algorithms and Elliptic Curves registries.  

There are four supported cipher suites in \mEdhoc -- we refer the reader to Section 3.4 of~\cite{selander-lake-edhoc-01} for the specifics of the algorithms allowed therein. Each cipher suite is identified by one of four predefined integer labels (0--3). Some algorithms are not used in some methods.  The signature algorithm and the signature algorithm curve are not used in methods without signature authentication (i.e. in \mPskPsk{} and \mStatStat).

In order to keep the presentation clean, we have omitted the cipher suite negotiation process from the description of the methods. However, this process happens as follows, at the beginning of every method, once the responder receives the first message. The initiator proposes an ordered list of cipher suites they support. This list presented in descending order to the responder who either accepts the topmost entry in this list (if they also support that suite) or makes a counter-proposal, namely the topmost entry which they support from the remaining part of the list. If there is no such entry the responder can reject, and the protocol does not continue. Similarly, the responder can reject the initiator's choices for the method and correlation parameters as well -- in the case of a reject for either of these values, the protocol aborts.


\subsection{Deriving an OSCORE context}
\mEdhoc{} is often used to set up parameters for \mOscore. In this case, the parties make sure that the connection identifiers are distinct, i.e. $\mCredi \neq \mCredr$, since these are used as \mOscore{} sender IDs. If the initiator plays the role of the CoAP client, and the responder the role of the CoAP server, the client gets the sender ID \mCredr and the server the ID \mCredi (the identifiers are swapped). The \mAead{} and hash algorithms for \mOscore{} stay the same as those used for the selected cipher suite in \mEdhoc, while the master secret for \mOscore{} is derived using the key length of the \mAead{} algorithm of \mEdhoc. 

\subsection{Expected security properties}
In this section, we list all the claims made by the authors of~\cite{selander-lake-edhoc-01} regarding the security properties satisfied by \mEdhoc. We will revisit these claims when we discuss the formal modelling and verification of \mEdhoc{} in Section~\ref{sec:formalization}. 

\knote{We should link forward to the formalization section where we describe
    exactly what we have verified and what not. We should also link forward to
    the discussion section where we "analyze" these claims from a
    standardization process view point.
}
\vnote{I've added two sentences about this at the end. Do you need me to say more?}

The following are the claims made by the authors of~\cite{selander-lake-edhoc-01}. 

\mEdhoc{} inherits some security properties from the \mSigma{} protocol. These are perfect forward secrecy, mutual authentication with aliveness, consistency, peer awareness (to the responder, but not to the initiator), identity protection, and Key Compromise Impersonation (KCI) resistance.

All methods other than \mPskPsk{} offer identity protection of the initiator against active attacks and that of the responder against passive attacks. The roles should be assigned to protect the more sensitive identity. This is usually the entity whose identity cannot be inferred from information in the lower layers.

\mEdhoc{} also provides protection against replay attacks by the attacker. The attacker also cannot affect any negotiated parameters. A single session of \mEdhoc{} enables the responder to verify that the selected cipher suite is the most preferred of the initiator which is supported by both parties, even though there is no negotiation of cipher suites per se.

In order to reduce the chances of pervasive monitoring, \mEdhoc{} only supports methods with perfect forward secrecy. One way to limit the effect of breaches is to minimize the use of symmetrical group keys for bootstrapping. \mEdhoc, therefore, uses raw public keys and self-signed certificates instead of symmetrical group keys for bootstrapping.

For the \mPskPsk{} method, compromising \mPsk{} lets the attacker impersonate either party in \mEdhoc{} exchanges with the other. For the other methods, compromising the private authentication keys of one party lets the attacker impersonate only the compromised party in exchanges with other parties. In particular, it does not let the attacker masquerade as any other parties in communications with the compromised party. 

Compromise of the \mHkdf{} input parameters (ECDH shared secret and/or \mPsk) leads to all session keys derived from that shared secret being deemed compromised. However, the compromise of one session key does not affect other session keys. If the long-term keys (\mPsk{} or private authentication keys) are compromised, this does not affect the security of instances of \mEdhoc{} which have completed prior to compromise. 

\vedit{}{In this paper, we verify secrecy, authentication, session independence, perfect forward secrecy, key-compromise impersonation, and some flavour of post-compromise security. We will also discuss, in Section~\ref{sec:discussion} what it means for these properties to hold about this model.}
