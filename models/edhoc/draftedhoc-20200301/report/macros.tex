\usepackage{ifthen}

% -- Editorial macros
% http://tug.ctan.org/tex-archkeIVe/macros/latex/contrib/todonotes/todonotes.pdf
\usepackage{todonotes}
\newcommand{\knote}[1]{\todo[inline,
                             color=orange,
                             size=\footnotesize
                            ]{\textsf{\textbf{Karl:} #1}}}
\newcommand{\anote}[1]{\todo[inline,
                             color=orange,
                             size=\footnotesize
                            ]{\textsf{\textbf{Alessandro:} #1}}}
\newcommand{\vnote}[1]{\todo[inline,
                             color=orange,
                             size=\footnotesize
                            ]{\textsf{\textbf{Vaishnavi:} #1}}}
\newcommand{\vedit}[2]{{\color{Blue3}\st{#1} #2}}
\newcommand{\pskstuff}[1]{{\color{violet}\textbf{#1}}}

% \newcommand{\kedit}[2]{{\color{teal}\st{#1}#2}}
% \renewcommand{\knote}[1]{}
% \renewcommand{\vnote}[1]{}
% \renewcommand{\anote}[1]{}
\newcommand{\mPoint}[1]{{\color{red}\textbf{Point: }#1}}
\newcommand{\mcneed}{\textbf{Citation needed}}
\newcommand{\mcfix}{\textbf{Fix citation}}

\definecolor{cbnavy}{RGB}{15, 32, 128}
\definecolor{cborange}{RGB}{245, 121, 58}
\definecolor{cbsky}{RGB}{133, 192, 249}

\newcommand{\runhead}[1]{\noindent\textbf{#1. }}

% -- Fonts and styles for names
%\newcommand{\mItemStyle}[1]{\ensuremath{#1}}
%\newcommand{\mSetStyle}[1]{\ensuremath{\mathrm{\mathbf{#1}}}}
%\newcommand{\mFunStyle}[1]{\text{\textup{\textsf{#1}}}}
%\newcommand{\mConstStyle}[1]{\text{\textup{\textsf{#1}}}}
%\newcommand{\mVarStyle}[1]{\mathit{#1}}
%\newcommand{\mFactStyle}[1]{\text{\textsf{#1}}}
%\newcommand{\mMethodStyle}[1]{\mbox{\mConstStyle{#1}}}
%\newcommand{\mProtocolStyle}[1]{\mbox{\textrm{#1}}}

%\newcommand{\mItemStyle}[1]{\ensuremath{#1}}
%\newcommand{\mSetStyle}[1]{\ensuremath{\mathrm{\mathbf{#1}}}}
\newcommand{\mFunStyle}[1]{\textsf{#1}}
\newcommand{\mConstStyle}[1]{\textsf{#1}}
\newcommand{\mVarStyle}[1]{\mathit{#1}}
\newcommand{\mFactStyle}[1]{\textsf{#1}}
\newcommand{\mMethodStyle}[1]{\mConstStyle{#1}}
\newcommand{\mProtocolStyle}[1]{\text{#1}}



% -- Macros for consistent wording
\newcommand{\mSpec}{specification}  % The EDHOC spec document we are analyzing



% -- Formalization notation
\newcommand{\mRevLTK}{\ensuremath{\mathbf{A}_\mFunStyle{LTK}}}
\newcommand{\mTEE}{\ensuremath{\mathbf{A}_\mFunStyle{TEE}}}
\newcommand{\mRevEph}{\ensuremath{\mathbf{A}_\mFunStyle{Eph}}}
\newcommand{\mIStart}{\ensuremath{\mathbf{I}_\mFunStyle{S}}}
\newcommand{\mIComplete}{\ensuremath{\mathbf{I}_\mFunStyle{C}}}
\newcommand{\mRStart}{\ensuremath{\mathbf{R}_\mFunStyle{S}}}
\newcommand{\mRComplete}{\ensuremath{\mathbf{R}_\mFunStyle{C}}}
\newcommand{\mPredPfs}{\ensuremath{\mathbf{PFS}}}
\newcommand{\mPredInjI}{\ensuremath{\mathbf{InjAgree}_I}}
\newcommand{\mPredInjR}{\ensuremath{\mathbf{InjAgree}_R}}
\newcommand{\mPredImpI}{\ensuremath{\mathbf{ImpAgree}_I}}
\newcommand{\mPredImpR}{\ensuremath{\mathbf{ImpAgree}_R}}
\newcommand{\mK}{\ensuremath{\mathcal{K}}}
\DeclareMathOperator{\mLogicDot}{.}


% -- Domain specific macros
\newcommand{\mArxiv}{\texttt{arXiv}}
\newcommand{\mTamarin}{\mProtocolStyle{Tamarin}}
\newcommand{\mProverif}{\mProtocolStyle{ProVerif}}
\newcommand{\mEdhoc}{\mProtocolStyle{EDHOC}}
\newcommand{\mOscore}{\mProtocolStyle{OSCORE}}
\newcommand{\mSigma}{\mProtocolStyle{SIGMA}}
\newcommand{\mSigmaI}{\mProtocolStyle{SIGMA\nobreakdash-I}}
\newcommand{\mCbor}{\mProtocolStyle{CBOR}}
\newcommand{\mCose}{\mProtocolStyle{COSE}}
\newcommand{\mCoseEncrypt}{\mProtocolStyle{COSE\_Encrypt0}}
\newcommand{\mCoseSign}{\mProtocolStyle{Cose\_Sign1}}
\newcommand{\mHkdf}{\mProtocolStyle{HKDF}}
\newcommand{\mHkdfExtract}{\mProtocolStyle{HKDF\nobreakdash-extract}}
\newcommand{\mHkdfExpand}{\mProtocolStyle{HKDF\nobreakdash-expand}}
\newcommand{\mHmac}{\mProtocolStyle{HMAC}}
\newcommand{\mAead}{\mProtocolStyle{AEAD}}
\newcommand{\mAeadDecrypt}{\mProtocolStyle{AEAD\nobreakdash-decrypt}}
\newcommand{\mDecrypt}{\mProtocolStyle{decrypt}}
\newcommand{\mOptls}{\mProtocolStyle{OPTLS}}
\newcommand{\mNoise}{\mProtocolStyle{Noise}}
\newcommand{\mTls}{\mProtocolStyle{TLS}}
\newcommand{\mDandTls}{\mProtocolStyle{(D)TLS}}
\newcommand{\mCtls}{\mProtocolStyle{cTLS}}
\newcommand{\mCoap}{\mProtocolStyle{CoAP}}

\newcommand{\mStat}{\mMethodStyle{STAT}}
\newcommand{\mSig}{\mMethodStyle{SIG}}
\newcommand{\mPsk}{\mMethodStyle{PSK}}
\newcommand{\mStatStat}{\mMethodStyle{STAT-STAT}}
\newcommand{\mStatSig}{\mMethodStyle{STAT-SIG}}
\newcommand{\mSigStat}{\mMethodStyle{SIG-STAT}}
\newcommand{\mSigSig}{\mMethodStyle{SIG-SIG}}
\newcommand{\mPskPsk}{\mMethodStyle{PSK-PSK}}

\newcommand{\mSid}{\mConstStyle{sid}}    % session id = (u, v, s-key)

\newcommand{\mXor}{\mConstStyle{XOR}}
\newcommand{\mSuites}{\mConstStyle{Suites\_I}}
\newcommand{\mMethod}{\mConstStyle{Method}}
\newcommand{\mCi}{\mConstStyle{C\_I}}
\newcommand{\mCr}{\mConstStyle{C\_R}}
\newcommand{\mGi}{\mConstStyle{G\_I}}
\newcommand{\mGr}{\mConstStyle{G\_R}}
\newcommand{\mGiy}{\mConstStyle{G\_IY}}
\newcommand{\mGrx}{\mConstStyle{G\_RX}}
\newcommand{\mGx}{\mConstStyle{G\_X}}
\newcommand{\mGy}{\mConstStyle{G\_Y}}
\newcommand{\mGxy}{\mConstStyle{G\_XY}}
\newcommand{\mIDPsk}{\mConstStyle{ID\_PSK}}
\newcommand{\mTH}{\mConstStyle{TH}}
\newcommand{\mTHtwo}{\mConstStyle{TH\_2}}
\newcommand{\mKtwoe}{\mConstStyle{K\_2e}}
\newcommand{\mKtwom}{\mConstStyle{K\_2m}}
\newcommand{\mKtwoae}{\mConstStyle{K\_2ae}}
\newcommand{\mSign}{\mConstStyle{sign}}

\newcommand{\mKthreeae}{\mConstStyle{K\_3ae}}
\newcommand{\mKthreem}{\mConstStyle{K\_3m}}

\newcommand{\mTHthree}{\mConstStyle{TH\_3}}
\newcommand{\mhplain}{\mConstStyle{h''}}
\newcommand{\mCredi}{\mConstStyle{CRED\_I}}
\newcommand{\mCredr}{\mConstStyle{CRED\_R}}
\newcommand{\mHash}{\mConstStyle{H}}

\newcommand{\mTHfour}{\mConstStyle{TH\_4}}
\newcommand{\mAuthi}{\mConstStyle{Auth\_I}}
\newcommand{\mAuthr}{\mConstStyle{Auth\_R}}

\newcommand{\mMactwo}{\mConstStyle{MAC\_2}}
\newcommand{\mMacthree}{\mConstStyle{MAC\_3}}

\newcommand{\mSigtwo}{\mConstStyle{Sig\_2}}
\newcommand{\mSigthree}{\mConstStyle{Sig\_3}}

\newcommand{\mMsgone}{\mConstStyle{m1}}
\newcommand{\mMsgtwo}{\mConstStyle{m2}}
\newcommand{\mMsgthree}{\mConstStyle{m3}}

\newcommand{\mCipher}{\mConstStyle{cipher\_2}}

\newcommand{\mAD}{\mConstStyle{AD}}
\newcommand{\mADone}{\mConstStyle{AD\_1}}
\newcommand{\mADtwo}{\mConstStyle{AD\_2}}
\newcommand{\mADthree}{\mConstStyle{AD\_3}}

\newcommand{\mPRK}{\mConstStyle{PRK}}
\newcommand{\mPRKtwo}{\mConstStyle{PRK\_2e}}
\newcommand{\mPRKthree}{\mConstStyle{PRK\_3e2m}}
\newcommand{\mPRKfour}{\mConstStyle{PRK\_4x3m}}

\newcommand{\mIdcredi}{\mConstStyle{ID\_CRED\_I}}
\newcommand{\mIdcredr}{\mConstStyle{ID\_CRED\_R}}
\newcommand{\mLtki}{\mConstStyle{ltk\_I}}
\newcommand{\mLtkr}{\mConstStyle{ltk\_R}}
\newcommand{\mLtk}{\mConstStyle{ltk}}

\newcommand{\cm}{\checkmark}

% TIKZ messages and actions
\newcommand{\msg}[4]{\draw[->,thick] ([yshift=-#1]#2.south) coordinate (l1)--(l1-|#3) node[midway, above]{#4}}
\newcommand{\action}[3]{\node[draw,thick,fill=white,align=center,below={#1} of {#2}]{#3}}

% Tamarin symbols
\newcommand{\ifarrow}[1][]{\ifthenelse{\equal{#1}{}}{\rightarrow}{-\hspace{-5.2pt}[{#1}]\hspace{-7pt}\rightarrow}}
\newcommand{\semarrow}[1][]{\ifthenelse{\equal{#1}{}}{\Rightarrow}{=\hspace{-3pt}[{#1}]\hspace{-3pt}\Rightarrow}}
\newcommand{\mIn}{\mathsf{In}}
\newcommand{\mOut}{\mathsf{Out}}
\newcommand{\mFr}{\mathsf{Fr}}
\newcommand{\mKD}{\mathsf{KD}}
\newcommand{\mKU}{\mathsf{KU}}
\newcommand{\mT}[1]{\lstinline[basicstyle=\ttfamily\normalsize]{#1}} % Tamarin code inline
\usepackage[final]{listings}


\lstdefinestyle{mystyle}{
    backgroundcolor=,
    commentstyle=\color{Green},
    identifierstyle=\color{black},
    keywordstyle=\color{ForestGreen},
    numberstyle=\color{Gray},
    stringstyle=,
    basicstyle=\ttfamily\small,
    %basicstyle=\sffamily\small,
    keywords={let,in,rule,restriction,axiom,lemma,all-traces,exists-trace,Ex,All,Fr,In,Out},
    breakatwhitespace=false,
    breaklines=true,
    captionpos=b,
    keepspaces=false,
    numbers=left,
    numbersep=5pt,
    showspaces=false,
    showstringspaces=false,
    showtabs=false,
    tabsize=2,
    morecomment=[l]{//},
    literate=
    {=}{{$=$} }{1}
    {[}{{$[$}}{1}%
    {]}{{$]$}}{1}%
    {<}{{$\langle$}}{1}%
    {>}{{$\rangle$}}{1}%
    %{(}{{$($}}{1}%
    %{)}{{$)$}}{1}%
    {&}{{$\wedge$}}{1}
    {|}{{$\vee$}}{1}
    {$}{\$}{1}
    {<\ \#}{{$<$\ \ \#}}{3}% trick here: we distinguish angle brackets from temporal comparisons because of the # for the variable on the rhs
    {--[}{$\mbox{-\hspace{-4.0pt}[}$}{3}%
    {]->}{$\mbox{]\hspace{-4.2pt}\rightarrow}$}{3}%
    {-->}{$\rightarrow$}{1}
    {==>}{$\Rightarrow$}{1}
    {XOR}{$\oplus$}{1}%
    {tid}{{tid}}{2}%
    {ltk}{{ltk}}{2}%
    {agreed}{{agreed}}{5}%
    %{assocData2}{{assocData2}}{9}%
    %{protected2}{{protected2}}{10}%
    %{protected2}{protected2}{7}%
    {extAad2}{{extAad2}}{6}%
    %{aeadEncrypt}{\mAead{}}{6}%
    %{decrypt}{\mDecrypt}{6}
    %{aeadDecrypt}{\mAeadDecrypt}{11}
    %{hkdfExtract}{\mHkdfExtract}{14}
    %{hkdfExpand}{\mHkdfExpand}{14}
    {All\ }{$\forall$\ }{3}
    {Ex\ }{$\exists$\ }{3}
    {h(}{h$($}{2}
    {~}{{\url{~}}}{1}
    }
\lstset{style=mystyle}
