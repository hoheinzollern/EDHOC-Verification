% !TEX root =  main.tex

To establish an authenticated security context for \mOscore,
\mEdhoc{} uses a three-message protocol structure
shown in Figure~\ref{fig:edhocFramework}.
%
In the first message \mMsgone{}, the initiator (\mT{I}) sends to the responder (\mT{R}) an
ephemeral DH half-key \mGx{}, an ordered list of supported cipher suites, and, in the \mMethod{} element, suggests
authentication methods to be used.
%
\mT{R} may reject the choice of cipher suite, leading to a negotiation across multiple \mEdhoc{} sessions (see Section~\ref{sec:ciphersuiteNegotiation}), or that of the authentication methods.
%
In the second message \mMsgtwo{}, \mT{R} provides its ephemeral
DH half-key \mGy{} and authenticates itself to \mT{I} by
providing authentication data \mAuthr{}.
%
Finally, in the third message \mMsgthree{}, \mT{I} authenticates itself
to \mT{R} by providing authentication data \mAuthi{}.
%
The authentication data consists of a MAC for all authentication methods.
%
This MAC is also signed when authentication is based on public key signatures.
%
In addition to the exchange of DH half-keys and authentication data, the
messages also include parameters for the \mOscore{} security context, such as
encryption algorithm identifiers (\mSuites{}) and \mOscore{} security context
identifiers \mCi{} and \mCr{}.
%
Optionally, messages may also include application layer data \mAD{}, which
enjoys differing levels of protection depending on which message it is part of.
%
We do not include \mCi, \mCr, and \mAD{} in our modelling here.
%
When authentication is based on
a pre-shared key, the \mCredi{} and \mCredr{} credential identifiers are not
used.
%
Instead, a single identifier for the pre-shared key is transmitted in \mMsgone{}.
%
\begin{figure}
\centering
\tikzset{>=latex, every msg/.style={draw=thick}, every node/.style={fill=none,text=black}}
\begin{tikzpicture}
    \node (ini) at (0, 0) {Initiator (\mT{I})};
    \draw [very thick] (0, -0.25) -- (0,-2.2);
    \draw [very thick] (7, -0.25) -- (7,-2.2);
    \node (res) at (7,0) {Responder (\mT{R})};
    \msg{1em}{ini}{res}{\mMsgone: \mMethod, \mSuites, \mGx, \mCi, \mADone};
\msg{3em}{res}{ini}{\mMsgtwo: \mCi, \mGy, \mCr, [\mIdcredr, \mAuthr, \mADtwo]};
    \msg{5em}{ini}{res}{\mMsgthree: \mCr, [\mIdcredi, \mAuthi, \mADthree]};
    \draw [line width=1mm] (-0.75,-2.2) -- (0.75,-2.2);
    \draw [line width=1mm] (7-0.75,-2.2) -- (7+0.75,-2.2);
    \end{tikzpicture}
\caption{Structure of \mEdhoc{}: $[t]$ means $t$ is encrypted and integrity protected}
\label{fig:edhocFramework}
\end{figure}

\knote{note that the version we look at has private keys, but the current spec
    in IETF does not. OTOH, they aim to add that later, and I can't see how they
    could avoid doing so. That is why I write "aims to..."
}
\mEdhoc{} aims to accommodate three credential types
a party may use for authentication -- certificates, private keys
and static long-term DH keys.
%
Depending on the types used, the authentication methods differ,
and \mAuthi{} and \mAuthr{} need to be constructed accordingly.
%
The three authentication methods are based on digital signatures (\mSig),
challenge-response signatures based on static long-term DH keys (\mStat) and
pre-shared symmetric keys (\mPsk).
%
The two parties may use different authentication methods in an \mEdhoc{} run.
%
While any combination of the \mSig{} and \mStat{} methods is possible,
when \mPsk{} is used, both parties must use \mPsk{}.
%
We refer to these \emph{combinations} of authentication methods as
\emph{methods} to follow the terminology in the
\mSpec{}~\cite{selander-lake-edhoc-01}.
%
Thus, we have five methods: \mSigSig, \mSigStat, \mStatStat, \mStatSig{} and
\mPskPsk.
%
The first method in a name denotes the
authentication method used by the initiator of the protocol run and the second denotes the
one used by the responder.
%
We refer to a method where at least one party uses \mSig{} as a \mSig-based
method, and similarly for \mStat{} and \mPsk.
%

We first briefly discuss \mEdhoc's relations to the protocols on which it is
based before providing details about the methods.
%

%-----------------------------------------------------------------------------
\subsection{Relation to \mSigma, \mOptls{}, and \mNoise{}}
\label{sec:relationsToOtherProtocols}
We now present how \mEdhoc{} uses \mSigma, \mOptls{}, and \mNoise{} as
cryptographic cores for various methods, and the ways in which it differs from them.
\\
%
Being designed as an industrial standard, \mEdhoc{} specifies more details
on top of these cores, e.g., connection identifiers for message multiplexing,
encoding and application specific parameters to be negotiated.
%
\\

%-----------------------------------------------------------------------------
\runhead{\mSigma{}}
%\subsubsection{\mSigma{}.}
\label{sec:sigma}
The \mSigSig{} method of \mEdhoc{} is closely modeled on the \mSigma{}
variant that provides identity protection for the initiator~\cite{sigma}.
%
Some aspects differ though, e.g., \mSigSig{} uses Mac-then-Sign like
\mTls{} instead of Sign-then-Mac. 
%
\\

%----------------------------------------------------------------------------- 
%\subsubsection{\mOptls{}.}
\runhead{\mOptls{}}
\label{sec:optls}
The \mStat-based methods use challenge-response signatures based on static
long-term DH keys, and proceed along the lines of
\mOptls~\cite{DBLP:conf/eurosp/KrawczykW16}.
%
The challenge-response works as follows.
Party $A$ sends a random challenge $r$
and an ephemeral DH half-key $e$ to a party $B$.
%
$B$ mixes $e$ and $r$ with its secret static
long-term DH key $s$ to obtain a temporary key, which
%
is used to compute a MAC that is returned to $A$.
%
This authenticates $B$ to $A$, which is sufficient for \mOptls, 
because it only considers server authentication, where the server acts as $B$
and the client as $A$.
%
The temporary key also affects the final session key.
%

In \mEdhoc, a party using the \mStat{} authentication method
essentially acts as an \mOptls{} client and the peer as an \mOptls{}
server.
%
However, unlike \mOptls{}, \mEdhoc{} requires mutual
authentication~\cite{ietf-lake-reqs-04}.
%
\mStatStat{} can be thought of as interleaving two \mOptls{}
runs, one initiated by each party, and then combining the resulting session key
material.
%
Also, \mOptls{} uses the conservative security approach where $e$ and $r$
are distinct elements, whereas \mEdhoc{} reuses the $e$ as the challenge $r$.
%
The latter can only weaken security.\\
%

%------------------------------------------------------------------------- sub
%\subsubsection{\mNoise{}.}
\runhead{\mNoise{}}
The \mStatStat{} and \mPskPsk{} methods follow the \mNoise{}
framework~\cite{perrin2016noise}.
%
\mNoise{} defines how to construct AKE protocols, using static long-term DH
keys or pre-shared keys as credentials.
%
Its goal is to use so-called patterns to harmonize the many static long-term
key DH-based AKE protocols, of which \mOptls{} is one.
%
\mNoise{} specifies how to derive keys, use transcript hashes to ensure
authentication of the message flow etc.
%
The \mNoise{} pattern closest to the \mStatStat{} method is called XX.
%
However, \mEdhoc{} does not use the functions of \mNoise{} exactly as
prescribed.
%
It is thus not obvious that \mEdhoc{} automatically inherits
the properties enjoyed by XX.
%
For example, according to \mNoise{}, the mixHash function is called on
pre-message public keys during the initialization.
%
This would correspond to \mEdhoc{} computing a first intermediate transcript
hash over those keys, which \mEdhoc{} does not do.
%

%------------------------------------------------------------------------- sub
\subsection{Methods of \mEdhoc{}}
\label{sec:methods}
Although \mCbor{} and \mCose{}~\cite{rfc8152} are important building blocks of
\mEdhoc{}, and we use their terminology in our \mTamarin{} model to better
reflect the \mSpec{}, our model does not cover the details of encoding and
\mCose{} interfaces.
%
Thus, we omit details about \mCbor{} and \mCose{}. Similarly, we refer the reader to~\cite{aead} for details about authenticated encryption with associated data (AEAD).
%
All methods make use of a common key hierarchy, which we describe first.
%
\\

%------------------------------------------------------------------------- sub
%\subsubsection{Key Hierarchy.}
\runhead{Key Hierarchy}
\label{sec:keyHierarchy}
\mEdhoc{} derives keys using \mHkdf{}~\cite{rfc5869}.
%
The \mHkdf{} interface provides two functions -- \mHkdfExtract{}, which
constructs uniformly distributed key material from random input and a seed,
and \mHkdfExpand{}, which generates keys from key material and a seed.
%
The seeds are used to bind the key material and keys to certain parameters.
%
Derived keys (\mKtwoe, \mKtwom{}, \mKthreeae, and \mKthreem) are then used as
input to encryption and integrity protection algorithms.
%

The key hierarchy is rooted in the ephemeral DH key \mGxy{}, the combination
of the two half-keys \mGx{} and \mGy{}.
%
From this, keys are successively constructed with each transmitted message,
weaving in information parties learn as the protocol progresses, culminating
in the established session key material.
%
From the session key material, a key exporter (\mEdhoc-Exporter) based on
\mHkdf{} can be used to extract the encryption and integrity keys required
for \mOscore{}.
%
Figure~\ref{fig:kdfdiagram} shows an abstract description of the key hierarchy.
%

From \mGxy{}, three intermediate keys \mPRKtwo, \mPRKthree{} and
\mPRKthree{} are derived.
%
Each one of them corresponds to a specific message in the protocol, and from
these intermediate keys, encryption and integrity keys for that message are
derived.
%
For \mPskPsk{}, the pre-shared key is used as seed for \mPRKtwo.
%
For all other methods the seed for \mPRKtwo{} is empty.
%
In the \mPskPsk{} and \mSigSig{} methods, all three intermediate keys
are the same, i.e., $\mPRKtwo{} = \mPRKthree{} = \mPRKfour$.
%
For the three \mStat-based methods, however, they differ.
%
This is because the intermediate key \mPRKthree{} is dependent on the static
long-term key of the party (or parties, if both) using \mStat{}-based
authentication.
%

\begin{figure}[!h]
\scalebox{.75}{
% !TEX root =  edhocProtocol.tex
% Start the picture
\begin{tikzpicture}[%
    >=latex,              % Nice arrows; your taste may be different
    start chain=going below,    % General flow is top-to-bottom
    node distance=10mm and 60mm, % Global setup of box spacing
    every join/.style={norm},   % Default linetype for connecting boxes
    ]
% ------------------------------------------------- 
% A few box styles 
% <on chain> *and* <on grid> reduce the need for manual relative
% positioning of nodes
\tikzset{
terminput/.style={rounded corners, text width=6em},
term/.style={rounded corners},
  base/.style={draw, thick, on chain, on grid, align=center, minimum height=6ex},
  prk/.style={base, draw=Red3, fill=Red3!25, rectangle, text width=4em},
  %hkdfext/.style={base, draw=Green3, fill=Green3!25, isosceles triangle, isosceles triangle apex angle=60, anchor=base, shape border rotate=-90, text width=6em},
  hkdfext/.style={base, draw=Green3, fill=Green3!25, rectangle, text width=8em},
  %hkdfexp/.style={base, draw=orange, fill=orange!50, isosceles triangle, isosceles triangle apex angle=60, anchor=base, shape border rotate=-90, text width=6em},
  hkdfexp/.style={base, draw=orange, fill=orange!50, rectangle, text width=8em},
  keyb/.style={base, draw=Blue3, fill=Blue3!25, rectangle, text width=4em},
  % -------------------------------------------------
  norm/.style={->, draw, Blue3},
}
% -------------------------------------------------
% Start by placing the nodes
\node [prk] (p0) {$g^{xy}$};
\node [hkdfext, join] (h1) {\mHkdfExtract};
\node [prk, join] (p2) {\mPRKtwo};
\node [hkdfext, join] (h3) {\mHkdfExtract};
\node [prk, join] (p3) {\mPRKthree};
\node [hkdfext, join] (h5) {\mHkdfExtract};
\node [prk, join] (p4) {\mPRKfour};

\node [hkdfexp, shape border rotate=180, left= 3cm of p4] (h6) {\mHkdfExpand};
\node [keyb, join, left=3cm of h6] (k3) {\mKthreem};

\node [hkdfexp, shape border rotate=180, left= 3cm of p3] (h4) {\mHkdfExpand};
\node [keyb, join, left=3cm of h4] (k2) {\mKtwom};

\node [hkdfexp, shape border rotate=180, left= 3cm of p2] (h2) {\mHkdfExpand};
\node [keyb, join, left=3cm of h2] (k1) {\mKtwoe};

\node [hkdfexp, shape border rotate=180, below= 1cm of h4] (h8) {\mHkdfExpand};
\node [keyb, below=1cm of k2] (k2b) {\mKthreeae};
%\node [draw=none, fill=none, below=0.5cm of h4] (dummy) {};

\draw [->, norm] (p3.south) -- ++(0,-0.58) -- (h8);
\draw [->, norm] (h8) -- (k2b);
\draw [->, norm] (p2) -- (h2); 
\draw [->, norm] (p3) -- (h4); 
\draw [->, norm] (p4) -- (h6);
%\node [term, fill=Green3!25, below = -1.3cm of h1] (h1name) {\mHkdfExtract};
%\node [term, left = 1cm of h1name] (u1) {H1};
%\draw [->, dotted, shorten >=1mm] (u1) -- (h1name);
\node [terminput, right = 1cm of h1] (u1) {Initial salt};
\draw [->, dotted, shorten >=1mm] (u1) -- (h1);

%\node [term, fill=Green3!25, below = -1.3cm of h3] (h3name) {\mHkdfExtract};
%\node [term, left = 1cm of h3name] (u2) {H2};
%\draw [->, dotted, shorten >=1mm] (u2) -- (h3name);
\node [terminput, right = 1cm of h3] (u2) {Additional input};
\draw [->, dotted, shorten >=1mm] (u2) -- (h3);

\node [terminput, right = 1cm of h5] (u3) {Additional input};
\draw [->, dotted, shorten >=1mm] (u3) -- (h5);


%\node [term, fill=orange!50, below = -0.9cm of h2] (h2name) {\mHkdfExpand};
%\node [term, above = 1cm of h2name] (u3) {H3};
%\draw [->, dotted, shorten >=1mm] (u3) -- (h2);
\node [term, above = 1cm of h2] (u4) {\mTHtwo};
\draw [->, dotted, shorten >=1mm] (u4) -- (h2);

%\node [term, fill=orange!50, below = -0.9cm of h4] (h4name) {\mHkdfExpand};
%\node [term, above = 1cm of h4name] (u5) {H5};
%\draw [->, dotted, shorten >=1mm] (u5) -- (h4);
\node [term, above = 1cm of h4] (u5) {\mTHtwo};
\draw [->, dotted, shorten >=1mm] (u5) -- (h4);

\node [term, above = 0.7cm of h6] (u6) {\mTHthree};
\draw [->, dotted, shorten >=1mm] (u6) -- (h6);
\draw [->, dotted, shorten >=1mm] (u6) -- (h8);
%
% ------------------------------------------------- 
% 
%\path (h2.east) to node [near start, yshift=1em] {$n$} (c3); 
%  \draw [o->,lccong] (h2.east) -- (p8);
%\path (p3.east) to node [yshift=-1em] {$k \leq 0$} (c4r); 
%  \draw [o->,lcnorm] (p3.east) -- (p9);
% -------------------------------------------------
% 
%\path (h1.east) to node [near start, yshift=1em] {$n$} (c1); 
%  \draw [o->,lcfree] (h1.east) -- (c1) |- (p4);
%\path (h3.east) -| node [very near start, yshift=1em] {$n$} (c1); 
%  \draw [o->,lcfree] (h3.east) -| (c1);
%\path (p3.west) to node [yshift=-1em] {$k>0$} (c4); 
%  \draw [*->,lcnorm] (p3.west) -- (c4) |- (p3);
%\path (h4.east) -| node [very near start, yshift=1em] {$n$} (c6); 
%  \draw [o->,lcfree] (h4.east) -| (c6); 
%\path (h5.east) to node [near start, yshift=1em] {$n$} (c6); 
%  \draw [o->,lcfree] (h5.east) -| (c7); 
%\path (p4.east) to node [yshift=-1em] {$k \leq 0$} (c7); 
%  \draw [o->,lcfree] (p4.east) -- (c7)  |- (p9);
%\path (p4.west) to node [yshift=-1em] {$k>0$} (c5); 
%  \draw [*->,lcfree] (p4.west) -- (c5) |- (p5);
% -------------------------------------------------
% A last flourish which breaks all the rules
%\draw [->,MediumPurple4, dotted, thick, shorten >=1mm]
%  (p9.south) -- ++(5mm,-3mm)  -- ++(27mm,0) 
%  |- node [black, near end, yshift=0.75em, it]
%    {(When message + resources available)} (p0);
% -------------------------------------------------
\end{tikzpicture}
% =================================================

}
\caption{Joint key hierarchy for all methods}
\label{fig:kdfdiagram}
\end{figure}

We now describe the \mEdhoc{} methods.
%
Since the \mSigSig{} method has been analyzed earlier, we do not focus on
it here.
%

%\subsubsection{\mPskPsk{}.}
In this method, the initiator and responder are assumed to share a pre-shared
key (\mPsk) identified by \mIDPsk.
%
The message sequence is shown in Figure~\ref{fig:edhocpsk}.
%
The first message contains the \mIDPsk{} identifier.
%
The responder obtains the corresponding \mPsk{} and uses that to construct the
authentication data, i.e., the MAC in the \mAead{} transform), included in
the second message.
%
n the third message, the initiator uses this same technique to authenticate
to the responder.
%
The transcript hash \mTH{} keeps track of the data sent in the messages.
%
One unconventional feature is that the transcript hash lags behind by one
message.
%
That is, the second transcript hash \mTHtwo{}, does not cover the content of
the second message (similarly for \mTHthree).
%
The reason for this is that \mTHtwo{} cannot cover the output of the
\mAead{} transform, since \mTHtwo{} is itself an input to it.
%
This is the same for all methods, but it does not cause any problems with
authentication, since the data not covered by the transcript hash is
instead MACed or signed.
%

\begin{figure}
\scalebox{.69}{
\tikzset{>=latex, every msg/.style={draw=thick}, every node/.style={fill=none,text=black}}
\begin{tikzpicture}
    \node (ini) at (0, 0) {Initiator};
    \draw [very thick] (0, -0.5) -- (0,-10.1);
    \draw [very thick] (11.5, -0.5) -- (11.5,-10.1);
    \node[below=0.5em of ini,fill=white] () {Knows $g,\ \mPsk,\ \mIDPsk,\ \mADone,\ \mADthree$};
    \node (res) at (11.5,0) {Responder};
    \node[below=0.5em of res,fill=white] () {Knows $g,\ \mPsk,\ \mIDPsk,\ \mADtwo$};
    \action{3em}{ini}{Generates \mMethod,\ \mSuites,\ \mCi,\ $x$\\$\mGx = g^{x}$};
    \msg{6.5em}{ini}{res}{\mMsgone: \mMethod, \mSuites, \mGx, \mCi, \mIDPsk, \mADone};
    \action{7em}{res}{$
      \begin{array}{c}
        \text{Generates } \mCr,\ $y$\\
        \mGy = g^{y}\\
        \mTHtwo = \mHash(\mMsgone, g^{y})\\
        \mPRKtwo = \mHkdfExtract(\mPsk, g^{xy}) \\
        \mKtwoae = \mHkdfExpand(\mPRKtwo, \mTHtwo)
      \end{array}$};
    \msg{15.5em}{res}{ini}{\mMsgtwo: \mCi, \mGy, \mCr, $\overbrace{\mAead(\mKtwoae; \mTHtwo, \mADtwo)}^{\mCipher}$};
    \action{16.5em}{ini}{$
      \begin{array}{c}
       \mTHtwo = \mHash(\mMsgone, \mGy)\\
       \mPRKtwo = \mHkdfExtract(\mPsk, g^{xy}) \\
        \mKtwoae = \mHkdf(\mPRKtwo, \mTHtwo)\\
        \mTHthree = \mHash(\mTHtwo, \mCipher)\\
        \mPRKthree = \mPRKtwo \\
        \mKthreeae = \mHkdfExpand(\mPRKthree, \mTHthree)
      \end{array}$};
    
    \msg{25.5em}{ini}{res}{\mMsgthree: \mCr, \mAead(\mKthreeae; \mTHthree; \mADthree)};
    \action{26em}{res}{$
    \begin{array}{c}
    	\mPRKthree = \mPRKtwo \\
        \mKthreeae = \mHkdfExpand(\mPRKthree, \mTHthree)
    \end{array}$};
    \draw [line width=1mm] (-2,-10.1) -- (2,-10.1);
    \draw [line width=1mm] (9.5,-10.1) -- (13.5,-10.1);
    \end{tikzpicture}}
\caption{The \mPskPsk{} method; $\mAead(k; x; y)$ is used to denote
\mAead{} encryption where $k$ is the key, $x$ is associated data to be integrity
protected, and $y$ is the plaintext}
\label{fig:edhocpsk}
\end{figure}
%

%------------------------------------------------------------------------- sub
\subsubsection{\mStat-based Methods.}
\mEdhoc{} provides three \mStat-based methods -- \mSigStat{}, \mStatStat{} and
\mStatSig{}.
%
One or both of the initiator and responder authenticate to their peer using
their secret static long-term DH key, \mLtki{} or \mLtkr{} respectively.
%
Authentication of the party using the \mStat{} method follows the
challenge-response signature pattern from \mOptls.
%
In the same terms as Section~\ref{sec:optls}, the ephemeral DH half-key of the
peer takes the place of both $r$ and $e$, i.e., $r=e$, and $s$ is the secret
static long-term DH key of the party to be authenticated.
%
Assuming the discrete log problem is hard in the underlying DH group,
only the initiator and responder will be able to compute $e^s$.
%
That value is then affecting the computation of \mAuthr{} or \mAuthi{}
(depending on who is the authenticator), and it is also
woven in to the key hierarchy, see
Figures~\ref{fig:kdfdiagram} and \ref{fig:edhocsigstat}.
%
The receiver of \mAuthi{}/\mAuthr{} assumes that only the responder knows
the corresponding $e^s$, and therefore considers the peer authenticated if
the \mAead{} transform completes successfully.\\
%
In contrast to the \mPskPsk{} method, in \mStat{}-based methods, each party provides an identifier, \mIdcredi{} and \mIdcredr{} respectively, for the credentials they use.
%
\\

%------------------------------------------------------------------------- sub
%\subsubsection{\mSigStat.}
\runhead{\mSigStat}
The initiator uses the \mSig{} method, authenticating with a signature,
while the responder uses the \mStat{} method, authenticating with a
challenge-response signature.
%
The detailed message sequence chart is shown in Figure~\ref{fig:edhocsigstat}.
%
\mCredi{} and \mLtki{} must be signature keys since the initiator is using the
\mSig{} method. \\
%

\begin{figure}[h]
\centering
\scalebox{.7}{
\tikzset{>=latex, every msg/.style={draw=thick}, every node/.style={fill=none,text=black}}
\begin{tikzpicture}
    \node (ini) at (0, 0) {Initiator};
    \draw [very thick] (0, -0.5) -- (0,-14.1);
    \draw [very thick] (9, -0.5) -- (9,-14.1);
    \node[below=0.5em of ini,fill=white] {$
    \begin{array}{c}
    \text{Knows}\ $g$,\ \mCredi,\ \mLtki,\ \mIdcredi,\\
    \mIdcredr, \mADone,\ \mADthree
    \end{array}
    $};
    \node (res) at (9,0) {Responder};
    \node[below=0.5em of res,fill=white] {$
    \begin{array}{c}
    \text{Knows}\ $g$,\ \mCredr,\ \mLtkr, \ \mIdcredr,\\
    \mIdcredi, \mADtwo
    \end{array}$};
    \action{4.5em}{ini}{Generates \mMethod,\ \mSuites,\ \mCi,\ $x$\\$\mGx = g^{x}$};
    \msg{9em}{ini}{res}{\mMsgone: \mMethod, \mSuites, \mGx, \mCi, \mADone};
    \action{9.5em}{res}{$
      \begin{array}{c}
        \text{Generates } \mCr,\ $y$\\
        \mGy = g^{y}\\
        \mTHtwo = \mHash(\mMsgone, \langle \mCi, \mGy, \mCr \rangle)\\
        \mPRKtwo = \mHkdfExtract(\textrm{``\phantom{}''}, g^{xy}) \\
        \mGrx = \mGx^{\mLtkr} \\
        \mPRKthree = \mHkdfExtract(\mPRKtwo, \mGrx) \\
        \mKtwom = \mHkdfExpand(\mPRKthree, \mTHtwo) \\
        \mMactwo = \mAead(\mKtwom; \langle \mIdcredr, \mTHtwo, \mCredr, \mADtwo \rangle; \textrm{``\phantom{}''}) \\
        \mKtwoe = \mHkdfExpand(\mPRKtwo, \mTHtwo)
      \end{array}$};
    \msg{24em}{res}{ini}{\mMsgtwo: \mCi, \mGy, \mCr, $\overbrace{\mKtwoe\ \mXor\ \langle \mIdcredr, \mMactwo, \mADtwo \rangle}^{\mCipher}$};
    \action{25em}{ini}{$
      \begin{array}{c}
        %\mTHtwo = \mHash(\mMsgone, \langle \mCi, \mGy, \mCr \rangle) \
        \mPRKtwo = \mHkdfExtract(\textrm{``\phantom{}''}, g^{xy}) \\
        %\mKtwoe = \mHkdfExpand(\mPRKtwo,\mTHtwo)\\
        \mGrx = \mCredr^{x} \\
        \mPRKfour = \mPRKthree = \mHkdfExtract(\mPRKtwo, \mGrx) \\
        %\mKtwom = \mHkdfExpand(\mPRKthree, \mTHtwo) \\
        \mKthreeae = \mHkdfExpand(\mPRKthree, \mTHtwo) \\
        \mTHthree = \mHash(\mTHtwo, \mCipher, \mCr)\\
        \mKthreem = \mHkdfExpand(\mPRKfour, \mTHthree) \\
        \mMacthree = \mAead(\mKthreem; \langle \mIdcredi, \mTHthree, \mCredi, \mADthree \rangle; \textrm{``\phantom{}''}) \\
        \mSigthree = \mSign(\mLtki; \langle \mIdcredi, \mTHthree, \mCredi, \mADthree \rangle, \mMacthree \rangle)
      \end{array}$};
    \msg{37em}{ini}{res}{$\mMsgthree: \mCr, \mAead(\mKthreeae; \mTHthree; \langle \mIdcredi, \mSigthree, \mADthree \rangle$)};
    \action{37.5em}{res}{$
    \begin{array}{c}
        \mTHthree = \mHash(\mTHtwo, \mCipher, \mCr)\\
        \mKthreem = \mHkdfExpand(\mPRKthree, \mTHthree) \\
        \mKthreeae = \mHkdfExpand(\mPRKthree, \mTHthree)
    \end{array}$};
    \draw [line width=1mm] (-2,-14.1) -- (2,-14.1);
    \draw [line width=1mm] (7,-14.1) -- (11,-14.1);
    \end{tikzpicture}}
    \caption{The \mSigStat{} method; (\mCredi{}, \mLtki), and
        (\mCredr{}, \mLtkr{}) are public-private key pairs; $\mSign(k; x)$ is
    used to denote the signing of message $x$ using key $k$}
\label{fig:edhocsigstat}
\end{figure}
%

%------------------------------------------------------------------------- sub
\runhead{\mStatSig}
This method mirrors the \mSigStat{} method.
%
The responder runs \mSig{} and creates \mAuthr{} as a signature over MAC,
while the initiator runs \mStat{}.
%
This is illustrated in Figure~\ref{fig:edhocstatsig}. \mCredr{} and \mLtkr{}
must be signature keys.\\
%

\begin{figure}[h]
\centering
\scalebox{.7}{
\tikzset{>=latex, every msg/.style={draw=thick}, every node/.style={fill=none,text=black}}
\begin{tikzpicture}
    \node (ini) at (0, 0) {Initiator};
    \draw [very thick] (0, -0.5) -- (0,-14.2);
    \draw [very thick] (9, -0.5) -- (9,-14.2);
    \node[below=0.5em of ini,fill=white,text=black] {$
    \begin{array}{c}
    \text{Knows}\ $g$,\ \mCredi,\ \mLtki,\ \mIdcredi,\\
    \mIdcredr, \mADone,\ \mADthree
    \end{array}
    $};
    \node (res) at (9,0) {Responder};
    \node[below=0.5em of res,fill=white] {$
    \begin{array}{c}
    \text{Knows}\ $g$,\ \mCredr,\ \mLtkr,\ \mIdcredr,\\
    \mIdcredi, \mADtwo
    \end{array}$};
    \action{4.5em}{ini}{Generates \mMethod,\ \mSuites,\ \mCi,\ $x$\\$\mGx = g^{x}$};
    \msg{8.5em}{ini}{res}{\mMsgone: \mMethod, \mSuites, \mGx, \mCi, \mADone};
    \action{9.5em}{res}{$
      \begin{array}{c}
        \text{Generates } \mCr,\ $y$\\
        \mGy = g^{y}\\
        \mTHtwo = \mHash(\mMsgone, \langle \mCi, \mGy, \mCr \rangle)\\
        \mPRKthree = \mPRKtwo = \mHkdfExtract(\textrm{``\phantom{}''}, g^{xy}) \\
        \mKtwom = \mHkdfExpand(\mPRKthree, \mTHtwo) \\
        \mMactwo = \mAead(\mKtwom; \langle \mIdcredr, \mTHtwo, \mCredr, \mADtwo \rangle; \textrm{``\phantom{}''}) \\
        \mSigtwo = \mSign(\mLtkr; \langle \langle \mIdcredr, \mTHtwo, \mCredr, \mADtwo \rangle, \mMactwo \rangle)\\
        \mKtwoe = \mHkdfExpand(\mPRKtwo, \mTHtwo)
      \end{array}$};
    \msg{23em}{res}{ini}{\mMsgtwo: \mCi, \mGy, \mCr, $\overbrace{\mKtwoe\ \mXor\ \langle \mIdcredr, \mSigtwo, \mADtwo \rangle}^{\mCipher}$};
    \action{24em}{ini}{$
      \begin{array}{c}
        \mPRKthree = \mPRKtwo = \mHkdfExtract(\textrm{``\phantom{}''}, g^{xy}) \\
        \mGiy = \mGy^{\mLtki} \\
        \mPRKfour = \mHkdfExtract(\mPRKthree, \mGiy) \\
        \mTHthree = \mHash(\mTHtwo, \mCipher, \mCr)\\
        \mKthreem = \mHkdfExpand(\mPRKfour, \mTHthree) \\
        \mMacthree = \mAead(\mKthreem; \langle \mIdcredi, \mTHthree, \mCredi, \mADthree \rangle; \textrm{``\phantom{}''}) \\
        \mKthreeae = \mHkdfExpand(\mPRKthree, \mTHthree) \\
      \end{array}$};
    \msg{34.7em}{ini}{res}{$\mMsgthree: \mCr, \mAead(\mKthreeae; \mTHthree; \langle \mIdcredi, \mMacthree, \mADthree \rangle$)};
    \action{35.2em}{res}{$
    \begin{array}{c}
       \mGiy = \mCredi^{y} \\
       \mPRKfour = \mHkdfExtract(\mPRKthree, \mGiy) \\
       \mTHthree = \mHash(\mTHtwo, \mCipher, \mCr)\\
        \mKthreem = \mHkdfExpand(\mPRKfour, \mTHthree) \\
        \mKthreeae = \mHkdfExpand(\mPRKthree, \mTHthree)
    \end{array}$};
    \draw [line width=1mm] (-2,-14.2) -- (2,-14.2);
    \draw [line width=1mm] (7,-14.2) -- (11,-14.2);
    \end{tikzpicture}}
\caption{The \mStatSig{} method of \mEdhoc}
\label{fig:edhocstatsig}
\end{figure}
%

\runhead{\mStatStat}
Both the initiator and the responder use the \mStat{}
authentication method.
%
Both parties' secret static long-term DH keys feed into the key hierarchy.
%
The initiator computes authentication data using the \mAead{} transform
and includes that in the third message for the responder to verify.
%
\\

%------------------------------------------------------------------------- sub
\runhead{\mSigSig{}}
Both parties run \mSig{} and authenticate with signatures.
%
As mentioned earlier, \mSigSig{} is very closely modeled on \mSigmaI{}, but
there are some notable differences.
%
For example, \mEdhoc{} aims to provide some degree of identity protection for
responders, and therefore uses the idea from \mSigmaI{} of encrypting the
responder identifier \mIdcredr{} (and other items) in the second message.
%
The designers of \mEdhoc{} consider it wasteful adding a second MAC in addition
to the already included \mMactwo{}, to limit bandwidth consumption.
%
\mEdhoc{} applies XOR encryption, with \mHkdf{} being used to generate the
key stream, whereas \mSigma{} assumes authenticated encryption for this
purpose (see Section 5.2 of~\cite{sigma}).
%
Because \mSigSig{} has been analyzed in~\cite{DBLP:conf/secsr/BruniJPS18}, we
do not focus on it here.
%
However, we model it and verify its properties as carefully as for the other methods.


%------------------------------------------------------------------------- sub
\subsection{Negotiating a cipher suite and method and correlation parameters}
\label{sec:ciphersuite}
Recall that we mentioned that the first message contains a list of cipher suites, ranked according to the preference of the initiator. What does a cipher suite actually contain? An \mEdhoc{} cipher suite consists of an ordered set of \mCose{} algorithms: an \mAead{} algorithm, a hash algorithm, an ECDH curve, a signature algorithm, a signature algorithm curve, an application \mAead{} algorithm, and an application hash algorithm from the \mCose{} Algorithms and Elliptic Curves registries.  

There are four supported cipher suites in \mEdhoc{} -- we refer the reader to Section 3.4 of~\cite{selander-lake-edhoc-01} for the specifics of the algorithms allowed therein. Each cipher suite is identified by one of four predefined integer labels (0--3). Some algorithms are not used in some methods.  The signature algorithm and the signature algorithm curve are not used in methods without signature authentication (i.e. in \mPskPsk{} and \mStatStat).

In order to keep the presentation clean, we have omitted the cipher suite negotiation process from the description of the methods. However, this process happens as follows, at the beginning of every method, once the responder receives the first message. The initiator proposes an ordered list of cipher suites they support. This list presented in descending order to the responder who either accepts the topmost entry in this list (if they also support that suite) or makes a counter-proposal, namely the topmost entry which they support from the remaining part of the list. If there is no such entry the responder can reject, and the protocol does not continue. Similarly, the responder can reject the initiator's choices for the method and correlation parameters as well -- in the case of a reject for either of these values, the protocol aborts.

%------------------------------------------------------------------------- sub
\subsection{Claimed Security Properties}
\label{sec:claimedProperties}
 
We present the security properties that the authors
of the \mSpec{}~\cite{selander-lake-edhoc-01} claim \mEdhoc{} satisfies.
%
Many of the claims are imprecisely expressed.
%
We will revisit these claims when we discuss the formal modeling and
verification of \mEdhoc{} in Section~\ref{sec:formalization} and in the
discussions in Section~\ref{sec:discussion}.
%
These are our interpretations of the claimed security properties described in
the \mSpec~\cite{selander-lake-edhoc-01}, Section~8.1:
\begin{itemize}
    \item Perfect Forward Secrecy (\textbf{PFS}) for the session key material
    \item Mutual authentication (this presumably refers to entity authentication
        because it is followed by claims of
        \textbf{consistency} (defined in~\cite{sigma}),
        \textbf{aliveness}, and
        \textbf{peer awareness} to the responder, but not to the initiator)
    \item \textbf{Identity protection} (the initiator against active attacks
        and the responder against passive attacks, except for \mPskPsk{})
    \item Key Compromise Impersonation (\textbf{KCI}) resistance
    \item A single session of \mEdhoc{} enables the responder to verify
            that the selected cipher suite is the most preferred of the
            initiator which is supported by both parties, even though there is
            no negotiation of cipher suites per se.
    \item \textbf{Session key independence}
\end{itemize}

